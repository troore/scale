
\section{Implementing Clef Declarations in C}
\label{sec:decls}

In this section we describe how to implement Clef declarations in C
and C++.

Generating declarations is not completely straightforward since
different languages have different rules concerning where declarations
are allowed.  In C, In C, a declaration must appear at the beginning
of a block (\ie after a curly bracket \{).  In contrast, C++ does not
restrict declarations to the beginning of a block.

Clef also allows declarations to appear within a group of statements
using the \textbf{StmtDeclStmt} generation interface routine.
Translating Clef declarations into C++ is very simple - we generate a
declaration for each \textit{DeclStmt} node.  Translating Clef
declarations into C is more difficult.  Clef2C must generate all
declarations at the beginiing of a block.  In order to facilitate
generating C style declarations, the Clef tree must maintain a link
from each Clef block node to the block's scope.  We traverse the
block's scope to generate declarations.

\subsection{Declaring Types}

Clef has two methods of declaring types, \code{DeclType} and
\code{NameType}.  We generate a \key{typedef} statement for 
\code{NameType} nodes.  For \code{DeclType} nodes, we just
generate the declaration.


\subsection{Array Declarations}

We represent a Fortran multiple dimension array as a single dimension
in C.  The size of the array is the product of the array dimension
sizes (\eg a(10,10) is a[100]).  We have to perform this translation
since Fortran arrays are laid out in column major order (\ie traverse
down the column in sequence) and C arrays are laid out in row major
order (\ie traverse down the row in sequence).

Generating C code for array subscripts is tricky since we declare
multi-dimension arrays as a single dimension.  See
Section~\ref{arraysubs} for details.

\subsection{Exception Declarations}

We don't do anything with exceptions.

\subsection{Opaque Declarations}

We don't do anything with opaque declarations.

\subsection{Formal Declarations}

We assume each formal parameter is passed by reference.  We
do not do anything special for other types of parameters.

\subsection{Field Declarations}

A field is a component of an aggregate data structure.  Clef allows an
initial expression for the field.  In C, an initial expression may
only be assigned to fields of an enumeration.

If a field has an initial expression, we generate code for it.
We do not check if the field is part of an enumeration.

\subsection{Constant Declarations}

We use the \texttt{const} keyword for constant declarations.

\subsection{Method Declarations}

We do not generate any code for method declarations.

\subsection{Override Declarations}

We do not generate any code for override declarations.  An override is
a Modula-3 construct.

\subsection{Type Conversion Declarations}

We do not generate any code for type conversion declarations.

\subsection{Initializers}

We do not generate code for initializers.  An initializer
is a C++ construct.

\subsection{Friend Declaration}

Nope.

\subsection{Namespace Declaration}

Nope.

\subsection{Import/Exports}

Nope.

\subsection{Entry Declaration}

An entry declaration represents a Fortran \key{Entry} statement which
allows a procedure to multiple entry points.  We currently do not
generate any code for an entry declaration.

Generating C code for \key{entry} statement is very difficult.  In
order to implement this feature, we really need to think about how the
entry declaration is represented in the Clef tree.  Both \textbf{f2c}
and \textbf{EDG} generate a completely new routine for enclosing
procedures which contain \key{entry} statements.  The new routine
accepts an additional parameter which indicates where execution
begins.  A \key{switch} statement transfers control to the appropriate
entry point based upon the value of the addition parameter.
Separate routines are created for the enclosing procedure and
any entry point.  The only function performed by these routines
are to call the new routine with the appropriate parameter which
indicates where the program begins execution.

For example, given the program:
\begin{codeseq}
\>\>integer a,b,c \\
\>\>call sub(a,b)\\
\>\>call sube(c)\\
\>\>end \\
\>\>subroutine sub(a,b) \\
\>\>integer a,b,d \\
\>\>a = b \\
\>\>entry sube(d)\\
\>\>b = d \\
\>\>return \\
\>\>end\\
\end{codeseq}

The \textbf{f2c} compiler generates three functions (besides main),
\code{sub\_(a, b)}, \code{sube\_(d)}, and \code{sub\_0\_(n\_\_, a, b, d)}.  
The routines \code{sub\_} and \code{sube\_} contain a call to 
\code{sub\_0\_} which different values for $n$ and different parameters.
For example, in \code{sube\_} the call is \code{sub\_0\_(1, 0, 0, d)}
and in \code{sub\_} the call is \code{sub\_0\_(0, a, b, 0)}.

\subsection{Common Blocks}

Clef2C does not need to do anything special for Fortran 77 common
blocks.  The Fortran front end (\ie the tool which translate EDG to
Clef) is responsible for creating the appropriate data structure for
the common memeory locations.  The front end is also responsible for
creating the appropriate Clef nodes for references to variables
located in common blocks.
