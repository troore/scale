
\section{Implementing Clef Types in C}

In this section we describe how to implement Clef types in C.  Table
2.2 in the Clef Design Document specifies the type constructor
correspondences between C++, Modula-3, and Fortran 77.  Many of these
types have corresponding C operators.  However, there are several
types that do not.

Table~\ref{tab:clef2ctypes} shows the correspondence between Clef types and C
operators.  For several of the Clef operators, especially those that
do not directly translate into C types, we provide more details about
generating C code.
\begin{table}
\begin{center}
\begin{tabular}{|l|c|} \hline \hline
Clef Type & C Type \\ \hline \hline
Clef\_PrimitiveCharType & \key{char} \\ \hline
Clef\_PrimitiveSymbolicIntType  & \key{short}, \key{int}, \key{long} \\
Clef\_PrimitveIntType & \key{unsigned short}, \key{unsigned int}, \key{unsigned long} \\ \hline
Clef\_PrimitiveFixedPoint & No C operator \\ \hline
Clef\_PrimitiveSymbolicRealType & \\ 
Clef\_PrimitiveRealType & \rb{\key{float}, \key{double}} \\ \hline
Clef\_PrimitveComplexType & No C operator \\ \hline
Clef\_PrimitiveVoid & \key{void} \\ \hline
Clef\_PrimitiveArrayType & [] \\ \hline
Clef\_BooleanType & \key{int} \\ \hline
Clef\_EnumerationType & \key{enum} \\ \hline
Clef\_RecordType & \key{struct} \\ \hline
Clef\_UnionType & \key{union} \\ \hline
Clef\_ClassType & No C operator \\ \hline
Clef\_IncompleteType & See Below \\ \hline
Clef\_PointerType & * \\ \hline
Clef\_IndirectType & \& \\ \hline
Clef\_Offset & TBD \\ \hline
Clef\_SetType & No C operator \\ \hline
Clef\_RangeType & \textit{Use Base Type} \\ \hline
Clef\_BrandedType & No C operator \\ \hline
Clef\_PackedType & \textit{Use Base Type} \\ \hline
Clef\_AliasType & \& \\ \hline
\end{tabular}
\caption{Corresponding C Types for Clef Types}
\label{tab:clef2ctypes}
\end{center}
\end{table}

\subsection{Type Attributes}

The Clef Design Document defines several type attributes,
\textbf{cTraced}, \textbf{cUntraced}, \textbf{cConstantType}, 
\textbf{cVolatile}, \textbf{cOrdered}, and \textbf{cUnordered}.

We ignore the \textbf{cTraced} and \textbf{cUntraced} attributes since
they are specific to Modula-3.  We also ignore the \textbf{cOrdered}
and \textbf{cUnordered} attributes.

We generate a \key{const} and \key{volatile} type specifier for the 
\textbf{cConstantType} and \textbf{cVolatile}, respectively.

\subsection{Primitive Types}

Clef specifies the size of primitive types either symbolically
or as a bit size.  Table~\ref{primtypes} shows the mapping
from the symbolic sizes and bit lengths to the C/C++ language
types.
\begin{table}[h]
\begin{center}
\begin{tabular}{llcc}
Symbolic Type & Size (bits) & C Int Type & C Float Type \\
cVeryshort & 8 & char & NA \\
cShort & 16 & short & NA \\
cNormal & 32 & int & float \\
cLong & 64 & long & double \\
cVerylong & 128 & NA & long double \\
\end{tabular}
\end{center}
\caption{Clef Types to C Types}
\label{primtypes}
\end{table}
The mappings for the bit lengths are the default values but they may
be changed.

If the type contains a symbolic value, we use it in preference to the
bit size to generate strings.

\subsection{Fixed Point Types}

The C language does not provide a convienient mechanism to handle
fixed point types.  Since fixed point types occur do not occur in many
language (of the languages we consider, only Ada has a fixed point
type) we do not generate any C code for fixed point types.

\subsection{Complex Types}

We represent complex types as structures containing
a real and imaginary field.  We create two predefined complex 
types in the Clef2C header file:
\begin{codeseq}
typedef struct \{ float r, i \} \_complex; \\
typedef struct \{ double r, i \} \_doublecomplex;
\end{codeseq}

The specific complex type that we generated depends upon the user
specified sizes for the real and imaginary parts of the complex type.

We either generate in-line sequences of code or calls to run-time
system routines for operations involving complex types.
Section~\ref{complexops} describes the details of generating code for
the different kinds of legal complex operations.

\subsection{Character Types}

Fortran contains a \key{character} type which is defined as a string
of characters.  In Clef, a \key{character} type is defined as a fixed
array of characters (\ie \code{TypeArrayFixed}).

Character types assignments are represented in the Clef tree using
\key{AssignFixedString} node which translate into calls to
\code{sassign(dest, src, destl, srcl, padding)}.  The routine \code{sassign}
is defined in the Clef2C library.

\subsection{Logical Types}

Fortran contains a \key{logical} type which may have the values
\key{true} and \key{false}.  We represent a Fortran logical type
as a C \key{int} type.  A value of 0 indicates false and a value
of 1 indicates true (in C, any value but 0 indicates true).

\subsection{Array Types}

Clef has constructs for three kinds of arrays - fixed length arrays,
unconstrained arrays, and open arrays.  Clef arrays contains
an element type and range type (depending on the array kind).

We represent fixed length arrays as C arrays.  The index type
specifies the number of dimensions and the number of elements in each
dimension.  The element type specifies the array type.  Clef
represents the index as a bound, in C/C++, an index is the size of the
array.  We convert the bound to a single value by subtracting the
minimum value from the maximum value of the bound and adding 1.

We represent unconstrained and open array as pointers.  The type of
the pointer is the element type.

Depending on the source language, we generate code to check the bounds
of an array whenever an array is accessed.  \question{Which
languages.}

\subsection{Class Types}

The output we generate for classes depends on the target language.  In
C++, we are able to explicitly represent the classes using the {\tt
class} structure.  In C, we define a class as a structure and we must
translate the object model to C.  We first describe the C
representation of an object and then we discuss the C++
representation.

\subsubsection{C Implementation}
We represent C++ classes as a C \key{struct} type.  We implement the
class members as fields of the \key{struct}.  The order of the
fields in the structure are the same as the order of the members in
the class.

If the class declares virtual functions then we must add an additional
field to the structure to manage the virtual table.  Note that this
field is not added unless the class declares a virtual function.  The
field is added to the end of the structure.  We call this field
\code{\_\_vptr}.  

\subsubsection{C++ Implementation}

The translation from a Clef class to C++ is straight forward.
The ClassType clef node represents a class.  The name
of the class is located in the parent if the parent is
a TypeDecl node.  Otherwise, the class is anonymous.

\subsubsection{The Virtual Table}

Clef does not explicitly represent the virtual function table.
Instead, functions contain an attribute indicating if a the function
is virtual.  Someone\footnote{TBD - not sure yet} must assign numbers
to the virtual functions in a class which can be used as indicies into
the virtual function table.

\subsubsection{Representing Base Classes}

We must also include the members of each base class in the class
structure.  The C++ language does not specify the order in which
storage is allocated for base classes\cite{ellis:90}.  However, we
will include the structure for the immediate base class as the first
field in the structure for the current class.  Then, we will allocate
space for the fields of the current class. For example,
\begin{codeseq}
class A \{ \\
\>int a;\\
\};\\
class B : public A \{\\
\>int b;\\
\};\\
\\
-- The corresponding structure\\
\\
struct A \{\\
\>int a;\\
\};\\
struct B \{\\
\>struct A \_\_A;\\
\>int b;\\
\};\\
\end{codeseq}

\subsection{Branded Types}

Branded types come from Modula-3 (and are useful in languages with
structural equivalence).  Brands are applied to reference types. 
The Modula-3 standard states:
\begin{quote}
Brands distinguish types that would otherwise be the same; they have
no other semantic effect. All brands in a program must be distinct.
\end{quote}

We do nothing special for branded types.

\subsection{Incomplete Types}

The C language only allows a forward reference for a structure, union,
and enumeration tag (and a statement label).  We generate
forward references for these types.  We raise an exception
for all other incomplete types (if there are any - I'm not sure
when this might occur).

\subsection{Offset}

An offset represents a pointer to member in C++.  The Annotated
C++ Reference Manual states on page 158:
\begin{quote}
Pointers to members must be implemented in a way that is
consistent with the layout of class objects and the implementation
of multiple inheritance.
\end{quote}
The Annotated C++ Reference Manual briefly discusses the implementation
of the pointers to members:
\begin{quote}
... pointers to members are implemented as structures holding
relative positions in objects and indexes into virtual
function tables.  Thus, examining the value of a pointer to a member
will not necessarily reveal a machine address.
\end{quote}

\question{The implementation of pointers to members depends upon 
our class layout}.

\subsection{Range Types}

C++ does not directly support ranges.  We represent a range type
as a base type.  Some source languages require that we generate
code to ensure that values assigned to a range type are within the
bounds.

\subsection{Set Types}

\question{I don't know what the heck I'm going to do with set types}.


\subsection{Type Names}

The arguments of \key{sizeof} and \key{new} require the name of a
type.
\question{Also, type conversion in C++, but I need to look into that.}
In Clef, we specify the arguments for \key{sizeof} and \key{new} using
a type identifer (\ie an index into the type table) which correspondes
to a Type Clef node.  We need to be careful when we generate code for
the type because we do not want to always generate the complete type
declaration (since the type may have already been declared).
Instead, we generate the type name when possible.  For
example, if we declare a type using \key{typedef} then we would
like to generate the typedef name instead of the type declaration.

We add a special flag, called \textit{typeName}, to \key{Clef2c} which
indicates when we want to generate complete type declarations and when
we want to generate just type name, if possible.  We try to do this
for \key{structs}, \key{unions}, \key{enums}, and \key{classes}.  We
are able to generate just the type name when the types' parent node is
either a \textit{DeclType} or \textit{DeclName} node (because the type
nodes never contain a name).







