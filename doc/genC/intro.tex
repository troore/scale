
\chapter{Introduction}\label{chap:intro}
\pagenumbering{arabic}

We describe how to convert Clef to C.  Using Clef we can
represent languages such as C++\cite{ellis:90}, Modula-3, and Fortran.

We should also have an option to convert Clef to C++. An advantages of
generating C++ is that we do not have to convert C++ classes to C.

We have thought of a couple of different implementation strategies
for converting Clef to C.
\begin{enumerate}
\item Create an implementation of the Generation Interface that
generates C.
\item Generate C as a separate pass over Clef.
\end{enumerate}

Creating an implementation of the Generation Interface that generates
C appears promising but it has several problems.  First, we would need
to create a program which calls each of the interface routines to
actually generate the C code.  We could call the C generating routines
when we call the Clef generating roitines, but this means that we only
generate C code at the same times as generating Clef.  We would like
to generate C code after Clef has already been generated.

Instead we will generate C as a separate pass over Clef using the
traversal mechanisms built into Clef.

The C++ language generator, called {\tt Clef2Cxx}, is a subclass of
the C language generator.  Since it is a subclass, we only need
to redefine those methods which are specific to C++.


