
\section{Implementing Clef Expressions in C}

Most of the Clef operators can be expressed using C operators.  We
implement Clef opertors that do not have equivalent C operators as
function calls.  C provides many of the functions in the standard
library.  We must implement functions that are not provided by the C
library.

The Clef Design Documents enumerates the operator correspondences
for C++, Modula-3, and Fortran\cite{clefdd}.

The Clef operators that do not have corresponding C operators 
are:

\begin{description}
\item[Clef\_AbsoluteOp] Generate a call to \code{abs()}, \code{fabs()}.
These are defined in \key{stdlib.h} and
\key{math.h}.
\item[Clef\_MinimumOp] Generate a small code sequence to test
for min. The code is \code{(x < y ? x : y)}.
\item[Clef\_MaximumOp] Generate a small code sequence to test 
for max.  The code is \code{(x > y ? x : y)}.
\item[Clef\_AndOp] This operator does not have short-circuit semantics.
Hmm, not sure what to do about this one.
\item[Clef\_OrOp] This operator does not have short-circuit semantics.
Hmm, not sure what to do about this one.
\item[Set Operators] These are from Modula-3.  See below.
\item[Clef\_NilOp] We implement this as NULL.
\end{description}

\subsection{Implementing True/False}

We represent true as 1 and false as 0.

\subsection{Implementing Nil Operator}

Don't know what to do with this yet.

\subsection{Implementing SelectRelative}

Includes the indirect version.  I don't know what to do 
about these yet.  If we are generating C++, we don't have
a problem.  We just generate the corresponding C++ operators.

\subsection{Array Operators}

Clef contains several operators that compare strings (arrays of
characters) in order to represent Fortran \key{character} operations.
We generate calls to the runtime system routine \code{\_scmp()} to
compare strings (see Section~\ref{charops}).

\question{What do we do about bounds checking}

\subsection{Implementing Set Operators}
 
These are Modula-3 - I don't know what I'm going to
do with them.

\subsection{Call Operators}

Currently, we only generate code for regular function calls.  Generating
code for methods is much more involved.

I also don't do anything interesting with arguments such 
as positional arguments, named arguments, or different type
of parameters.

\subsection{Heap Operators}

Some of these work fine in C++ mode, but in C mode
we'll have trouble.

\subsection{Type Operators}

Most of them do not have C/C++ equivalent functions.

\subsection{Type Conversion}

Table~\ref{convert} lists the mapping between Clef's type conversion
routines and the C code that Clef2C generates.
\begin{table}
\begin{center}
\begin{tabular}{|l|c|c|} \hline
Conversion Routine & Expression & C Code \\ \hline \hline
cReal & float & \code{(float)e} or \code{(double)e} \\
cReal & complex & \code{(float)e.r} or \code{(double)e.r} \\
cFloor & float & \code{floor(e)} \\
cCeiling & float & \code{ceil(e)} \\
cRound & float & \code{(float)(e+0.5)} or \code{(float(e-0.5))} \\
cTruncate & complex & \code{(int)e.r} \\
cTruncate & float & \code{(int)e} \\
cOrdinal & enumeration & \code{e} \\
cEnumerationValue & int & \code{e} \\
cLoophole & any & \code{(T)e} \\
cCast & int & \code{(T)e} \\
cComplex & any & \code{createComplex(e)} \\ \hline
\end{tabular}
\end{center}
\caption{Type Conversion Code Sequences}
\label{convert}
\end{table}

\subsection{Aggregation Operators}

See Section 2.6.2 of Clef document.  We currently
do not do anything with these operators.

\subsection{Assigning Aggregates and Arrays}

The C languages allows aggregates and arrays to be assigned using the
assignment (=) operator.

In Modula-3, we need to insert a run-time check if either
the source or target is {\em open}.  We currently do not add
this check.

\subsection{Assigning Strings}

If the length function is {\tt cTerminated}, we generate a call to
{\tt strcpy()} for string assignment.  If the length function is {\tt
cFixedLength}, we generate the expression, {\tt strncpy(e1, e2,
strlen(e1))}.  We currently do not do anything with the {\tt padding}
expression if {\tt e1 > e2}.  Instead, we use the null character as
padding.

\subsection{Minimum and Maximum}

The C language does not contain primitives for minimum or maximum.
We defined macros for \key{min} and \key{max} in the Clef2C
header file (see Section~ref{header}).  The code sequences
for the macrors are:
\begin{codeseq}
\>( e1 < e2 ? e1 : e2)
\end{codeseq}
we use {\tt <} for minimum and {\tt >} for maximum.

\subsection{Exponentiation}

Fortran has a built-in exponential operator (**).  Clef2c generates
calls to run-time system routine to perform the exponentiation
operation.  The run-time library contains several different routines
to support exponentiation with different types.

\subsection{Operations on Complex Types}\label{complexops}

The easiest way to implement operations on complex types is to
generate calls to run-time routines.  The EDG Fortran compiler also
takes this approach.  Conversly, the f2c compiler generates in-line
code sequences.  Although in-line code sequences are more efficient,
they are difficult to generate.  To generate in-line code sequence we
need to create lots of temporaries which complicates Clef2C.

Table~\ref{complexarith} lists the run-time routines for complex operations.
\begin{table}
\begin{center}
\begin{tabular}{|l|c||c|c|} \hline
Operation & Fortran Op & Complex Type & Double Complex Type \\ \hline \hline
Negation & \code{-e} & \code{c \_negCmpx(c e)} 
		& \code{dc \_negDblCmpx(dc e)} \\
Addition & \code{e1+e2} & \code{c \_addCmpx(c e1, c e2)} 
		        & \code{dc \_addDblCmpx(dc e1, dc e2)} \\
Subtraction & \code{e1-e2} & \code{c \_subCmpx(c e1, c e2)} 
		           & \code{dc \_subDblCmpx(dc e1, dc e2)} \\
Multiplication & \code{e1*e2} & \code{c \_multCmpx(c e1, c e2)} 
			      & \code{dc \_multDblCmpx(dc e1, dc e2)} \\
Division & \code{e1/e2} & \code{c \_divCmpx(c e1, c e2)} 
		        & \code{dc \_divDblCmpx(dc e1, dc e2)} \\
Conversion & \code{cmplx(e)} & \code{c \_createCmpx(e, 0.0)} 
		      & \code{dc \_createDblCmpx(e, 0.0)} \\
Absolute Value & \code{abs(e)} & \code{float \_cabsCmpx(e)}
		     & \code{double \_cabsDblCmpx(e)} \\
Square Root & \code{sqrt(e)} & \code{c \_csqrtCmpx(e)}
		      & \code{dc \_csqrtDblCmpx(e)} \\
Exponetial & \code{exp(e)} & \code{c \_cexpCmpx(e)}
		    & \code{dc \_cexpDblCmpx(e)} \\
Logarithm & \code{log(e)} & \code{c \_clogCmpx(e)}
		    & \code{dc \_clogDblCmpx(e)} \\
Sine & \code{sin(e)} & \code{c \_csinCmpx(e)}
		    & \code{dc \_csinDblCmpx(e)} \\
Cosine & \code{cos(e)} & \code{c \_ccosCmpx(e)}
		    & \code{dc \_ccosDblCmpx(e)} \\
Imaginary Part & \code{aimag(e)} & \code{float \_caimagCmpx(e)}
	          	        & \code{double \_caimagDblCmpx(e)} \\
Conjugate & \code{conjg(e)} & \code{c \_cconjgCmpx(e)}
			    & \code{c \_cconjgDblCmpx(e)} \\ \hline
\end{tabular}
\end{center}
\caption{Runtime Complex Arithmetic Functions}
\label{complexarith}
\end{table}
Note that although f2c defines operations for most
of the intrinsic functions (\eg \key{abs}, \key{sqrt}, etc.)
we need to define our own versions.  The f2c versions take
pointers to the function arguments.  This is a problem if the
argument is an expression.  The f2c compiler deals with this
by generating a temporary and passing the address of the temporary
to its RTS routines.  Since generating temporaries is difficult, 
we call our own RTS routines which, in most cases, call the
f2c routines.

We implement the simple complex operations as in-line code sequences.
These operations do not involve temporary variables.  
Table~\ref{complexinline} lists the in-line complex operations.
\begin{table}
\begin{center}
\begin{tabular}{|l|c||c|} \hline
Operation & Fortran Op & Code Sequence \\ \hline \hline
Equal & \code{e1 .eq. e2} & \code{e1.r == e2.r \&\& e1.i == e2.i} \\
Not Equal & \code{e1 .ne. e2} & \code{e1.r != e2.r || e1.i == e2.i} \\
Assignment & \code{e1 = e2} & \code{e1 = e2} \\
\end{tabular}
\end{center}
\caption{Inline Complex Arithmetic Code}
\label{complexinline}
\end{table}

\subsection{Operations on Character Types}
\label{charops}

The Fortran character type is represented as a fixed array of type
\key{char}.  Fortran characters may be involved in assignment,
concatenation, and comparison operations.

We generate a call to a Clef2C library routine for assignments
(\code{\_sassign()}), concatenations (\code{\_sconcat()}), and
comparisons (\code{\_scmp()}).  We have not implemented the library
routines to correctly handle the case when the target appears on the
right hand side (\ie the NO\_OVERWRITE option for f2c).  The
implementation of \code{\_sconcat} allocates space for the result and
resturns a pointer to the allocated space.  The \code{\_scmp(str1,
str2)} routine compares two strings and returns -1 if str1 is less
than str2, 0 if str1 equals str2, or 1 if str1 is greater than str2.

\subsection{Fortran Intrinsic Operations}

Clef represents some of the Fortran intrinsic operations as direct
function calls (these operations do not map into Clef operators).
Some of these operations may be implemented as in-line sequences of
code, but we also have run-time system library routines.  We discuss
some of the intrinsic operations in other section (\eg operations
involving complex types), but we summarize all the intrinsic
operations here.  Table~\ref{intrinsic} lists each of the intrinsic
operations, the runtime system routine that implements the function,
if applicable, and the reprepresentation in Clef.

\begin{table}[h]
\begin{footnotesize}
\begin{center}
\begin{tabular}{|l|l|l|} \hline
Description & RTS Function & Clef Tree Operation \\ \hline \hline
Type Conversion & Inline Sequence & Type Conversion \\ \hline
Truncate towards Zero & \code{double \_aint(double)} & Type Conv. - cTruncate \\ \hline
Nearest Whole No. & \code{double \_anint(double)} & Type Conv. - cRound \\ \hline
Nearest Int. & \code{int \_nint(double)} & Type Conv. - cRound \\ \hline
 & C library \code{abs}, \code{fabs} & Absolute \\
Absolute Value & \code{\_complex \_cabsCmpx(\_complex)} & Absolute \\
 & \code{\_doublecomplex \_cabsDblCmpx(\_doublecomplex)} & Absolute \\ \hline
 & Inline Sequence \% or \code{\_mod(int)}  & Remainder/Func. Call \\ 
\rb{Remainder} & C Library \code{fmod} & Remainder \\ \hline
 & \code{int \_isign(int, int)} & Function Call \\
\rb{Transfer of Sign} & \code{double \_dsign(double, double)} & Function Call \\ \hline
 & \code{int \_idim(int, int)} & Function Call \\
\rb{Positive Difference} & \code{double \_idimdouble, double)} & Function Call \\ \hline
Double Precision Product & Inline Sequence or \code{\_dprod(float, float)} & (double)expr1*expr2 \\ \hline
Maximum & (macro) max(expr1, expr2) & Maximum \\ \hline
Minimum & (macro) min(expr1, expr2) & Minimum \\ \hline
Length & \code{int \_len(char *, int)} & Function Call \\ \hline
Index & \code{int \_index(char *, char *)} & Function Call \\ \hline 
 & \code{float \_caimagCmpx(\_complex)} & Function Call \\
\rb{Imaginary Part} & \code{double \_caimagDblCmpx(\_doublecomplex)} & Function Call \\ \hline
 & \code{\_complex \_cconjgCmpx(\_complex)} & Funcation Call \\
\rb{Conjugate} & \code{\_doublecomplex \_cconjgDblCmpx(\_doublecomplex)} & Funcation Call \\ \hline
 & C Library \code{sqrt} & Function Call \\ 
Square Root & \code{\_complex \_csqrtCmpx(\_complex)} & Function Call \\
 & \code{\_doublecomplex \_csqrtDblCmpx(\_doublecomplex)} & Function Call \\ \hline
 & C Library \code{exp} & Function Call \\ 
Exponential & \code{\_complex \_cexpCmpx(\_complex)} & Function Call \\
 & \code{\_doublecomplex \_cexpDblCmpx(\_doublecomplex)} & Function Call \\ \hline
 & C Library \code{log} & Function Call \\ 
Natural Log & \code{\_complex \_clogCmpx(\_complex)} & Function Call \\
 & \code{\_doublecomplex \_clogDblCmpx(\_doublecomplex)} & Function Call \\ \hline
Common Log & C library \code{log10} & Function Call \\ \hline
 & C Library \code{sin} & Function Call \\ 
Sine & \code{\_complex \_csinCmpx(\_complex)} & Function Call \\
 & \code{\_doublecomplex \_csinDblCmpx(\_doublecomplex)} & Function Call \\ \hline
 & C Library \code{cos} & Function Call \\ 
Cosine & \code{\_complex \_ccosCmpx(\_complex)} & Function Call \\
 & \code{\_doublecomplex \_ccosDblCmpx(\_doublecomplex)} & Function Call \\ \hline
Logarithm & & \\
Arcsine & & \\
Arctangent & C library routines & Function Call \\
Hyperbolic Sine & & \\
Hyperbolic Cosine & & \\
Hyperbolic Tangent & & \\ \hline
Lexical $\geq$& \code{int \_lge(char *, char *, int, int)} & Function Call \\ \hline
Lexical $>$  & \code{int \_lgt(char *, char *, int, int)} & Function Call \\ \hline
Lexical $\leq$ & \code{int \_lle(char *, char *, int, int)} & Function Call \\ \hline
Lexical $<$ & \code{int \_llt(char *, char *, int, int)} & Function Call \\ \hline
\end{tabular}
\caption{Fortran Intrinsic Operations in Clef2C\label{intrinsic}}
\end{center}
\end{footnotesize}
\end{table}

\subsection{Array Subscripts}
\label{arraysubs}

Fortran and C differ in the layout of arrays.  Clef2C must translate
Fortrans arrays into the correct C code and adjust for the differences
between Fortran and C.  We discuss two cases: single dimension arrays
and multi-dimension array.

C arrays starts at index 0 and Fortran arrays start at index 1.  When
the source language is Fortran, we need to subtract one from all array
subscript expressions.

\subsubsection*{Multi-Dimension Arrays}

Table 1 on page 5-6 in the Fortran Standard describes the translation
between mulidimensional arrays and subscript values~\cite{f77}.

	

