\section{Implementing Clef Statements in C}

In this section we describe how to implement Clef statement
in C.

\subsection{Mult\-way Branch Statements}

These statements implement Fortran constructs for computed \key{goto},
assigned \key{goto}, and the \key{assign} label statement.

\subsubsection{Assign Statement}

We need to maintain a mapping between label strings and integers.  The
Fortran \key{assign} statement assigns a label to an integer variable.
We convert the \key{assign} statement to a C integer asignment
statement and create an integer to represent the label (starting with
1).  We then emit an assignment of the integer value to the integer
variable.  We need to maintain the mapping of labels and integer
values so that if we need to refer to the label as an integer we can
look it up in.  We initialize the mapping to empty at procedure
boundaries.

\subsubsection{Assigned Goto}

We implement the \key{Assigned Goto} statement in C using a switch
statement.  In Clef, the assigned goto has an expression which is the
\key{switch} expression and a list of statement labels which
are the \key{case} alternatives.  Clef represents the statement labels 
as label declarations. We must translate the statement labels
to integer values using the label $\Leftrightarrow$ integer
mapping function.  Each \key{case} alternative is a \key{goto}
statement which jumps to the appropriate label.

\subsubsection{Computed Goto}

We treat a \key{Computed Goto} very similar to an \key{Assigned
Goto}. We implement the \key{Computed Goto} statement in C using a
switch statement.  The expression node in the Clef computed
goto node is the \key{switch} expression.  We associate
integer values for each statement label which correspondes
to the position of the statement in the list.  That is,
a list of statement labels \code{($s_1, s_2, ..., s_n$)}
are consecutively numbered 1, 2, ..., N.

\subsection{Case/Switch Statement}

We implement Modula-3 case statements using the C {\tt switch}
statement.  Currently, the code won't work if the case key for a case
alternative is a range.

\subsection{Typecase}

The {\tt typecase} statement is a Modula-3 construct.
We do not do anything for these yet.

\subsection{DoLoops}

This statement corresponds to a Fortran 77 iterating loop.

We implement a Fortran 77 iterating loop using a C \code{for} loop.
The \code{for} loop expression depends upon the direction and
expression in the Fortran loop.  If the loop increment expression is a
literal, then we can easily determine the direction of the loop and
generate more efficient code.  Otherwise, we do not know the direction
of the loop and we must generate an \code{for} loop expression that
checks the loop direction and performs the appropriate loop
termination check.

\subsection{Throw, Raise}

We do not generate any code for these statements.

\subsection{Alternate Return}

We generate a simple \code{return} statement for a Fortran alternative
return.  The return expression indicates the label in the caller's
routine where control is returned.  The real work for handling an
alternative return statement is performed by the caller since
the caller is responsible for transfering control to the 
proper statement.

The caller of a routine which contains an alternative return handles
the return using a \key{switch} statement.  The \key{switch} statement
expression is the function call.  The case alternative taken depends
upon the value returned by the function.  The case alterntives are
integer values which correspond to the labels in the function call.
We number the labels, 1..n, for n labels in the function call.
If the function returns 1, we generate a \key{goto} to the first
label. 

For example, the statement \code{call sub(a, *10, *20)} is
translated into the following C code:
\begin{codeseq}
\>switch (sub(\&a, 1, 2)) \{ \\
\>\>case 1: goto L10; \\
\>\>case 2: goto L20; \\
\>\}
\end{codeseq}

\subsection{Exception Handling Statements}

We do not generate any code for any exception handling
statements.

\subsection{Miscellaneous Statements}

We do not generate code for the following statement nodes:
SpcfyUsing, UsingDirective, WithAlias.

\subsection{Function Calls}

If the Clef tree represents a C++ program then we add an addition
parameter to each member function (\ie a method).  This parameter is
the \key{this} pointer.  For example (from \cite{ellis:90}),
given a call to the member function \code{A::f},
\begin{codeseq}
A * pa; \\
ps$\rightarrow$f(2); \\
\end{codeseq}
could be implemented as:
\begin{codeseq}
f(pa, 2);
\end{codeseq}

Note that the type of \code{this} in a member function of class
\code{X} is \code{X * const}\footnote{Unless the member function
is declared \key{const} or \key{volatile}.  See Section~\ref{sec:decls}.}.

\subsection{Destructors}


\subsection{Stop and Pause}

Fortran 77 contains statements \code{stop} and \code{pause}.  
We implement these statements as runtime system routines
(\code{\_stop()} and \code{\_pause()}).  Clef
represents these statements as calls to the runtime system
routines.

