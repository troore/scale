
\section{Miscellaneous Stuff}

\subsection{Generating New Symbols}

We need to generate new symbols in order to create names for
variables, data structures, etc. that are not defined in the originial
source.  Users do not typically use identifiers that begin with two
underscores\footnote{I believe these are reserved}.

Since we are converting languages other than C or C++ we must be
careful about identifier names.  Specifically, we must not all
identifiers that are also keywords.  If we encounter an identifier
that is a keyword then we prepend the identifier with two underscores.

\subsubsection{Function Name Encoding - aka. Name Mangling}

If the source language is C++, we implement a function name encoding
scheme to ensure type-safe linkage.  This is also know as name
mangling.  We use the same encoding scheme document in the Annotated
C++ Reference Manual~\cite{ellis:90}.

The encoding scheme works by appending the function's signature
to its name.  A double underscore (\_\_) separates the name
of the function with the encoded signature name.

The name of a function includes :
\begin{itemize}
\item An encoding of the class in which it is defined,
\item A special prefix for global function names,
\item An encoding of the types of the function's arguments,
\item A special prefix for member function (for the \key{this} pointer),
\item An encoding for type modifiers (\ie unsigned, const, etc),
\item An encoding for type declarators,
\item A special case for pointers to functions
\item Special prefixes for operator functions and special members
(such as constructors and destructors).
\end{itemize}

The name of a class is encoded as the length of the name followed by
the name itself.  For example, \code{class foo} is encoded as
\texttt{3foo}.  A qualified class name has a special encoding,
the character \texttt{Q} followed by the number of encodines.
For example, \code{class Outer::Inner} is encoded as
\texttt{Q25Outer5Inner}, where 2 indicates there are two names.

A global function name is encoded by prefixing the signature with the
character \texttt{F}.

Table~\ref{typeencode} shows the encoding for the basic types,
type modifiers, and type declarators.
\begin{table}[h]
\begin{center}
\begin{footnotesize}
\begin{tabular}{|l|c|c||l|c|c|} \hline
Type & \multicolumn{2}{c||}{Encoding} & Type & \multicolumn{2}{c|}{Encoding} \\ \hline
\key{void} & \multicolumn{2}{c||}{\texttt{v}} & 
\key{char} & \multicolumn{2}{c|}{\texttt{c}} \\
\key{short} & \multicolumn{2}{c||}{\texttt{s}} & 
\key{int} & \multicolumn{2}{c|}{\texttt{i}} \\ 
\key{long} & \multicolumn{2}{c||}{\texttt{l}} & 
\key{float} & \multicolumn{2}{c|}{\texttt{f}} \\
\key{double} & \multicolumn{2}{c||}{\texttt{d}} & 
\key{long double} & \multicolumn{2}{c|}{\texttt{r}} \\
\key{...} & \multicolumn{2}{c||}{\texttt{e}} & &
\multicolumn{2}{c|}{ } \\ \hline \hline
Type Modifier & \multicolumn{2}{c||}{Encoding} & 
Type Modifier & \multicolumn{2}{c|}{Encoding} \\ \hline
\key{unsigned} & \multicolumn{2}{c||}{\texttt{U}} & 
\key{const} & \multicolumn{2}{c|}{\texttt{C}} \\
\key{volatile} & \multicolumn{2}{c||}{\texttt{V}} & 
\key{signed} & \multicolumn{2}{c|}{\texttt{S}} \\ \hline\hline
Type Declarator & Notation & Encoding & Type Declarator & Notation & Encoding \\ \hline
Pointer & * & \texttt{P} &
Reference & \& & \texttt{R} \\ 
Array & [10] & \texttt{A10\_} &
Function & () & \texttt{F} \\
Pointer To Member & S::* & \texttt{M1S} & & & \\ \hline
\end{tabular}
\end{footnotesize}
\caption{Type Encodings for Name Mangling}
\label{typeencode}
\end{center}
\end{table}

The encoding for a member function begins with the an encoding
for the class name.  For example, \code{bar::foo(int)} is 
encoded as \texttt{foo\_\_3barFi}.

For an argument that is a function pointer, we encode the arguments
types as well as the return type.  The encoded return type appears
after the argument types and is preceded by an underscore.  For
example, \code{f(int (*)(char *))} (\ie function f has a single
parameter which is a function pointer that returns an int
and has a single parameter which is a char pointer) is encoded
as \texttt{f\_FPFPc\_i}.

We also use special prefixes for all the operator functions.  (\eg the
built-in C++ operators that may be overloaded.)  We do not list the
encoding here since they are listed on pg. 125 of ~\cite{ellis:90}.
Pg. 125 also lists the encodings for the special functions such as
constructors, destructors, new, etc.
