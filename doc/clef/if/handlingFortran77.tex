% -*- Mode: latex; Mode: auto-fill; -*-

\documentclass[11pt]{article}

\usepackage[]{doublespace}
\setstretch{1.1}

%\usepackage{epsf}
\usepackage[dvips]{epsfig}

\usepackage{pstricks}
\usepackage{fancybox}

%\usepackage{draftstamp}

\oddsidemargin=0in
\evensidemargin=0in
\textwidth=6.5in
\textheight=8.0in
%\columnsep=0.25in
\footskip 0.2truein
%\footheight 0.5truein  TeTeX doesn't recognize it!
\marginparwidth=.75in
\marginparsep=.1in
\marginparpush=5pt
\renewcommand{\textfraction}{0.25}
\renewcommand{\topfraction}{0.75}
\renewcommand{\bottomfraction}{.85}
\renewcommand{\floatpagefraction}{.85}
\renewcommand{\dbltopfraction}{0.85}
\renewcommand{\dblfloatpagefraction}{.85}
\raggedbottom

\newcommand{\PARTITLE}[1]{\par\vspace{1ex}\noindent{\textit{ #1}}\\*\noindent}
\newcommand{\partitle}[1]{\PARTITLE{#1}} % Same as capitalized version
\newcommand{\eg}{{\itshape e.g.},\ }
\newcommand{\ie}{{\itshape i.e.},\ }
\newcommand{\etal}{{\itshape et al.}}
\newcommand{\clef}{Clef}
\newcommand{\tbs}{{\bfseries\itshape T.B.S}}

\newcommand{\key }[1]{\textsf{#1}}
\newcommand{\code}[1]{\texttt{#1}}
\newcommand{\node}[1]{\textmd{#1}}
\newcommand{\method}[1]{\textsl{#1}}

% For operator table:
\newcommand{\tabhead}[1]{\textit{#1}}
\newcommand{\na}{\textsl{N/A}}

\newcommand{\interface}[4]{
\par\vspace{1ex}\noindent
  \begin{samepage}
  %\hspace{-1ex}Interface Method:
  \textsl{#1} \textbf{#2}\index{#2} \textsl{(#3)} $\Rightarrow$ 
	\textmd{#4\index{#4}}\par
  \end{samepage}
}

\newcommand{\question}[1]{
\begin{center}
  \boxput*(-0.75,1)
    {\psframebox*{Question!}}
    {\psshadowbox[framesep=12pt]{\parbox{6.0in}{#1}}}
\end{center}
}

\newcommand{\befunc}[2]{
  At the point that the \method{#1\_end} routine is called, the top of
  the stack (\ie the portion added since the \method{#1\_begin} call)
  should contain only #2 nodes.  
}
\newcommand{\oparg}[2]{
  Operands for #1 operators must be of type #2.  
}
\newcommand{\modulaOp}{
  This routine implements the corresponding Modula3 operator.
}

\newcommand{\todo}[1]{
\par\noindent
%\begin{verbatim}
%#1
%\end{verbatim}
}

\newcommand{\EnumOptions}[2]{
  \begin{center}
  \code{enum #1 \{#2\};}
  \end{center}
}

\newcommand{\Section}[1]{\newpage\section{#1}}
\newcommand{\Subsection}[1]{
  \begin{samepage}
    \setlength{\epsfxsize}{\textwidth}
    \vspace{3.0ex}\noindent
    \epsfbox{clefbar.eps}
    \nopagebreak\vspace{-5.0ex}\nopagebreak
    \subsection{#1}
  \end{samepage}}
\setcounter{secnumdepth}{5}  % Number subsections.
\setcounter{tocdepth}{3}     % Include subsubsections in TOC.

\newcommand{\Paragraph}[1]{\noindent\paragraph{#1}\par}

\newenvironment{functionality}
               {\list{}{\rightmargin 0em
			\listparindent \parindent}%
                \item[]\hspace{\parindent}}
               {\endlist}

\newlength{\argwidth}
\setlength{\argwidth}{4.5in}

\newenvironment{parameters}
               {\begin{itemize}\item[]\textit{Parameters}:\begin{description}}
               {\end{description}\end{itemize}}

\newenvironment{Description}
               {\begin{itemize}\item[]\begin{description}}
               {\end{description}\end{itemize}}

%$$$$$$$$$$$$$$$$$$$$$$$$$$$$$$$$$$$$$$$$$$$$$$$$$$$$$$$$$$$$$$$$$$$$$$$$$$$$$$
%$$$$$$$$$$$$$$$$$$$$$$$$$$$$$$$$$$$$$$$$$$$$$$$$$$$$$$$$$$$$$$$$$$$$$$$$$$$$$$

\begin{document}


\paragraph{Handling unusual structures}
\begin{description}
\item [Intrinsic Functions] These functions will be places in an
implicit library.  
\item [Implicit] Existing routines are sufficient.
\item [Parameter] Existing routines are sufficient.
\item [Data] Existing routines are sufficient.
\item [Save] Existing routines are sufficient.
\end{description}

\paragraph{New attributes}
\begin{description}
\item [intrinsic] Declaration attribute
\item [external] Declaration attribute (but called something else)
\item [array layout] Need unit attribute to indicate array layout.
\end{description}

\paragraph{Aliases}
I forgot to handle C++ reference types, which are similar to
Fortran~77's equivalence construct.  Since C++ treats the reference
symbol (\&) as a type construtor, we need a reference type attribute.
However, when a reference value is actually declared, we want to
separate out the definition of the variable from the specification of
its alias in order to support Fortran~77.

\interface{void}{alias}{Declaration d1, Declaration d2}{void}
\begin{functionality}
This routine defines \code{d1} as an alias for \code{d2}.  For C++,
\code{d1} is the reference.  For C++, the declaration of a reference
name should be followed immediately by a definition of what that name
aliases.  
\end{functionality}

\paragraph{Strings}
Some languages, such as Fortran, have strings as a primitive type.
The following operators assume Fortran string handling semantics.

%--
\interface{void}{string\_assignment}{Expression e1, Expression e2}{Expression}
%--
\interface{void}{string\_equality}{Expression e1, Expression e2}{Expression}
%--
\interface{void}{string\_inequality}{Expression e1, Expression e2}{Expression}
%--
\interface{void}{string\_greater}{Expression e1, Expression e2}{Expression}
%--
\interface{void}{string\_greater\_equal}{Expression e1, Expression e2}{Expression}
%--
\interface{void}{string\_less}{Expression e1, Expression e2}{Expression}
%--
\interface{void}{string\_less\_equal}{Expression e1, Expression e2}{Expression}
%--
\interface{void}{substring}{Expression s, Expression b, Expression e}{Expression}
\begin{functionality}
\begin{parameters}
\item [s] The string to be operated on.
\item [l] The beginning position of the substring.
\item [e] The ending position of the substring.
\end{parameters}
\end{functionality}
%--
\interface{void}{substring\_remainder}{Expression s, Expression l}{Expression}
\begin{functionality}
This operator returns the rest of a string.
\begin{parameters}
\item [s] The string to be operated on.
\item [l] The beginning position of the substring.
\end{parameters}
\end{functionality}
%--
\interface{void}{concatenation}{Expression s1, Expression s2}{Expression}
\begin{functionality}
Concatentation is performed into a temporary variable.
\end{functionality}

\paragraph{Statements}
%--
\interface{void}{three\_way\_if}{Expression e, LabelDecl l, LabelDecl e,
	LabelDecl g}{Statement}
\begin{functionality}
This routine implements the semantics of Fortran~77's arithmetic if
statement.  
\end{functionality}
%--
\interface{void}{labels\_begin}{}{void}
%--
\interface{void}{labels\_end}{}{Labels}
\begin{functionality}
\befunc{labels}{LabelDecl}
\end{functionality}
%--
\interface{void}{computed\_goto}{Labels l, Expression e}{Statement}
\begin{functionality}
This routine implements the semantics of Fortran~77's computed goto
statement.  It is similar to a switch statement, except that the
target labels are not limited to a single statement of code.
\end{functionality}
%--
\interface{void}{assign\_label}{ValueDecl v, LabelDecl l}{Statement}
\begin{functionality}
This routine implements the semantics of Fortran~77's assign
statement.  I'm considering representing this as a type conversion
instead.  
\end{functionality}
%--
\interface{void}{assigned\_goto}{Expression e, Labels l}{Statement}
\begin{functionality}
This routine implements the semantics of Fortran~77's assigned goto
statement.  
\end{functionality}

\interface{void}{alternative\_return}{Expression e}{Statement}
\begin{functionality}
This routine implements the semantics of Fortran~77's alternative
return construct. 
\end{functionality}

\paragraph{Alternative Entry Points}
Fortran allows routines to have multiple entry points and alternative
return points.  
\partitle{Alternative Entries}

\interface{NameID}{entry}{Identifier name, Signature s}{RoutineDecl}
\begin{functionality}
This routine implements the semantics of Fortran~77's entry construct.
We define an EntryDecl which is a subclass of a FunctionDecl (only
functions may have alternative entry points).  
\end{functionality}

\paragraph{Decisions}
\begin{itemize}
\item Should we explicitly represent Fortran's output statements, or
should we put these routines into our implicit library?
\item Fortran's assigned goto could be handled by having variables of
label type.  Assign statements would be represented as a type
conversion from an integer to a label.
\item Can users write \emph{generic} functions?
\end{itemize}


\paragraph{Miscellaneous}
Fortran specifies that overflow and underflow errors are caught, so we
need alternate versions of arithmetic operators to handle these cases.

\end{document}
