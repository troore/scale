% -*- Mode: latex; Mode: auto-fill; -*-
\Section{Generating Declarations}\label{sec:decl}

This section covers how declarations are handled by the generation
interface.  Declarations are one way of binding attributes (or values)
to an identifier (or name).  The generation interface divides a
declaration into three parts: a name, a value, and miscellaneous
attributes.  Anonymous declarations exist, but they behave as though
the system creates a unique identifier for the declaration.  The
generation interface supports the declaration of types, values (\eg
variables and record fields), labels, procedures, and exceptions (for
Modula-3).  The interface also supports a variety of additional
attributes.

%(*-------------------------------------------------------- runtime hooks ---*)

\todo{
set_runtime_hook (n: Name;  v: Var;  o: ByteOffset);
(* declares 'n' as a runtime symbol at location 'ADR(v)+o' *)

get_runtime_hook (n: Name;  VAR v: Var; VAR o: ByteOffset);
(* returns the location of the runtime symbol 'n' *)
}

%==============================================================================
\Subsection{Declaring Types}

%--
\interface{NameID}{DeclType}{Identifier name}{Type t}{TypeDecl}
\begin{functionality}
  This routine creates a new type by making an entry for it in the
symbol table.  

\emph{Implementation Note:} To support our type matching scheme, this
routine should perform an extra step.  It should first brand \args{t}
with a \node{NameBrandType}, and then use the updated type for the
declaration.  
\end{functionality}

\interface{NameID}{NameType}{Identifier name, TypeID type}{}{TypeDecl}
\begin{functionality}
  This routine creates an alias for a type which may already have an
entry in the symbol table.  This routine implements the semantics of
C++'s \key{typedef} construct.  Notice the use of \args{TypeID} rather
than \node{Type}.

\emph{Implementation Note:} If the type already has a name brand, then
this name should simply be added to the existing name brand.
Otherwise, a name brand should be created for it.
\end{functionality}

%%--
%\interface{NameID}{DeclAnonType}{}{Type td}{TypeDecl}
%--

\subsubsection{Forward Declarations}
%--
\interface{NameID}{ForwardDeclProcedure}{Identifier
name}{Signature}{RoutineDecl}
\begin{functionality}
In Java terms, this routine has been \textbf{deprecated}.  Use
\texttt{SpcfyProcedure} instead.
\end{functionality}

%------------------------------------------------------------------------------
\subsubsection{Opaque Types}

Opaque types are type names for which the full type structure is
unknown.  Instead, we know that the type is a subtype of the specified
type.  

Modula-3 distinguishes between an opaque declaration and a partial
revelation.  This distinction is not important to the generation
interface, so a single routine exists.  From the generation
interface's point of view, an opaque declaration creates a name for a
partially specified type.

Fully revealed opaque types are really just a type declaration, and
should be represented as such.

Note that implementations will most likely wish to create a special
type constructor that marks an opaque type as representing the super
type.

\interface{NameID}{DeclOpaque}{Identifier name, TypeID superType}{}
	{TypeDecl}
\begin{functionality}
This routine handles the declaration of opaque types and partial
revelations.  An opaque type is a Modula-3 construct whereby a type is
only specified as being a subtype of another type.  Additional
information may be provided by partial revelations, which merely
provide a more specific super type.  

We considered associating partial revelations with the original opaque
declaration.  However, the interface does not need to do error
checking, so this associationg is not necessary.  

A complete revelation provides the full type definition, and is
therefore represented as a normal type declaration.

\emph{Implementation suggestion:} An opaque declaration associates a
name with a type.  However, the name is not an instance of the type,
but rather an instance of a subtype of that type.  Hence, we recommend
creating a \emph{subtype} type constructor.  The symbol table entry
for an opaque type may then point to a subtype node in the type table.
Note that the generation interface does not provide any routines to
directly generate a subtype type node.
\end{functionality}

%==============================================================================
\Subsection{Declaring Values}

Many source languages optionally allow programmers to give a default
value (default value for formal parameters and initial value for
variables) to a named values in its declaration.  Since the generation
interface requires an expression corresponding to the default value to
be on the stack, user code should use the \method{noExpression}
routine to push a special expression onto the stack.

\emph{Implementation note}: Implementations may choose to represent
initial values for variables and formals as a simple assignment
following the declaration.  Since the generation interface is intended 
to support a variety of intermediate representations, it provides as
much support for the original syntax as possible and allows
implementations to choose how best to represent initial values.
Moreover, an initialization of a C++ reference type is different from
an assignment to it.

%--
\interface{NameID}{DeclVariable}{Identifier name}
	{Type t, Expression initialValue}{ValueDecl}
\begin{functionality}
This routine declares a variable.
\begin{Parameters}
\Param{name} The variable's name.
\Param{t} The variable's type.
\Param{initialValue} The variable's initial value.
\end{Parameters}
\end{functionality}
%--
%--
\interface{NameID}{DeclTemporary}{Identifier name}
	{Type t, Expression initialValue}{ValueDecl}
\begin{functionality}
This routine declares a temporary variable.  We provide this
routine to distinguish programmer declared variables and compiler
created variables.
\begin{Parameters}
\Param{name} The variable's name.
\Param{t} The variable's type.
\Param{initialValue} The variable's initial value.
\end{Parameters}
\end{functionality}
%--
\interface{NameID}{DeclFormal}{Identifier name, Mode m}
	{Type t, Expression defaultValue}{ValueDecl}
\begin{functionality}
This document uses the term \emph{argument} to refer to an actual
parameter and \emph{formal} to refer to a formal parameter.  

\begin{Parameters}
\Param{name} The formal's name.
\Param{m} The formal's parameter mode.
\Param{t} The formal's type.
\Param{defaultValue} The formal's default value.
\end{Parameters}

Implementations of the generation interface must provide an enumerated
list of parameter mode options, such as the following:
\EnumOptions{Mode}{cInValue, cValue, cReference, cValueResult, cResult}
where:
\begin{Description}
\item [cInValue] Pass-by-value but the formal's value may not be
altered.  This mode is used for Ada.
\item [cValue] Pass-by-value but the formal's value may be altered.
This mode is the default for Modula-3 and the only one for C/C++.
\item [cReference] Formal parameter is an alias for the argument.
Therefore, an update to the formal parameter is a direct update the
argument (actual parameter) as well.  This mode is the only parameter
mode for Fortran and represents the \key{var} mode for Modula-3.
\item [cReadOnly] Pass-by-reference, but the value cannot be updated.
This mode supports Modula-3's \key{readonly} mode, and is efficient
for large data structures.
\item [cValueResult] Represents copy-in and copy-out semantics.  This
mode is used for Ada's \key{inout} mode. 
\item [cInOut] Represents Ada's \key{inout} keyword, which allows
implementations to choose between \textbf{Reference} and
\textbf{ValueResult}.  
\item [cResult] Represents Ada's \key{out} mode, which allows
implementations to choose between \textbf{Reference} and
\textbf{ValueResult}. 
\end{Description}

Handling C++'s parameter modes is awkward.  In C++, programmers may
declare a formal to be constant (\ie pass-by-inValue) or to be a
reference (\ie pass-by-reference).  Unfortunately, this declaration
may be buried in the type declaration used in the definition of the
formal.  These unfortunate declaration semantics requires searching
through the symbol table to determine the parameter passing mode, and
would require that clients recognize the duplication and ignore it.
Hence, the generation interface dictates that for C++, all calls to
\method{declFormal} should use pass-by-value.
\end{functionality}

%--
%\interface{NameID}{DeclAnonymousFormal}{Mode m}
%	{Type t, Expression defaultValue}{ValueDecl}
%\begin{functionality}
%This routine creates an anonymous formal.  Anonymous formals may be
%used to declare forward declarations and procedure types.  See
%\method{declFormal} for a more detailed description
%\end{functionality}
%--
\interface{void}{UnknownFormals}{}{}{ValueDecl}
\begin{functionality}
This routine indicates that the remainder of a routine's signature has
an unknown number of formal parameters.  Hence, this routine
implements C++'s ``...'' construct.
\end{functionality}

%--
\interface{void}{AlternateReturnFormal}{}{}{ValueDecl}
\begin{functionality}
Fortran provides an unusual version of a return statement called an
alternate return.  An alternate return returns to one of possibly
many line numbers, where the possible return locations are provided in
special procedure parameters.  The Fortran syntax for these special
parameters is an asterik ($*$), which this routine represents.

Alternate return parameters may occur anywhere in the parameter list;
however, the alternate return construct views these special parameters
as forming a simple linear list from 1 to n.
\end{functionality}

%--
\interface{NameID}{DeclField}{Identifier name}
	{Type t, Expression initialValue}{FieldDecl}
\begin{functionality}
This routine declares a field, which is a component of an aggregate
data structure.

\begin{Parameters}
\Param{name} The field's name.
\Param{t} The field's type.
\Param{initialValue} The field's initial value.
\end{Parameters}
\end{functionality}
%--
%\interface{NameID}{DeclAnonymousField}{}
%	{Type t, Expression initialValue}{FieldDecl}
%\begin{functionality}
%This routine declares an anonymous field, as in C bit fields.  See the
%\method{declField} routine for more information.

%\begin{Parameters}
%\Param{t} The field's type.
%\Param{initialValue} The field's initial value.
%\end{Parameters}
%\end{functionality}

%--
\interface{NameID}{DeclConstant}{Identifier name}
	{Type td, Expression value}{ConstantDecl}
\begin{functionality}
This routine declares a constant value.  A constant value does not
require storage to be allocated, which is different from other
declarations with a constant attribute.

\begin{Parameters}
\Param{name} The constant's name.
\Param{t} The contant's type.
\Param{value} The contant's value.
\end{Parameters}
\end{functionality}

%(*------------------------------------------------ variable declarations ---*)

%(* Clients must declare a variable before generating any statements or
%   expressions that refer to it;  declarations of global variables and
%   temps can be intermixed with generation of statements and expressions.

%   In the declarations that follow:

%|    n: Name            is the name of the variable.  If it's M3ID.NoID, the
%|                         the back-end is free to choose its own unique name.
%|    s: ByteSize        is the size in bytes of the declared variable
%|    a: Alignment       is the minimum required alignment of the variable
%|    t: Type            is the machine reprentation type of the variable
%|    m3t: TypeUID       is the UID of the Modula-3 type of the variable,
%|                       zero is used to represent "void" or "no type"
%|    in_memory: BOOLEAN specifies whether the variable must have an address
%|    exported: BOOLEAN  specifies whether the variable must be visible in
%|                         other compilation units
%|    init: BOOLEAN      indicates whether an explicit static initialization
%|                         immediately follows this declaration.
%|    up_level: BOOLEAN  specifies whether the variable is accessed from
%|                         nested procedures.
%|    f: Frequency       is the front-end estimate of how frequently the
%|                         variable is accessed.

%*)


%import_global (n: Name;  s: ByteSize;  a: Alignment;  t: Type;
%               m3t: TypeUID): Var;
%(* imports the specified global variable. *)

%declare_segment (n: Name;  m3t: TypeUID): Var;
%bind_segment (seg: Var;  s: ByteSize;  a: Alignment;  t: Type;
%              exported, init: BOOLEAN);
%(* Together declare_segment and bind_segment accomplish what
%   declare_global does, but declare_segment gives the front-end a
%   handle on the variable before its size, type, or initial values
%   are known.  Every declared segment must be bound exactly once. *)

%segment_types (seg: Var);

%declare_global (n: Name;  s: ByteSize;  a: Alignment;  t: Type;
%                m3t: TypeUID;  exported, init: BOOLEAN): Var;
%(* declares a global variable. *)

%declare_constant (n: Name;  s: ByteSize;  a: Alignment;  t: Type;
%              m3t: TypeUID;  exported, init: BOOLEAN): Var;
%(* declares a read-only global variable *)
 
%declare_local (n: Name;  s: ByteSize;  a: Alignment;  t: Type;
%               m3t: TypeUID;  in_memory, up_level: BOOLEAN;
%               f: Frequency): Var;
%(* declares a local variable.  Local variables must be declared in the
%   procedure that contains them.  The lifetime of a local variable extends
%   from the beginning to end of the closest enclosing begin_block/end_block. *)

%declare_param (n: Name;  s: ByteSize;  a: Alignment;  t: Type;
%               m3t: TypeUID;  in_memory, up_level: BOOLEAN;
%               f: Frequency): Var;
%(* declares a formal parameter.  Formals are declared in their lexical
%   order immediately following the 'declare_procedure' or
%   'import_procedure' that contains them.  *)

%declare_field (n: Name;  o: BitOffset;  s: BitSize;  t: TypeUID);

%declare_temp (s: ByteSize;  a: Alignment;  t: Type;
%              in_memory: BOOLEAN): Var;
%(* declares an anonymous local variable.  Temps are declared
%   and freed between their containing procedure's begin_procedure and
%   end_procedure calls.  Temps are never referenced by nested procedures. *)

%free_temp (v: Var);
%(* releases the space occupied by temp 'v' so that it may be reused by
%   other new temporaries. *)

\todo{
(*---------------------------------------- static variable initialization ---*)

(* Global variables may be initialized only once.  All of their init_*
   calls must be bracketed by begin_init and end_init.  Within a begin/end
   pair, init_* calls must be made in ascending offset order.  Begin/end
   pairs may not be nested.  Any space in a global variable that's not
   explicitly initialized is zeroed.  *)

begin_init (v: Var);
end_init (v: Var);
(* must precede and follow any init calls *)

init_int (o: ByteOffset;  READONLY value: Target.Int;  t: Type);
(* initializes the integer static variable at 'ADR(v)+o' with
   the low order bits of 'value' which is of integer type 't'. *)

init_proc (o: ByteOffset;  value: Proc);
(* initializes the static variable at 'ADR(v)+o' with the address
   of procedure 'value'. *)

init_label (o: ByteOffset;  value: Label);
(* initializes the static variable at 'ADR(v)+o' with the address
   of the label 'value'.  *)

init_var (o: ByteOffset;  value: Var;  bias: ByteOffset);
(* initializes the static variable at 'ADR(v)+o' with the address
   of 'value+bias'.  *)

init_offset (o: ByteOffset;  var: Var);
(* initializes the static variable at 'ADR(v)+o' with the integer
   frame offset of the local variable 'var' relative to the frame
   pointers returned at runtime in RTStack.Frames *)

init_chars (o: ByteOffset;  value: TEXT);
(* initializes the static variable at 'ADR(v)+o' with the characters
   of 'value' *)

init_float (o: ByteOffset;  READONLY f: Target.Float);
(* initializes the static variable at 'ADR(v)+o' with the
   floating point value 'f' *)
}

%==============================================================================
\Subsection{Declaring Procedures and Methods}

Unlike most other declared entities, procedures are generally not fully
declared at the point of their declaration.  The generation interface
refers to a function declaration without a body (\ie a C/C++ prototype
or a Modula-3 procedure interface) as a \emph{specification}, and a
function declaration with a body as a \emph{declaration}.  

Hence, the compiler or linker must ultimately resolve all function
specifications back to their corresponding declarations.  For Modula-3
this process is straightforward because of the well-structured
import/export mechanism.  For C/C++, this process is a little more
work.  

C++ (\cite{ellis:90}, p. 138) and Modula-3 (\cite{Cardelli:95}, p. 27)
define slightly different type matching rules for procedures.  Since
the generation interface currently supports only statically typed
languages and then only when semantic resolution has already been
done, we ignore these differences.

%\partitle{Overrides}
%%--
%\interface{void}{Override}{}{Identifier name, ConstantExpression proc}{Member}
%	%declare_override (obj: TypeUID; n: Name; offset: BitOffset);
%%--
%\interface{void}{OverrideBegin}{}{}{}
%%--
%\interface{void}{OverrideEnd}{}{\manyPops}{Overrides}
%\begin{functionality}
%\befunc{override}{\node{Overrides}}
%\end{functionality}

\partitle{Initializers}
C++ allows a list of initializers to be provided along with
constructors.  These routines handle the representations of
initializers.

%--
\interface{void}{Initializer}{NameID initializedEntity}
	{ArgumentList al}{Initializer}
%--
\interface{void}{InitializersBegin}{}{}{}
\interface{void}{InitializersEnd}{}{\manyPops}{Initializers}
\begin{functionality}
\befunc{Initializers}{ArgumentList}
\end{functionality}
%--
\interface{void}{NoInitializers}{}{}{Initializers}
\begin{functionality}
This routine creates an empty initializer list.
\end{functionality}

\partitle{Routine Specification}
A specification indicates the name and type of a procedure but does
not provide its body.  

%--
\interface{NameID}{SpcfyProcedure}{Identifier name}{Signature s}{RoutineDecl}
%--
\interface{NameID}{SpcfyNestedProcedure}
	{Identifier name, int level, NameID parentRoutine}
	{Signature s}{RoutineDecl}

%--
\interface{NameID}{SpcfyMethod}{Identifier name}{Signature s}{RoutineDecl}

%--
\interface{NameID}{SpcfyFriend}{Identifier name}{Signature s}{RoutineDecl}

%--
\interface{NameID}{SpcfyTypeConversion}{Identifier name}{Signature s}
	{RoutineDecl}

%--
\interface{NameID}{SpcfyConstructor}{Identifier name}
	{Signature s, Initializers i}{RoutineDecl}

%--
\interface{NameID}{SpcfyDestructor}{Identifier name}{Signature s}
	{RoutineDecl}


\partitle{Routine Declaration}
A routine declaration includes the routine's body.  

%--
%\interface{NameID}{DeclProcedure}{Identifier name}{Signature s, 
%	Statement body}{RoutineDecl}
%--
\interface{NameID}{DeclProcedure}
	{Identifier name, int level, NameID parentRoutine} 
	{Signature s, Statement body}{RoutineDecl}

%--
\interface{NameID}{DeclMethod}{Identifier name, TypeID class}
	{Signature s, Statement body}{RoutineDecl}
\begin{functionality}
The class type is specified if the method is defined outside the
class.  Otherwise, use \node{NoType}.
\end{functionality}

%--
\interface{NameID}{DeclMethodReference}{Identifier name, NameID proc}
	{Signature s}{RoutineDecl}
\begin{functionality}
This routine supports Modula-3 style method definition in which a
method is defined by a top level procedure.
\end{functionality}

%--
\interface{NameID}{DeclOverride}{Identifier name, NameID proc}
	{Signature s}{RoutineDecl}
\begin{functionality}
A Modula-3 override looks a lot like a method definition; however, its
affects the method table differently.
\end{functionality}

%--
\interface{NameID}{DeclTypeConversion}{Identifier name, TypeID class}
	{Signature s, Statement body}{RoutineDecl}
\begin{functionality}
The class type is specified if the method is defined outside the
class.  Otherwise, use \node{NoType}.
\end{functionality}

%--
\interface{NameID}{DeclConstructor}{Identifier name, TypeID class}
	{Signature s, Initializers, Statement body}{RoutineDecl}
\begin{functionality}
The class type is specified if the method is defined outside the
class.  Otherwise, use \node{NoType}.
\end{functionality}

%--
\interface{NameID}{DeclDestructor}{Identifier name, TypeID class}
	{Signature s, Statement body}{RoutineDecl}
\begin{functionality}
The class type is specified if the method is defined outside the
class.  Otherwise, use \node{NoType}.
\end{functionality}

%--
\interface{NameID}{DeclEntry}{Identifier name, NameID enclosingProcedure}
	{Signature s, Statement s1}{Statement}
\begin{functionality}
Fortran~77 allows procedures to have multiple entry points.  The
handling of variables for multiple entry points is messy at best.  The
generation interface uses Fortran~77 semantics.  Since each entry can
have its own argument list, the location (on the stack) of a parameter
depends upon which entry point into the procedure was called.
Moreover, parameters for one entry point may not be defined by another
in which case the latter entry point cannot execute code which refers
to such parameters.

The operation of this routine is different from most routines in the
generation interface.  In order to mark where the entry point begins,
the generation interface requires that the first statement in the body
of the entry be passed in \code{s1}.  Instead of returning a 
\node{RoutineDecl} like the other routines in this section, it returns
a \node{statement}, which is in fact \code{s1}.  The \node{RoutineDecl} 
created by this routine is accessible through the \code{NameID} 
returned by this routine.
\begin{Parameters}
\Param{name} Name of the entry point.
\Param{s} Signature of the entry point.
\Param{enclosingProcedure} The procedure which encloses this entry
point. 
\Param{s1} The first statement in entry.  Note that \code{s1} is 
only the first statement executed, not the entire body of the routine
as used in other routine declarations.
\end{Parameters}
\end{functionality}



%%--
%\interface{void}{BeginBody}{}{}{}
%	%begin_procedure (p: Proc);
%	%(* begin generating code for the procedure 'p'.  Sets "current 
%	% procedure" to 'p'.  Implies a begin_block.  *)

%%--
%\interface{void}{EndBody}{}{}{}
%	%end_procedure (p: Proc);
%	%(* marks the end of the code for procedure 'p'.  Sets "current 
%	% procedure" to NIL.  Implies an end_block.  *)




	%declare_procedure (n: Name;  n_params: INTEGER;  return: Type;
	%                   lev: INTEGER;  cc: CallingConvention;
	%                   exported: BOOLEAN;  parent: Proc;
	%                   ret_type_uid: TypeUID): Proc;
	%(* declare a procedure named 'n' with 'n_params' formal parameters
	% at static level 'lev'.  Sets "current procedure" to this procedure.
	% If the name n is M3ID.NoID, a new unique name will be supplied by the
	%back-end.  The type of the procedure's result is specifed in 'return'.
	% If the new procedure is a nested procedure (level > 1) then 'parent'
	% is the immediately enclosing procedure, otherwise 'parent' is NIL.
	% The formal parameters are specified by the subsequent 'declare_param'
	% calls. *)
%(*----------------------------------------------------------- procedures ---*)

%(* Clients compile a procedure by doing:

%      proc := cg.declare_procedure (...)
%        ...declare formals...
%        ...declare locals...
%      cg.begin_procedure (proc)
%        ...generate statements of procedure...
%      cg.end_procedure (...)

%  Nested procedure bodies may be generated before their parent's
%  begin_procedure, after their parent's end_procedure, or inline
%  where they occur in the source.  If they are not inline,
%  note_procedure_origin is used to mark their original source
%  position.  Runtime flags passed to the front-end determine where
%  nested procedure bodies are generated.  Back-ends are free to
%  require any one of these three placements for nested procedures.
%  (At the moment:  m3cc -> inline,  m3back -> after)
%*)

%import_procedure (n: Name;  n_params: INTEGER;  return: Type;
%                  cc: CallingConvention): Proc;
%(* declare and import the external procedure with name 'n' and 'n_params'
%   formal parameters.  It must be a top-level (=0) procedure that returns
%   values of type 'return'.  'lang' is the language specified in the
%   procedure's <*EXTERNAL*> declaration.  The formal parameters are specified
%   by the subsequent 'declare_param' calls. *)

%declare_procedure (n: Name;  n_params: INTEGER;  return: Type;
%                   lev: INTEGER;  cc: CallingConvention;
%                   exported: BOOLEAN;  parent: Proc;
%                   ret_type_uid: TypeUID): Proc;
%(* declare a procedure named 'n' with 'n_params' formal parameters
%   at static level 'lev'.  Sets "current procedure" to this procedure.
%   If the name n is M3ID.NoID, a new unique name will be supplied by the
%   back-end.  The type of the procedure's result is specifed in 'return'.
%   If the new procedure is a nested procedure (level > 1) then 'parent' is
%   the immediately enclosing procedure, otherwise 'parent' is NIL.
%   The formal parameters are specified by the subsequent 'declare_param'
%   calls. *)

%begin_procedure (p: Proc);
%(* begin generating code for the procedure 'p'.  Sets "current procedure"
%   to 'p'.  Implies a begin_block.  *)

%end_procedure (p: Proc);
%(* marks the end of the code for procedure 'p'.  Sets "current procedure"
%   to NIL.  Implies an end_block.  *)

%note_procedure_origin (p: Proc);
%(* note that nested procedure 'p's body occured at the current location
%   in the source.  In particular, nested in whatever procedures,
%   anonymous blocks, or exception scopes surround this point. *)

%==============================================================================
\Subsection{Declaring Exceptions}
The routines in this section are for declaring exceptions.  Modula-3
has entities of Exception type; C++ never actually creates an
exception entity.  Rather than pass an entity of type Exception, when a
C++ program throws an exception, it passes the arguments to an
exception handler.

%--
\interface{NameID}{DeclException}{Identifier name, TypeID type}
	{}{ExceptionDecl}
\begin{functionality}
For exceptions without an argument, use \node{NoType} as the
\args{type} argument.
\end{functionality}

	%declare_exception (n: Name;  arg_type: TypeUID;  raise_proc: BOOLEAN;
	%                     base: Var;  offset: INTEGER);

	%(* declares exception 'n' identified with the address 'base+offset'
	% that carries an argument of type 'arg_type'.  If 'raise_proc', then
	% 'base+offset+BYTESIZE(ADDRESS)' is a pointer to the procedure that
	% packages the argument and calls the runtime to raise the exception.*)
   
%==============================================================================
\Subsection{Declaring Labels}

C and C++ treat case alternatives as labels.  This bizarre notion
stems from their excessively low-level view of the switch statement.  
The generation interface handles case alternatives differently from
general labels.  

%--
\interface{NameID}{DeclLabel}{Identifier i}{}{LabelDecl}
	%(*------------------------------------------------- ID counters ---*)
	%next_label (n: INTEGER := 1): Label;
	%(* reserve 'n' consecutive labels and return the first one *)

%==============================================================================
\Subsection{Code Units}\label{sec:unit}
A code unit groups related code.  Code units include modules and
namespaces.  

% Module names are declared, whereas file names are not
% visible to the program text.

%------------------------------------------------------------------------------
\subsubsection{Namespace}\label{sec:namespace}
A C++ namespace can be thought of as a named scope.  

%--
\interface{void}{DeclNamespaceUnitBegin}{}{}{}
%%--
%\interface{NameID}{DeclAnonNamespaceBegin}{}{}{}
%--
\interface{NameID}{DeclNamespaceUnitEnd}{Identifier name}
	{\manyPops}{NamespaceDecl nd}
\begin{functionality}
\befunc{DeclNamespaceUnit}{\node{Declaration}}
\end{functionality}

%------------------------------------------------------------------------------
\subsubsection{File}
These routines are meant for demarcating a C/C++ file scope.  These
routines are not meant for annotating a graph with which source file a
particular piece of code originates from.  
%--
\interface{void}{FileUnitBegin}{}{}{}
%--
\interface{void}{FileUnitEnd}{String name}{\manyPops}{Unit}
\begin{functionality}
\befunc{FileUnit}{\node{Declaration}}
\end{functionality}

%------------------------------------------------------------------------------
\subsubsection{Interface}
%--
\interface{void}{DeclInterfaceUnitBegin}{}{}{}
%--
\interface{NameID}{DeclInterfaceUnitEnd}{Identifier name}
	{\manyPops}{InterfaceDecl}
\begin{functionality}
\befunc{DeclInterfaceUnit}{\node{Declaration}}
\end{functionality}

%------------------------------------------------------------------------------
\subsubsection{Module}
%--
\interface{void}{ImportUnit}{NameID interface, Identifier name}{}{Import}
\begin{functionality}
This routine imports interface \code{i} as \code{name}.  If
\code{name} is NoName, then the interface is not renamed.
\end{functionality}
%--
\interface{void}{ImportMember}{NameID interface, NameID declaration}
	{}{Import}
\begin{functionality}
This routine imports a single member (\code{declaration}) of
\code{interface}.  Note that this routine accepts more information
(\ie \code{interface}) than it strictly needs.
\end{functionality}
%--
\interface{void}{ImportsBegin}{}{}{}
%--
\interface{void}{ImportsEnd}{}{\manyPops}{Imports}
\begin{functionality}
\befunc{Imports}{\node{Import}}
\end{functionality}

%--
\interface{void}{ExportsBegin}{}{}{}
%--
\interface{void}{ExportsEnd}{}{\manyPops}{Exports}
\begin{functionality}
\befunc{Exports}{\node{Identifier}}
\end{functionality}

%--
\interface{void}{DeclModuleUnitBegin}{}{}{}
%--
\interface{NameID}{DeclModuleUnitEnd}{Identifier name}{\manyPops}{Unit}
\begin{functionality}
\befunc{DeclModuleUnit}{\node{Declaration}}
\end{functionality}

%==============================================================================
%\Subsection{Miscellaneous Declarations}
%This section includes those declarations that do not warrant their own
%individual sections.  Generally, these are constructs which are
%limited to a single language.  

%------------------------------------------------------------------------------
%\subsubsection{Initializers}
%%--
%\interface{void}{DefaultValue}{}{Declaration d, Expression e}{Declaration}
%\begin{functionality}
%This routine specifies a default value for an entity.  A default value
%is much like an initial value, except that a default value can be
%overridden.  This routine will generally only be used with formal
%parameters.
%\end{functionality}

%%--
%\interface{void}{InitialValue}{}{Declaration d, Expression e}{Declaration}
%\begin{functionality}
%This routine specifies an initial value.  
%\end{functionality}

%==============================================================================
\Subsection{Setting Declaration Attributes}

Declarations may have attributes associated with them.  The
current design specifies the attributes in an enumerated type.  

An alternate design would use a separate routine for each attribute, so
that a compiler/linker could catch non-compliance to the interface.
However, it seems unnecessarily verbose when compared to a single
function with an enumerated type

The attributes break down into roughly five categories.  The first
group of attributes define which entities have values that can change.
The second group of attributes indicate in what part of memory an
entity should be located.  The most important aspect of the location
decision is whether or not the entity's value is saved across
procedure calls.  The third group of attributes determine the
visibility of the entity.  These attributes correspond to C/C++'s
notion of linkage.  The remaining two groups of attributes control
method visibility and overloading.

The \code{DeclAttributes} enumerated type is defined as follows:

\EnumOptions{DeclAttributes}{ cConstantDecl, cMutable, cLocationStack,
    cLocationRegister, cLocationStatic, cLocationInline, cLinkageLocal,
    cLinkageFile, cLinkageGlobal, cLinkageForeign, cPublic, cProtected,
    cPrivate, cAbstract, cNonvirtual, cVirtual, cNoConstructorCalls,
    cNoDestructorCalls, cDestructorCalls }

where:
\begin{Description}
\item [] \partitle{Entity mutability}
The mutability attributes only apply to variables, fields, and routines.
\item [cConstantDecl] Indicates that an entity is a constant.  This
attribute is equivalent to the type attribute of the same name, but
this one is preferred.
\item [cMutable] Indicates than an entity is \emph{not} a constant.
This is the default value for mutability.  It only needs to be
explicitly specified to indicate that a component of a constant
aggregate entity is not constant (as in C++).

\item [] \partitle{Entity location}
The location attributes only apply to variables and fields.
\item [cLocationStack] Marks a data value as being locally 
allocated (\ie on the stack).  Local allocation is the default form of
allocation and so seldom (if ever) needs to be explicitly specified.
This attribute represents C/C++'s \key{auto} construct.
\item [cLocationRegister] Recommends that a value be assigned to a register.
If the value cannot be assigned to a register, it should be allocated
on the stack.
\item [cLocationStatic] Indicates that an entity is assigned to permanently
allocated space.
\item [cLocationInline] Recommends that a procedure be inlined.  This
attribute applies only to procedures and methods.

\item [] \partitle{Entity visibility}
The entity visibility attributes apply to all declarations.  
\item [cLinkageLocal] Indicates that an entity is visible only within
its current scope.  This is the default linkage.
\item [cLinkageFile] Indicates that the entity is visible only within
its file scope.  This attribute is intended to represent one meaning
of the C/C++ \key{static} construct.  Admittedly, this attribute
should be redundant with \code{linkageLocal} when at the file scope
level.  However, this captures the notion of C/C++'s \key{static}
construct; whereas, \code{linkageLocal} should never be used.
\item [cLinkageGlobal] Indicates that an entity has globally visible
in the program.  This attribute implements the common case of C/C++'s
\key{extern} construct.
\item [cLinkageForeign] Indicates that an entity is visible to a
different source language.  This attribute eliminates C++'s
overloading of the \key{extern} keyword.

\item [] \partitle{Method visibility}
The method visibility attributes potentially apply to all
declarations, but for currently support languages, the attributes only
affect fields and methods.
\item [cPublic] Specifies that an identifier is visible outside of its
namespace.  This value is the default.
\item [cProtected] Specifies than an identifier is visible only within
its namespace.  This attribute may only be used for class members, in
which case it denotes the semantics of C++'s \key{protected} construct.
\item [cPrivate] Specifies that an identifier is visible only within
its namespace.

\item [] \partitle{Method overloading}
The method overloading attributes apply only to methods.
\item [cAbstract] Marks a method as undefined for the current class,
and therefore requiring definition in derived classes.  By default,
the generation interface assumes that all methods are fully defined.
\item [cNonvirtual] Indicates that a method may not be overloaded.
\item [cVirtual] Indicates that a method may be overloaded.  This value
is the default.

\item [] \partitle{Automatic methods}
The automatic method attributes apply only to class declarations.
\item [cNoConstructorCalls] Marks a class as not requiring automatic
invocation of its constructor.  This is the default value.
\item [cConstructorCalls] Marks a class as requiring automatic
invocation of its constructor.
\item [cNoDestructorCalls] Marks a class as requiring automatic
invocation of its destructor.  This is the default value.
\item [cDestructorCalls] Marks a class as requiring automatic
invocation of its destructor.  Note that automatic constructor and destructor
are indicated separately in order to support Java which does not
have destructors.
\item [cMainProcedure] Marks the procedure as the main procedure in
the program.

\end{Description}

%--
\interface{void}{SetDeclarationAttribute}{DeclAttributes da}
	{Declaration d}{Declaration}
\begin{functionality}
This routine assigns attribute \code{da} to declaration \node{d}.
For those attributes which do not require a value, use NoExpression.  
\end{functionality}


