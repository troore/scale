% -*- Mode: latex; Mode: auto-fill; -*-
\Section{Annotations}\label{sec:annote}

The generation interface allows user code to pass information that
does not directly correspond to the structure of the program.  We
refer to this information as annotations.  The interface currently
provides two kinds of annotations: point annotations and range
annotations.  A \emph{point annotation} applies to a specific piece of
the program (\eg an expression or function).  A \emph{range
annotation} may apply to several pieces of the program and is marked
with a begin/end pair.  

\Subsection{Point Annotations}
A point annotation applies to the top element of the stack.  A point
annotation may modify the top element of the stack but does not
(permanently) remove it.

\interface{void}{SetSourceLine}{int line}{}{Node}

\Subsection{Range Annotations}
A range annotation applies to all elements pushed onto the stack after
the begin routine and before the corresponding end routine.

\interface{void}{SourceFileBegin}{String filename}{}{}
\interface{void}{SourceFileEnd}{String filename}{\manyPops}{}
\begin{functionality}
Strictly speaking, the file name would not have to be specified at
both the beginning and end, but doing so may help in debugging.
\end{functionality}


