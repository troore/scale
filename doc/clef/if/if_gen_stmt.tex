% -*- Mode: latex; Mode: auto-fill; -*-

\Section{Generating Statements}\label{sec:stmt}
%(*----------------------------------------------------------- statements ---*)

This section describes the routines available for representing
statements.  

%(*------------------------------------------------------------ load/store ---*)
%(* Note: When an Int_A..Int_D value is loaded, it is sign extended to
%   full Int width.  When a Word_A..Word_D value is loaded, it is zero
%   extended to full Word width.  When an Int_A..Int_D or Word_A..Word_D
%   value is stored, it is truncated from full width to the indicated size.

%   Note: all loads and stores are aligned according to the specified type.
%*)

%load (v: Var;  o: ByteOffset;  t: MType);
%(* push; s0.t := Mem [ ADR(v) + o ].t *)

%load_address (v: Var;  o: ByteOffset := 0);
%(* push; s0.A := ADR(v) + o *)

%load_indirect (o: ByteOffset;  t: MType);
%(* s0.t := Mem [s0.A + o].t *)

%qualify_expr (holder: TypeUID; field: TEXT; 
%t: Type; lval: BOOLEAN; obj: BOOLEAN);

%osubscript_expr (t: Type; elt_size: INTEGER);
%(* s1 := s0[s1]; s0 is an open array type. pop *)

%(* Note that in osubscript and fsubscript, the order of array and index
%   on the stack is different.  Unfortunate... *)
%fsubscript_expr (t: Type; elt_size: INTEGER);
%(* s0 := s1[s0]; s1 is a fixed array type. pop *)

%array_size (n: INTEGER);
%(* s0 := number of elements in nth dimension of open array s0 *)

%store (v: Var;  o: ByteOffset;  t: MType);
%(* Mem [ ADR(v) + o : s ].t := s0.t; pop *)

%store_indirect (o: ByteOffset;  t: MType);
%(* Mem [s1.A + o].t := s0.t; pop (2) *)


%store_ref (v: Var;  o: ByteOffset := 0);
%(* == store (v, o, Type.Addr), but also does reference counting *)

%store_ref_indirect (o: ByteOffset;  var: BOOLEAN);
%(* == store_indirect (o, Type.Addr), but also does reference counting.
%     If "var" is true, then reference counting depends on whether the
%     effective address is in the heap or stack. *)

%==============================================================================
\Subsection{Block Statements}
Different programming languages have different rules about when a
block opens a scope.  The generation interface separates the issues of
grouping statements from that of declaring scopes (see
Section~\ref{sec:symbolTable}.  Hence, none of the routines in this
section imply the creation of a new scope.  A block of statements
may have an associated scope (\eg a procedure
with local variables).  Client code is responsible for
associating a scope with a block of statements.

\interface{void}{StmtBlockBegin}{}{}{}
\interface{void}{StmtBlockEnd}{}{\manyPops}{Statement}
\begin{functionality}
\befunc{StmtBlock}{\node{Statement}}
\end{functionality}

%==============================================================================
\Subsection{Labeled Statements}\label{sec:label}
This section describes routines for representing labeled statements.

\interface{void}{StmtLabel}{}{LabelDecl, Statement}{LabelStmt}

%==============================================================================
\Subsection{Conditional Statements}\label{sec:cond}

This section describes routines for representing decision statements.
This section also defines routines for specifying conditions.  

%------------------------------------------------------------------------------
\subsubsection{If statements}
The routines in this section handle \key{if} statements.  The
interface does not provide direct support for \key{else-if} clauses,
so these clauses will have to be transformed to nested \key{if}
statements.  The design of these routines require that for nested if
statements, outer \key{if} statements remain on the stack until inner
\key{if} statements have been processed.


%--
\interface{void}{StmtIfThenElse}{}
	{Expression e, Statement s1, Statement s2}{Statement}
\begin{functionality}
The expression representing the test must be of boolean type.  If the
statement does not have an \key{else} clause, then a null statement
should be used for \node{s2}.
\end{functionality}

%--
\interface{void}{StmtArithmeticIf}{NameID lessLabel, NameID equalLabel,
	NameID moreLabel}{Expression e}{Statement}
\begin{functionality}
This routine implements the semantics of Fortran~77's arithmetic if
statement.  
The expression representing the test must be of boolean type.
\end{functionality}

%------------------------------------------------------------------------------
\subsubsection{Multi-way Branch Statements}
These routines perform a single test and jump to one of potentially
many points in the program.

%--
\interface{void}{LabelsBegin}{}{}{}
%--
\interface{void}{LabelsEnd}{}{}{Labels}
\begin{functionality}
\befunc{Labels}{LabelRef}  
\end{functionality}
%--
\interface{void}{StmtComputedGoto}{}{Labels l, Expression e}{Statement}
\begin{functionality}
\fortranSemantics{computed goto}
It is similar to a switch statement, except that the
target labels are not limited to a single statement of code.
\end{functionality}
%--
\interface{void}{StmtAssignLabel}{NameID value, NameID label}{}{Statement}
\begin{functionality}
\fortranSemantics{assign}

Note that the semantics of this operation could be represented with a
type conversion from integer to label.  However, this would require
creating variables of type label.  Since labels are deprecated, this
more direct representation has been chosen.
\end{functionality}
%--
\interface{void}{StmtAssignedGoto}{}{Expression e, Labels l}{Statement}
\begin{functionality}
\fortranSemantics{assigned goto}
\end{functionality}

%------------------------------------------------------------------------------
\subsubsection{Case/Switch Statements}
The generation interface supports two types of multiple test
conditional statement.  The first form is a well structured statement,
as found in Modula-3.  We refer to this form as a case statement.  The
body of a case statement consists of blocks of code, each of which
when paired with their case keys is called a case alternatives.
Each time a program passes through a case statement, it will execute
zero or one of the case alternatives.  A case alternative is executed
when the value of the conditional expression matches one of its case
keys.  The other form matches the unstructured form of C/C++'s
\key{switch} statement.   This form still uses case keys, but does not
separate the \key{switch} statement body into case alternatives.

The generation interface follows Modula-3's terminology of \key{case}
statement.  Each unique body of code within a \key{case} statement is
termed a case alternative, and each case alternative may have more
than one case key.  A case key is the value which is compared against
the expression at the top of the \key{case} statement.

\interface{void}{CaseKeyBegin}{}{}{}
\interface{void}{CaseKeyEnd}{}{\manyPops}{CaseKeys}
\begin{functionality}
\befunc{CaseKey}{\node{Expression}}
\end{functionality}
\interface{void}{CaseKey}{}{Expression e}{CaseKeys}
\begin{functionality}
The routines \method{CaseKeyBegin} and \method{CaseKeyEnd}
build a list of case keys.  The \method{CaseKey} routine is a short
cut for building a list with one member.
\end{functionality}

\interface{void}{CaseAlt}{}{CaseKeys ck, Statement s}{CaseAlt}
%\interface{void}{CaseAlt}{}{CaseKeys ck, StatementList s}{CaseAlt}

\interface{void}{CaseOthersAlt}{}{StatementList}{CaseAlt}
\begin{functionality}
This routine represents the default case alternative.
\end{functionality}

\interface{void}{CaseBodyBegin}{}{}{}
\interface{void}{CaseBodyEnd}{}{\manyPops}{CaseAlts}
\begin{functionality}
The routines \method{CaseBodyBegin} and \method{CaseBodyEnd}
build the body of a case statement.

\befunc{CaseBody}{\node{CaseAlt}}
\end{functionality}

\interface{void}{StmtCase}{}{Expression e, CaseAlts}{Statement}
\begin{functionality}
This routine implements the semantics of Modula-3's \key{case}
statement, in which the order of the case alternatives does not matter.
The expression representing the test must be of boolean type.
\end{functionality}

\interface{void}{StmtSwitch}{}{Expression e, Statement s}{Statement}
\begin{functionality}
This routine implements the semantics of C/C++'s \key{switch} construct.
The expression representing the test must be of boolean type.
\end{functionality}

%------------------------------------------------------------------------------
\subsubsection{Typecase}
Modula-3's \key{typecase} statement allows a conditional expression
based on an expressions type.  The routines provided for handling a
\key{typecase} mirror those provided for the \key{case} statement.

%--
\interface{void}{TypecaseKey}{TypeID type, NameID variable}
	{}{TypecaseKey}
\begin{functionality}
If \code{variable} is NoName, then no variable has been specified.
\end{functionality}
%--
\interface{void}{TypecaseKeyBegin}{}{}{}
%--
\interface{void}{TypecaseKeyEnd}{}{\manyPops}{TypecaseKeys}
\begin{functionality}
\befunc{TypecaseKey}{\node{TypecaseKeys}}
\end{functionality}

%--
\interface{void}{TypecaseAlt}{}{TypecaseKeys tck, Statement s}{TypecaseAlt}
%\interface{void}{TypecaseAlt}{}{TypecaseKeys tck, StatementList s}{TypecaseAlt}

%--
\interface{void}{TypecaseBodyBegin}{}{}{}
%--
\interface{void}{TypecaseBodyEnd}{}{\manyPops}{TypecaseAlts}
\begin{functionality}
\befunc{TypecaseBody}{\node{TypecaseAlt}}
\end{functionality}
%--
\interface{void}{StmtTypecase}{}{Expression e, TypeCaseAlts}{Statement}
\begin{functionality}
This routine implements the semantics of Modula-3's \key{case}
statement.  In a \key{typecase} the order of the case alternatives
does matter.
The expression representing the test must be of boolean type.
\end{functionality}

%case_jump (READONLY labels: ARRAY OF Label);
%(* tmp := s0.I; pop; GOTO labels[tmp]  (NOTE: no range checking on s0.I) *)

\todo{
if_true  (l: Label;  f: Frequency);
%(* tmp := s0.I; pop; IF (tmp # 0) GOTO l *)

if_false (l: Label;  f: Frequency);
%(* tmp := s0.I; pop; IF (tmp = 0) GOTO l *)
}

%if_eq (l: Label;  t: ZType;  f: Frequency); (*== eq(t); if_true(l,f)*)
%if_ne (l: Label;  t: ZType;  f: Frequency); (*== ne(t); if_true(l,f)*)
%if_gt (l: Label;  t: ZType;  f: Frequency); (*== gt(t); if_true(l,f)*)
%if_ge (l: Label;  t: ZType;  f: Frequency); (*== ge(t); if_true(l,f)*)
%if_lt (l: Label;  t: ZType;  f: Frequency); (*== lt(t); if_true(l,f)*)
%if_le (l: Label;  t: ZType;  f: Frequency); (*== le(t); if_true(l,f)*)

%==============================================================================
\Subsection{Looping Statements}

\interface{void}{StmtWhileLoop}{}{Expression e, Statement s}{Statement}
\begin{functionality}
The expression representing the test must be of boolean type.
\end{functionality}

\interface{void}{StmtRepeatWhileLoop}{}{Statement s, Expression e}{Statement}
\begin{functionality}
The expression representing the test must be of boolean type.
\end{functionality}

\interface{void}{StmtRepeatUntilLoop}{}{Statement s, Expression e}{Statement}
\begin{functionality}
The expression representing the test must be of boolean type.
\end{functionality}

\interface{void}{StmtLoop}{}{Statement s}{Statement}

%--
\interface{void}{StmtDoLoop}{NameID index}{Expression first, 
	Expression last, Expression step, Statement s}{Statement}
\begin{functionality}
This routine represents an iterating loop.  When \args{step} is
negative, the loop terminates when \args{index} becomes lower than
\code{last}.  Otherwise, the loop terminates when \args{index} exceeds
\args{last}.  The \args{first}, \args{last}, and \args{step}
expressions are evaluated once, at entry to the loop.  
\begin{Parameters}
\Param{index} Variable (or other assignable name) being incremented.
\Param{first} Initial value of \code{index}.
\Param{last} Maximum value of \code{index}.
\Param{step} Amount by which to increment \code{index} each time
through the loop.
\end{Parameters}

This routine follows Fortran~77 semantics, which requires that the
index be updated.  However, some source languages do not guarantee
that the value of the index is meaningful after the loop finishes,
which provides additional opportunities for optimizations.  For such
source languages, user code may create a local (\ie temporary)
variable for the loop index.  In this case, the index will have no
uses beyond the loop, so client code may optimize more aggressively.

\end{functionality}
%--
\interface{void}{StmtForLoop}{}
	{Expression d, Expression e, Expression e, Statement s}{Statement}
\begin{functionality}
The expression representing the test must be of boolean type.
\end{functionality}
%==============================================================================
\Subsection{Branch Statements}

This section lists branching statements.  The statements included here
are simple branches, Fortran~77 has several elaborate branch
statements which may be found in Section~\ref{sec:cond}.

\interface{void}{StmtBreak}{}{}{Statement}
\interface{void}{StmtContinue}{}{}{Statement}
\interface{void}{StmtGoto}{NameID label}{}{Statement}
\interface{void}{StmtReturn}{}{Expression e}{Statement}
\begin{functionality}
If the return statement does not have an expression, then
\node{NoExpression} should be used.
\end{functionality}
\interface{void}{StmtExit}{}{Expression e}{Statement}
\begin{functionality}
This routine terminates a program.  If no expression is specified,
\node{NoExpression} should be used.
\end{functionality}
	%exit_proc (t: Type);
	%(* Returns s0.t if t is not Void, otherwise returns no value. *)
	%jump (l: Label);
	%(* GOTO l *)

\interface{void}{StmtThrow}{NameID value}{}{Statement}
\begin{functionality}
This routine throws a C++ style exception and branches to an
appropriate exception handler.
\end{functionality}

\interface{void}{StmtRaise}{NameID exception}{Expression e}{Statement}
\begin{functionality}
This routine raises a Modula-3 style exception and branches to an
appropriate exception handler.  If no expression is available,
\node{NoExpression} should be used.
\end{functionality}

%--
\interface{void}{StmtAlternateReturn}{}{Expression e}{Statement}
\begin{functionality}
This routine implements the semantics of Fortran~77's alternate
return construct.  The expression indexes the special alternate return
parameters.  
\end{functionality}

%==============================================================================
\Subsection{Exception Handling Statements}

Though exception handling is similar in Modula-3 and C++, The
generation interface provides separate routines for these languages.
The primary difference is that a Modula-3 exception passes an entity
of type \node{Exception}, whereas a C++ exception passes an argument
list for an exception handler.  The syntax of their respective
\key{try} blocks is also slightly different.

\partitle{Modula-3 style exceptions}
%--
\interface{void}{ExceptionKey}{NameID e}{}{ExceptionKey}
%--
\interface{void}{ExceptionKeyBegin}{}{}{}
%--
\interface{void}{ExceptionKeyEnd}{}{\manyPops}{ExceptionKeys}
\begin{functionality}
\befunc{ExceptionKey}{\node{ExceptionKey}}
\end{functionality}

%--
\interface{void}{ExceptionHandler}{}{ExceptionKeys eks, Statement s}{Handler}
%--
\interface{NameID}{ExceptionHandlerWithArgument}
	{NameID exception, Identifier i}{Statement s}{Handler}
\begin{functionality}
This routine represents an exception with an argument.  Generation
interface implementations are responsible for generating the
declaration for \args{i}.
\end{functionality}
%--
\interface{void}{ElseHandler}{}{Statement s}{Handler}
\begin{functionality}
This routine implements Modula-3's else handler, which is called for
any exception that reaches it.
\end{functionality}

%--
\interface{void}{HandlerBegin}{}{}{}
%--
\interface{void}{HandlerEnd}{}{\manyPops}{Handlers}
\begin{functionality}
Each try block has a list of elements of type \node{Handler}
associated with it.  

\befunc{Handler}{\node{Handler}}
\end{functionality}

%--
\interface{void}{StmtTryExcept}{}{Statement s, Handlers hs}{Statement}
\begin{functionality}
This routine implements Modula-3's \key{tryExcept} construct.  
\end{functionality}
%--
\interface{void}{StmtTryFinally}{}{Statement s1, Statement s2}{Statement}
\begin{functionality}
This routine implements Modula-3's \key{tryFinally} construct.  
\end{functionality}

\partitle{C++ style exceptions}

%--
\interface{void}{CatchException}{}{FormalDecl vd, Statement stmt}{Catch}
%--
\interface{void}{CatchAll}{}{Statement stmt}{Catch}
\begin{functionality}
This routine represents a catch clause that can handle any exception.
This routine represents C++'s ``...'' notation.
\end{functionality}

%--
\interface{void}{CatchBegin}{}{}{}
%--
\interface{void}{CatchEnd}{}{\manyPops}{Catchers}
\begin{functionality}
Each try block has a list of elements of type \node{Catch}
associated with it.  

\befunc{Catch}{\node{Catch}}
\end{functionality}

%--
\interface{void}{StmtTry}{}{Statement s, Catchers cs}{Statement}
\begin{functionality}
This routine implements C++'s \key{try} construct.  
\end{functionality}

\todo{
(*------------------------------------------------ traps & runtime checks ---*)

assert_fault   ();
narrow_fault   ();
return_fault   ();
case_fault     ();
typecase_fault ();
(* Abort *)

check_nil ();
(* IF (s0.A = NIL) THEN Abort *)

check_lo (READONLY i: Target.Int);
(* IF (s0.I < i) THEN Abort *)

check_hi (READONLY i: Target.Int);
(* IF (i < s0.I) THEN Abort *)

check_range (READONLY a, b: Target.Int);
(* IF (s0.I < a) OR (b < s0.I) THEN Abort *)

check_index ();
(* IF NOT (0 <= s1.I < s0.I) THEN Abort END; pop
   s0.I is guaranteed to be positive so the unsigned
   check (s0.W < s1.W) is sufficient. *)

check_eq ();
(* IF (s0.I # s1.I) THEN Abort;  Pop (2) *)
}

%==============================================================================
\Subsection{Miscellaneous Statements}

\interface{void}{StmtSpcfyUsing}{NameID id}{}{Statement}

\interface{void}{StmtUsingDirective}{NameID namespace}{}{Statement}

\interface{NameID}{StmtWithAlias}{Identifier id}
	{Expression e, Statement s}{Statement}

\interface{void}{StmtEval}{}{Expression e}{Statement}
\begin{functionality}
This routine converts an expression to a statement.  It executes an
expression and then discards the result.  Hence, the expression is
being executed only for its side effect.  This routine implements
Modula-3's \key{eval} construct and is used to represent C/C++'s
implicit conversion from an expression to a statement.  
\end{functionality}

\interface{void}{StmtDeclStmt}{}{Declaration d}{Statement}
\begin{functionality}
This routine converts a declaration to a statement, which allows a
statement block to consist of all statements.
\end{functionality}

\interface{void}{StmtNull}{}{}{Statement}
\begin{functionality}
This routine represents a null statement.  
\end{functionality}

\todo{
set_label (l: Label;  barrier: BOOLEAN := FALSE);
(* define 'l' to be at the current pc, if 'barrier', 'l' bounds an exception
   scope and no code is allowed to migrate past it. *)

%(*---------------------------------------------------------------- misc. ---*)
comment (a, b, c, d: TEXT := NIL);
(* annotate the output with a&b&c&d as a comment.  Note that any of a,b,c or d
   may be NIL. *)
}
