% -*- Mode: latex; Mode: auto-fill; -*-
\Section{Specifying Types}\label{sec:type}

This section describes the routines which user code can use to pass
type information to clients (see Table~\ref{tab:typeConstructors}.
The routines for declaring type names can be found in
Section~\ref{sec:decl}.  Type information is stored in the type table
(see Section~\ref{sec:tables}).

\begin{table}[b]
\centering
\begin{tabular}{|l||c|c|c|}\hline
\multicolumn{4}{|c|}{Type Constructor Correspondences} \\\hline
\textbf{Method Name} & \textbf{C++} & \textbf{Modula-3} & \textbf{Fortran~77}
\\\hline

TypePrimitiveCharacter		& 
	char & 
	char & 
	character \\\hline
TypePrimitiveInteger		
	& int	
	& integer
	& integer \\\hline
TypePrimitiveFixedPoint		
	& \na	
	& \na	
	& \na \\\hline
TypePrimitiveReal			
	& float/double	
	& real/longreal/extended
	& real \\\hline
TypePrimitiveVoid			
	& void	
	& null
	& \na \\\hline
TypePrimitiveBoolean		
	& bool	
	& boolean
	& logical \\\hline
TypeArrayFixed
	& []	
	& array-of	
	& (), $*$ \\\hline
TypeArrayOpen			
	& \na	
	& array-of	
	& \na \\\hline
TypeArrayUnconstrained		
	& \na	
	& \na	
	& \na \\\hline
TypeEnum begin/end
	& enum	
	& \{\}	
	& \na  \\\hline
TypeRecord begin/end
	& struct
	& record	
	& \na \\\hline
TypeIncompleteRecord		
	& struct
	& \na
	& \na \\\hline
TypeUnion begin/end
	& union	
	& \na
	& equivalence \\\hline
TypeIncompleteUnion		
	& union	
	& \na	
	& \na \\\hline
TypeClass begin/end
	& class	
	& object
	& \na \\\hline
TypeIncompleteClass		
	& class	
	& \na
	& \na \\\hline
TypePointer				
	& $*$	
	& \na	
	& \na \\\hline
TypeIndirect			
	& \&
	& ref
	& \na \\\hline
TypeOffset
	& ::* 
	& \na
	& \na \\\hline
TypeProcedureType			
	& ()	
	& procedure	
	& \na \\\hline
TypeSet				
	& \na	
	& set-of
	& \na \\\hline
TypeRange				
	& \na	
	& [..]	
	& \na \\\hline
TypeBrand				
	& \na	
	& branded
	& \na \\\hline
TypePacked				
	& :	
	& bits-for
	& \na \\\hline

\end{tabular}
\caption{\label{tab:typeConstructors}Correspondence between generation 
interface routine names and language type constructors.}
\end{table}
\clearpage


Different languages employ slightly different versions of derived
types and subtypes.  A derived type has all the associated operators
of the parent type, but is not type compatible with other types
derived from the same parent.  A subtype is much like a derived type,
except that a subtype may have additional fields and operators to
those of the parent type.  The generation interface provides type
attributes and branding for representing derived types.  Subtyping is
only available for classes.

%==============================================================================
\Subsection{Primitive Type Constructors}

This section describes routines for declaring primitive language
types.  In term of type representation, these routines specify the
leaves of type DAGs.  Different languages use different names for the
same primitive type, and some languages do not completely specify what
their primitive types mean.  The generation interface requires user
code to indicate what name and format their source code needs.  The
type name is only useful for debugging.\footnote{The name does not
distinguish primitive types.  Hence, implementations should be able to
handle having the same primitive type with two different names.
Without a name, a debugger could not identify (to the user) the
primitive types.}  These routines may be called multiple times with
different values to create different primitive types (\eg C's
\key{short}, \key{int}, and \key{long} are all integers).  

The generation interface offers two mechanisms for defining primitive
types.  User code may either specify the type with constraints on the
number of bits used in the representation, or the user code may use
purely symbolic names for the size of the types.  Symbolic names will
be mapped to the natural size supported by the target architecture.
For symbolic types, their size is specified with a symbolic value from
the following enumerated list:
\EnumOptions{SymbolicSize}{cVeryShort, cShort, cNormal, cLong, cVeryLong}

%------------------------------------------------------------------------------
\subsubsection{Sized Primitive Types}

%--
\interface{TypeID}{TypePrimitiveCharacter}
	{Identifier name, CharacterFormat cf}{}{Type} 
\begin{functionality}
Ideally, the language should not specify a particular character
format.  However, many C programs rely on the ANSI character set, and
Java prescribes the use of Unicode.  Hence, user code may use this
routine to specify the format (and implicitly the size) of the
character set.  If the language does not require a particular format
it may specify \code{Any}.  The generation interface assumes a default
of \code{Any}.

The generation interface provides the following enumeration:
\EnumOptions{CharacterFormat}{cAny, cAnsi, cEbcdic, cUnicode}
\end{functionality}

%--
\interface{TypeID}{TypePrimitiveInteger}{Identifier name, int minBitSize,
	IntegerRepresentation rep}{}{Type}
\begin{functionality}
This routine creates a primitive integer type in the symbol table.
\begin{Parameters}
\Param{name} The name of the type.
\Param{minBitSize} The minimum number of bits required to represent 
this type.
\Param{rep} Machine representation for this integer type.  Its value
comes from the following enumerated type.
\EnumOptions{IntegerRepresentation}{cUnsigned, cTwosComplement}
\end{Parameters}
\end{functionality}

%--
\interface{TypeID}{TypePrimitiveSymbolicInteger}{Identifier name, 
	SymbolicSize size, IntegerRepresentation rep}{}{Type}
\begin{functionality}
This routine creates a primitive integer type in the symbol table.
\begin{Parameters}
\Param{name} The name of the type.
\Param{size} The symbolic size of the integer type.
Our expectation is that for current machines, \code{cVeryShort} equals
eight bits, \code{cShort} equals sixteen bits, \code{cNormal} equals 32
bits, \code{cLong} equals 64 bits, and \code{cVeryLong} equals 128 bits.
\Param{rep} Machine representation for this integer type.  Its value
comes from the following enumerated type.
\EnumOptions{IntegerRepresentation}{cUnsigned, cTwosComplement}
\end{Parameters}
\end{functionality}

%--
\interface{TypeID}{TypePrimitiveFixedPoint}
	{Identifier name, int minBitSize, int minScaleBits}{}{Type}
\begin{functionality}
This routine creates a primitive fixed point type in the symbol table.
\begin{Parameters}
\Param{name} The name of the type.
\Param{minBitSize} The minimum number of bits required to represent 
this type.
\Param{minScaleBits} Minimum number of bits by which to scale the
representation.  This parameter may be negative.  \code{minBitSize}
includes \code{minScaleBits}.
\end{Parameters}
\end{functionality}

%--
\interface{TypeID}{TypePrimitiveReal}{Identifier name, int minBitSize}
	{}{Type}
\begin{functionality}
\begin{Parameters}
\Param{name} The name of the type.
\Param{minBitSize} The minimum number of bits required to represent this
type.
\end{Parameters}
Note that the format of real numbers is considered a back end issue.  
\end{functionality}

%--
\interface{TypeID}{TypePrimitiveSymbolicReal}
	{Identifier name, SymbolicSize size}{}{Type}
\begin{functionality}
\begin{Parameters}
\Param{name} The name of the type.
\Param{size} The symbolic size of the real type.  
\end{Parameters}
Note that the format of real numbers is considered a back end issue.  
\end{functionality}

%--
\interface{TypeID}{TypePrimitiveComplex}
	{Identifier name, int realMinBitSize, int imaginaryMinBitSize}
	{}{Type}
\begin{functionality}
\begin{Parameters}
\Param{name} The name of the type.
\Param{realMinBitSize} The minimum number of bits required to
represent the real part of this type.
\Param{imaginaryMinBitSize} The minimum number of bits required to
represent the imaginary part of this type.
\end{Parameters}
Note that the format of real numbers is considered a back end issue.  
\end{functionality}

%--
\interface{TypeID}{TypePrimitiveVoid}{Identifier name}{}{Type}
\begin{functionality}
\node{Void} is the null type.  
\end{functionality}

%--
\interface{TypeID}{TypePrimitiveBoolean}{Identifier name}{}{Type}
\begin{functionality}
Boolean values are defined in Section~\ref{sec:logicOp}.
\end{functionality}

%An important open issue is how will basic data types be handled in a
%language independent fashion.  Each interface user could declare all
%of their source languages primitive types; however, how will interface
%implementation recognize identical types with different names?  The
%interface could require implementations to seed their symbol tables
%with basic types, but how will users get a handle to these type
%declarations? 

	%(*The debugging information for a type is identified by a gloablly unique
	% 32-bit id generated by the front-end.  The following methods generate
	%   the symbol table entries needed to describe Modula-3 types to the
	%   debugger. *)
		
	%declare_typename (t: TypeUID;  n: Name);
	%(* associate the name 'n' with type 't' *)

	%declare_builtin (t: TypeUID; n: TEXT);
	%(* Declare a builtin type with uid t and name n *)

	%declare_builtin_object (t: TypeUID; n: TEXT; super: TypeUID);
	%(* Declare a builtin object type with uid t and name n *)

%==============================================================================
\Subsection{Type constructors}

Languages such as C and C++ which provide weak type naming
capabilities encourage programmers to use type constructors when
describing the type of non-type entities (\eg variables) instead of
type names.  Implementations may choose to handle such circumstances
by internally creating an anonymous type for the entity.  

%------------------------------------------------------------------------------
\subsubsection{Array Type Constructors}
These routines support the construction of arrays.  In some languages,
arrays carry knowledge of their length, and in other languages (\eg C
and C++) they do not.  Though knowing their length does impact data
layout, it does not affect type equivalence.  Hence, we mark each
array as to whether or not it must carry a length:
%--
\interface{TypeID}{TypeArrayFixed}{bool lengthField}
	{RangeTypes indexType, Type elementType}{Type}
	%declare_array (t, index, elt: TypeUID;  s: BitSize);
\begin{functionality}
This routine constructs a fixed length array.
\begin{Parameters}
\Param{indexType} Indicates the type of the index expression 
(a list of ranges - for a single dimension array it will be a list
containing one range).
\Param{elementType} Indicates the type of the array elements.
\end{Parameters}
\end{functionality}

%--
\interface{TypeID}{TypeArrayUnconstrained}{bool lengthField}
	{RangeTypes indexType, Type elementType}{Type}
\begin{functionality}
This routine represents an unconstrained array type, as found in Ada.
An unconstrained array is an incomplete type, because it lacks an
index range.  Therefore, the type of \node{indexType} should not
include range information (\ie the bounds should be \node{NoBounds}).  
\end{functionality}

%--
\interface{TypeID}{TypeArrayOpen}{bool lengthField}{Type elementType}
	{Type}
	%declare_open_array (t, elt: TypeUID;  s: BitSize);
	%(* s describes the dope vector *)
\begin{functionality}
This routine constructs an open array.  The size of an open array is
determined at runtime but cannot change once set.  An open array acts
like an unconstrained array, where the only unknown entity is the
maximum index value.  
\end{functionality}

%------------------------------------------------------------------------------
\subsubsection{Enumeration Type Constructors}
The generation interface assumes that the front end has assigned a
value to each enumeration item.  This approach frees 
implemenations of the interface from concern about source language
specific vagaries in enumeration element assigment.

%--
\interface{NameID}{DeclEnumElement}{Identifier id}{Expression e}
	{EnumElementDecl}

%--
\interface{void}{TypeEnumBegin}{}{}{}
%--
\interface{TypeID}{TypeEnumEnd}{Identifier id}{\manyPops}{Type}
\begin{functionality}
\befunc{TypeEnum}{\node{EnumElementDecl}}
\end{functionality}

%--
%\interface{void}{Enum}{}{EnumElements elts}{Type}
%	%declare_enum (t: TypeUID; n_elts: INTEGER;  s: BitSize);
%	%declare_enum_elt (n: Name);

%------------------------------------------------------------------------------
\subsubsection{Incomplete Type Constructors}

The generation interface requires that all entities be defined before
being used.  This restriction keeps the interface free of language
specific name resolution rules.  Unfortunately for recursive types,
this restriction implies that user code needs a mechanism for
specifying a type before definining it completely.  The generation
interface uses a mechanism similar to forward declarations found in
many languages.  

The generation interface allows the creation of \emph{incomplete
types}.  An incomplete type does not carry any additional type
information, and therefore may be used anywhere a type is valid.  User
code should build recursive type structures with recursive references
pointing to the incomplete type.  After the recursive type is
completely built, user code should use \method{CompleteType} to
allow implementations of the interface to patch up its representation
to reflect the true recursive structure.  Incomplete types are only to
aid in conveying the actual type structure through the interface.
Hence, user code must associate complete type with each incomplete
type.

For languages such as Modula-3 which do not require forward
declarations for mutually recursive types, user code must generate
incomplete types.  

Incomplete types can be created at any time.  Hence, user code can
begin building an aggregate structure and generate an incomplete type
only if the aggregate type turns out to be recursive.  However, the
user code must be careful to clean up the \node{Type} node on the stack.

%--
\interface{TypeID}{TypeIncompleteType}{Identifier name}{}{Type}
%--
\interface{TypeID}{TypeCompleteType}{TypeID tic, TypeID tc}{}{Type}
\begin{functionality}
This routine completes the declaration of an incomplete type.  It
informs the implementation that incomplete type \code{tic} is really
the completed type \code{tc}.  The \node{Type} node left on the stack
corresponds to \code{tc}.
\begin{Parameters}
\Param{tic} \ret{TypeID} identifying incomplete type.
\Param{tc} \ret{TypeID} identifying complete type.
\end{Parameters}
\end{functionality}

%------------------------------------------------------------------------------
\subsubsection{Aggregate Type Constructors}

%--
\interface{void}{TypeRecordBegin}{}{}{}
%--
\interface{TypeID}{TypeRecordEnd}{}{\manyPops}{Type}
\begin{functionality}
\befunc{TypeRecord}{\node{FieldDecl}}
In C++, a class struct is represented as a \node{ClassType}.
\end{functionality}

%--
\interface{void}{TypeUnionBegin}{}{}{}
%--
\interface{TypeID}{TypeUnionEnd}{}{\manyPops}{Type}
\begin{functionality}
\befunc{TypeUnion}{\node{FieldDecl}}
In C++, a class union is represented as a \node{ClassType}	
\end{functionality}

%--
\interface{void}{Superclass}{NameID class, AccessSpecifier as}{}{SuperClass}
\begin{functionality}
\begin{Parameters}
\Param{class} Indicates the super class.
\Param{as} Indicates the access specifier (as defined by C++).  For
other languages, a suitable value for \code{as} should be selected.
The value for \code{as} comes from the following:
\EnumOptions{AccessSpecifier}{cPrivateAccess, cProtectedAccess, cPublicAccess}
\end{Parameters}
\end{functionality}
%--
\interface{void}{SuperclassBegin}{}{}{}
%--
\interface{void}{SuperclassEnd}{}{\manyPops}{SuperClasses}
\begin{functionality}
\befunc{Superclass}{\node{SuperClass}}
\end{functionality}
%--
\interface{void}{SingleSuperclass}{NameID class, AccessSpecifier as}
	{}{SuperClasses}
\begin{functionality}
This routine is a shortcut for cases when only a single super class
exists (as with single inheritance).
\end{functionality}

%--
\interface{void}{TypeClassBegin}{}{}{}
%--
\interface{TypeID}{TypeClassEnd}{}{SuperClasses, \manyPops}{Type}
	%declare_object (t, super: TypeUID;  brand: TEXT;
	%        traced: BOOLEAN;  n_fields, n_methods, n_overrides: INTEGER;
	%        field_size: BitSize);
	%(* brand=NIL ==> t is unbranded *)
\begin{functionality}
The generation interface uses its own variation of C++ and Modula-3
terminology when discussing objects.  An \emph{object} is an instance
of a \emph{class} datatype.  The components of a class are called
\emph{members}, which may be either data \emph{fields}, \emph{routines},
conversion functions, constructors, and destructors.

\befunc{TypeClass}{\node{Declaration}}
\end{functionality}

%------------------------------------------------------------------------------
\subsubsection{Pointer Type Constructors}
The routines in this section allow the construction of pointer types.

%--
\interface{TypeID}{TypePointer}{}{Type t}{Type}
	%declare_pointer (t, target: TypeUID;  brand: TEXT;  traced: BOOLEAN);
	%(* brand=NIL ==> t is unbranded *)
\begin{functionality}
This routine constructs a pointer type which must be explicitly dereferenced.
\end{functionality}

%--
\interface{TypeID}{TypeIndirect}{}{Type t}{Type}
	%declare_indirect (t, target: TypeUID);
	%(*an automatically dereferenced pointer! (WITH variables, 
	%	VAR formals, ...) *)
\begin{functionality}
This routine describes automatically dereferenced pointers.  In
Modula-3 these pointers are used in implementing \key{with} aliases,
\key{var} formals, etc.  
\end{functionality}
	
%--
\interface{TypeID}{TypeOffset}{NameID aggregate}{Type t}{Type}
\begin{functionality}
This routine represents a pointer-to-member as found in C++.  A
pointer-to-member is an offset to a member of \args{aggregate}.  
\end{functionality}
	
%------------------------------------------------------------------------------
\subsubsection{Procedure Type Constructors}
A method is a procedure who is a member.

\partitle{Parameters}
%--
\interface{void}{FormalsBegin}{}{}{}
%--
\interface{void}{FormalsEnd}{}{\manyPops}{Formals}
\begin{functionality}
\befunc{Formals}{\node{FormalDecl}}
\end{functionality}

\partitle{Exceptions}
%--
\interface{void}{RaiseException}{NameID exception}{}{Raise}
	%declare_raises (n: Name);
\begin{functionality}
This routine represents a Modula-3 style exception, when specifying a
procedure's throw list.
\end{functionality}
%--
\interface{void}{RaiseType}{TypeID type}{}{Raise}
\begin{functionality}
This routine represents a C++ style exception, when specifying a
procedure's throw list.
\end{functionality}
%--
\interface{void}{RaisesAny}{}{}{Raise}
\begin{functionality}
This routine indicates that the associated procedure can raise any
exception.  By default, C++ functions may raise any exception;
Modula-3 procedures must explicitly indicate that they can generate
any exception.
\end{functionality}
%--
\interface{void}{RaisesNone}{}{}{Raises}
\begin{functionality}
This routine indicates that the associated procedure cannot raise any
exceptions.  By default, Modula-3 procedures cannot raise any exceptions;
C++ procedures must explicitly indicate that they cannot raise
exceptions.  Using this routine is equivalent to specifying
\method{RaisesBegin}/\method{RaisesEnd} without any \node{Raise}
nodes in between.
\end{functionality}
%--
\interface{void}{RaisesBegin}{}{}{}
%--
\interface{void}{RaisesEnd}{}{\manyPops}{Raises}
\begin{functionality}
\befunc{Raises}{\node{Raise}}
\end{functionality}

%--
\interface{void}{Signature}{}{Formals f, Type ret, Raises r}{Signature}
\begin{functionality}	
This function defines a routine signature.  All signatures must
indicate what exceptions it can raise.  For those languages which do
not support exceptions, they should simply call \method{RaiseNone}.
\end{functionality}	

%object_info (t: TypeUID);
%(* s1 = method offset, s0 = field offset*)

%--
\interface{TypeID}{TypeProcedure}{}{Signature s}{Type}
	%declare_proctype (t: TypeUID;  n_formals: INTEGER;
	%                  result: TypeUID;  n_raises: INTEGER;
	%                  cc: CallingConvention);
	%(* n_raises < 0 => RAISES ANY *)

	%declare_method (obj: TypeUID; n: Name;  signature: TypeUID; 
	%		offset:INTEGER);

%------------------------------------------------------------------------------
\subsubsection{Set Type Constructors}
The routines in this section permit the construction of set types.
%--
\interface{TypeID}{TypeSet}{}{Type type}{Type}
	%declare_set (t, domain: TypeUID;  s: BitSize);

%------------------------------------------------------------------------------
\subsubsection{Range Type Constructors}\label{sec:range}
\par
\partitle{Bounds} 

Bounds specify a minimum and maximum value.  Bounds are primarily used
for specifying the minimum and maximum for range types.  Support for
defining bounds varies between languages, so the minimum and maximum
values are arbitrary expressions.  We also use ranges to represent
array indicies.  For languages (such as Ada) that support true
multi-dimensional arrays, ranges may be chained together since a range
type contains a single bound.  However, the common case for C/C++ and
Modula-3 is to use a single range.

%--
\interface{void}{Bound}{}{Expression min, Expression max}{Bound}
\begin{functionality}
This routine creates a single bounds which can be chained together
with a begin/end pair.
\end{functionality}
%--
\interface{void}{BoundsBegin}{}{}{}
%--
\interface{void}{BoundsEnd}{}{\manyPops}{Bounds}
\begin{functionality}
\befunc{Bounds}{\node{Bound}}
\end{functionality}
%--
\interface{void}{Bounds}{Expression min, Expression max}{}{Bounds}
\begin{functionality}
This routine is a short cut for specifying a single dimensional bounds
specification.
\end{functionality}
%--
\interface{void}{Nobounds}{}{}{Bound}
\begin{functionality}
This routine pushes onto the stack a special \node{Bounds} that
connotes that no bounds have been specified.  This value 
may be used to construct arrays without bounds.
\end{functionality}
%--
%\interface{bool}{Isnobounds}{}{Bounds}{Bounds}
%\begin{functionality}
%This routine allows user code to test if a bounds is undefined.  The
%\node{Bounds} on the top of the stack is unmodified.
%\end{functionality}

%--
\interface{TypeID}{TypeRange}{}{Type basetype, Bound b}{Type}
	%declare_subrange (t,domain: TypeUID; READONLY min,max: Target.Int; 
	%		s: BitSize);
\begin{functionality}
This routine creates a range type with a single bound.
\end{functionality}
%--
\interface{void}{RangeBegin}{}{}{}
%--
\interface{void}{RangeEnd}{}{\manyPops}{RangeTypes}
\begin{functionality}
\befunc{RangeType}{\node{RangeType}}
\end{functionality}

%------------------------------------------------------------------------------
\subsubsection{Branded Type Constructors}
Branded type constructors are unique in that though they do indeed
introduce a new type, they do not change the structure of the type.
Branding is useful in languages with structural equivalence (\eg
Modula-3).
%--
\interface{TypeID}{TypeBrand}{}{Type t, Expression b}{Type}
\begin{functionality}
This routine creates a new type which is structurally identical to
type \code{t}.  
\end{functionality}

%------------------------------------------------------------------------------
\subsubsection{Packed Type Constructors}
The generation interface treats packed types a unique types.  For
C/C++ this distinction between types would not be necessary.  However,
other languages such as Modula-3 clearly distinguish between a base
type and its packed version.  This distinction implies that user code
must insert explicit conversions between the base type and packed
type.

%--
\interface{TypeID}{TypePacked}{}{Type basetype, Expression bitSize}{Type}
\begin{functionality}
This routine builds a new type which compresses type \code{basetype}
into \code{bitSize} bits.  
\end{functionality}

%------------------------------------------------------------------------------
\subsubsection{Alias Type Constructors}

C++ provides a type constructor for defining alias.  C++ calls these
aliases \emph{references}.  An alias does not have memory space of its
own, but rather refers to another entity's memory.  

I've considered representing references as an alias declaration which
looks better, but return types can be references.  I've also
considered treating references as indirect pointers.  However, this
approach might hurt alias analysis.

%--
\interface{TypeID}{TypeAlias}{}{Type type}{Type}

%==============================================================================
\Subsection{Type Attributes}\label{sec:typeAttributes}
This section describes how to associate an attribute with a type.
Attributes are non-type information that is associated with a type.
Attributes are associated with a type rather than a \node{TypeDecl}
because they can be associated with just a part of a type declaration.
Attributes should generally not affect type equivalence (except
perhaps in some minor cases).  The notion of associating an attribute
with a type has strong implications for the implemenation.  Ideally,
equivalent types would share the same physical representation.
However, types that differ only in attributes should be equivalent,
yet cannot share attributes.

The possible attributes are provided by the following enumeration:
\EnumOptions{TypeAttribute}{ cTraced, cUntraced, cConstantType, cVolatile,
			  cOrdered, cUnordered }
where:
\begin{Description}
%\item [brand] Supports Modula-3's branding of data types.
\item [cTraced] Supports Modula-3's traced data types.  
\item [cUntraced] Supports Modula-3's untraced data types.  This is the
default attribute for all types.
%\item [constructor] Indicates that a method is an object constructor.
%\item [destructor] Indicates that a method is an object destructor.
%\item [conversion] Marks a routine as a user-defined type conversion routine.
%\item [abstract] Marks a method as undefined for the current class,
%and therefore requiring definition in derived classes.
%\item [packed] Indicates that a type should use less than the
%natural number of bits.  This attribute has a value: the size of the
%bit field.

%This attribute should be sufficient to handle both Modula-3 packed types
%and C++ bit fields.  

%\item [auto] Marks a data value as being locally allocated (\ie on the
%stack).  Local allocation is the default form of allocation and so
%seldom needs to be explicitly specified.
%\item [register] Recommends that a value be assigned to a register.
%If the value cannot be assigned to a register, it should be allocated
%on the stack.
%\item [static] Indicates that an entity is assigned to permanently
%allocated space.
%\item [globalLinkage] Indicates that an entity has globally visible
%in the program.  This attribute implements the common case of C/C++'s
%\key{extern} construct.
%\item [foreignLinkage] Indicates that an entity is visible to a
%different source language.  This attribute eliminates C++'s
%overloading of the \key{extern} keyword.
%\item [fileLinkage] Indicates than an entity is visible only within
%the current file.
%\item [public] Specifies that an identifier is visible outside of its
%namespace.  
%\item [protected] Specifies than an identifier is visible only within
%its namespace.  This attribute may only be used for class members, in
%which case it denotes the semantics of C++'s \key{protected} construct.
%\item [private] Specifies that an identifier is visible only within
%its namespace.
\item [cConstantType] Indicates that instances of this type have a constant
value.  By default, instances of a type are mutable.

One could reasonably argue that the immutability of a value is not a
property (or attribute) of a type.  However, C++'s typedef construct
permits immutability to be included with the type.
%\item [mutable] Indicates than instances of this type are \emph{not} 
%constant.  This value is the default for all types and should
%therefore never actually be specified.
%The primary purpose of this attribute is to indicate that a component
%of a composite constant entity is not constant.
%\item [friend] Represents C++'s \key{friend} construct.
%\item [nonvirtual] Indicates that a method may not be overloaded.
%\item [virtual] Indicates that a method may be overloaded.
\item [cVolatile] Marks a value which may be changed by something which
a compiler cannot detect.
\item [cOrdered] For types with substructures, this attribute indicates
if the source language requires the data layout to preserve the order
of substructures (declaration order is assumed).  
\item [cUnordered] For types with substructures, this attribute indicates
if the type may be layed out in an arbitrary order.
\end{Description}

%--
\interface{void}{SetTypeAttribute}{TypeAttributes ta}{Type t}{Type}
\begin{functionality}
This routine associates an attribute with a type.
\end{functionality}
%--
%\interface{void}{SetTypeAttributeWithValue}{}
%	{Type t, TypeAttributes ta, Expression e}{Type}
%\begin{functionality}
%This routine associates an attribute with a type when the attribute
%has a value.
%\begin{Parameters}
%\Param{t} The type to which the attribute is being associated.
%\Param{ta} The attribute which is an element of the
%\code{TypeAttribute} enumeration.
%\Param{e} The attributes value.
%\end{Parameters}
%\end{functionality}

