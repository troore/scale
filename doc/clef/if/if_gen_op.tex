% -*- Mode: latex; Mode: auto-fill; -*-

\Section{Generating Expressions}\label{sec:op}

This section shows the interface routines used to specify expressions.
Note that the interface implementation is responsible for identifying
type specific versions of the operators.

Operators can represent user-defined functions in languages which
allow operators to be overloaded (\eg C++).  In the generation
interface, operators represent only language defined operators, not
overloaded operators.  Users should translate overloaded operators to
function calls.

Tables~\ref{tab:ops1},~\ref{tab:ops2}, and~\ref{tab:ops3} show the
correspondences between the interface's routines and operators of a few
languages.  This table shows correspondences between operators which
programmers may use and the routines of the generation interface.
However, user code may use operators which the programmer cannot
directly use (\eg C/C++ implicitly truncate real numbers but users of
this interface must explicitly call a conversion routine).

\begin{table}
\centering
\begin{tabular}{|l||c|c|c|}\hline
\multicolumn{4}{|c|}{Operator Correspondences-- Part I} \\\hline
\textbf{Routine Name} & \textbf{C++} & \textbf{Modula-3} & \textbf{Fortran~77}
\\\hline
\multicolumn{4}{|c|}{\tabhead{Assignment Operators}} \\\hline
ExpAssignment 			& =	& :=	& =	\\\hline
\multicolumn{4}{|c|}{\tabhead{Arithmetic Operators}} \\\hline
ExpPositive			& +	& +	& \na	\\\hline
ExpNegative			& $-$	& $-$	& $-$	\\\hline
ExpAbsoluteValue			& \na	& abs	& abs, ?abs \\\hline
ExpMinimum				& \na	& min	& min, ?min$*$ \\\hline
ExpMaximum				& \na	& max	& max, ?max$*$ \\\hline
ExpAddition			& +	& +	& +	\\\hline
ExpSubtraction 			& $-$	& $-$	& $-$	\\\hline
ExpMultiplication			& $*$	& $*$	& $*$	\\\hline
ExpDivision 			& /	& /	& /	\\\hline
ExpModulus 			& \na	& \na 	& \na 	\\\hline
ExpRemainder 			& \%	& mod	& mod, ?mod \\\hline
ExpExponentiation			& \na	& \na	& $**$	\\\hline
ExpPreDecrement			& $--$	& \na 	& \na	\\\hline
ExpPreIncrement			& ++	& \na	& \na	\\\hline
ExpPostDecrement 		& $--$	& \na	& \na	\\\hline
ExpPostIncrement 		& ++	& \na	& \na	\\\hline
\multicolumn{4}{|c|}{\tabhead{Relational Operators}} \\\hline
ExpEquality 			& ==	& =	& .EQ.	\\\hline
ExpNotEqual 			& !=	& \#	& .NE.	\\\hline
ExpGreater 			& $>$	& $>$	& .GT.	\\\hline
ExpGreaterEqual			& $>=$	& $>=$	& .GE.	\\\hline
ExpLess				& $<$	& $<$	& .LT.	\\\hline
ExpLessEqual 			& $<$	& $<=$	& .LE.	\\\hline
\multicolumn{4}{|c|}{\tabhead{Bitwise Operators}} \\\hline
ExpBitComplement 		& $\sim$& \na	& \na	\\\hline
ExpBitAnd 			& \&	& \na	& \na	\\\hline
ExpBitXor 			& $\wedge$& \na	& \na	\\\hline
ExpBitOr 			& $\mid$& \na	& \na	\\\hline
ExpBitShiftLeft 		& $<<$	& \na	& \na	\\\hline
ExpBitShiftRight 		& $>>$	& \na	& \na	\\\hline
\end{tabular}
\caption{\label{tab:ops1}Correspondence between generation interface
ExpRoutine names and language operators.  
ExpContinued in Table~\protect\ref{tab:ops2}.}
\end{table}

%--

\begin{table}
\centering
\begin{tabular}{|l||c|c|c|}\hline
\multicolumn{4}{|c|}{Operator Correspondences-- Part II} \\\hline
\textbf{Routine Name} & \textbf{C++} & \textbf{Modula-3} & \textbf{Fortran~77}
\\\hline
\multicolumn{4}{|c|}{\tabhead{Compound Assignment Operators}} \\\hline
ExpMultiplicationAssignment 	& $*=$	& \na	& \na	\\\hline
ExpDivisionAssignment		& /=	& \na	& \na	\\\hline
ExpRemainderAssignment		& \%=	& \na	& \na	\\\hline
ExpAdditionAssignment		& +=	& \na	& \na	\\\hline
ExpSubtractionAssignment 	& $-\!=$& \na	& \na	\\\hline
ExpBitShiftRightAssignment	& $>>$=	& \na	& \na	\\\hline
ExpBitShiftLeftAssignment	& $<<$=	& \na	& \na	\\\hline
ExpBitAndAssignment		& \&=	& \na	& \na	\\\hline
ExpBitXorAssignment		& $\wedge\!=$& \na & \na\\\hline
ExpBitOrAssignment 		& $\mid=$& \na	& \na	\\\hline
\multicolumn{4}{|c|}{\tabhead{Logical Operators}} \\\hline
ExpNot				& !	& not	& .NOT.	\\\hline
ExpAnd				& \na	& \na	& .AND.	\\\hline
ExpOr				& \na	& \na	& .OR.	\\\hline
ExpAndConditional 		& \&\&	& and	& \na	\\\hline
ExpOrConditional 		& $\mid\mid$& or & \na	\\\hline
ExpArithmeticIf			& ?:	& \na	& \na	\\\hline
\multicolumn{4}{|c|}{\tabhead{Set Operators}} \\\hline
ExpSetEquality			& \na	& =	& \na	\\\hline
ExpUnion				& \na	& +	& \na	\\\hline
ExpDifference 			& \na	& $-$	& \na	\\\hline
ExpIntersection			& \na	& $*$	& \na	\\\hline
ExpSymmetricDifference		& \na	& /	& \na	\\\hline
ExpSubset				& \na	& $<\!=$& \na	\\\hline
ExpSuperset 			& \na	& $>\!=$& \na	\\\hline
ExpElement 			& \na	& in	& \na	\\\hline
\multicolumn{4}{|c|}{\tabhead{Pointer Operators}} \\\hline
ExpAddress 			& \&	& adr	& \na	\\\hline
ExpDereference 			& $*$	& $\sim$& \na	\\\hline
ExpNil	 			& \na	& \textsl{nil}&	\na \\\hline
\multicolumn{4}{|c|}{\tabhead{Aggregate Operators}} \\\hline
ExpThis				& this	& \na	& \na	\\\hline
ExpSelect				& .	& .	& \na	\\\hline
ExpSelectIndirect 		& $-\!>$& .	& \na	\\\hline
ExpSelectRelative	 	& $.*$	& \na	& \na	\\\hline
ExpSelectRelativeIndirect		& $-\!>\!*$& \na & \na	\\\hline
\end{tabular}
\caption{\label{tab:ops2}Correspondence between generation interface
ExpRoutine names and language operators.
ExpContinued from Table~\protect\ref{tab:ops1}.  
ExpContinued in Table~\protect\ref{tab:ops3}.}
\end{table}

%--

\begin{table}
\centering
\begin{tabular}{|l||c|c|c|}\hline
\multicolumn{4}{|c|}{Operator Correspondences-- Part III} \\\hline
\textbf{Routine Name} & \textbf{C++} & \textbf{Modula-3} & \textbf{Fortran~77}
\\\hline
\multicolumn{4}{|c|}{\tabhead{Array Operators}} \\\hline
ExpSubscript 			& $[]$	& $[]$	& ()	\\\hline
ExpArrayEquality			& \na	& =	& .EQ.	\\\hline
ExpArrayInequality			& \na	& \#	& .NE.	\\\hline
ExpArrayGreater			& \na	& \na	& .GT.	\\\hline
ExpArrayGreaterEqual		& \na	& \na	& .GE.	\\\hline
ExpArrayLess			& \na	& \na	& .LT.	\\\hline
ExpArrayLessEqual			& \na	& \na	& .LE.	\\\hline
ExpSlice				& \na	& subarray & : 	\\\hline
ExpRemainingSlide			& \na	& \na	& : 	\\\hline
ExpConcatenation			& \na	& \na	& //	\\\hline
\multicolumn{4}{|c|}{\tabhead{Invocation Operators}} \\\hline
ExpCallProcedure			& ()	& ()	& ()	\\\hline
ExpCallMethod 			& () 	& () 	& \na 	\\\hline
\multicolumn{4}{|c|}{\tabhead{Heap Operators}} \\\hline
ExpAllocate 			& new	& new	& \na	\\\hline
ExpDelete				& delete& dispose & \na	\\\hline
ExpDeleteArray			& delete[]& dispose &\na\\\hline
\multicolumn{4}{|c|}{\tabhead{Type Operators}} \\\hline
ExpBytesizeVariable		& sizeof& bytesize &\na	\\\hline
ExpBytesizeType			& sizeof& bytesize &\na	\\\hline
ExpBitsizeVariable			& \na	& bitsize & \na	\\\hline
ExpBitsizeType 			& \na	& bitsize & \na	\\\hline
ExpAdrsizeVariable		& \na	& adrsize & \na	\\\hline
ExpAdrsizeType 			& \na	& adrsize & \na	\\\hline
ExpIstype				& \na	& istype & \na	\\\hline
ExpNarrow				& \na	& narrow & \na	\\\hline
ExpTypecode 			& \na	& typecode &\na	\\\hline
%Ordinal 			& \na	& ord	& \na	\\\hline
%Value				& \na	& val	& \na	\\\hline
ExpNumber				& \na	& number & \na 	\\\hline
ExpFirst				& \na	& first	& \na	\\\hline
ExpLast				& \na	& last	& \na	\\\hline
\multicolumn{4}{|c|}{\tabhead{Type Conversion Operators}} \\\hline
ExpTypeConversion 		& ()	& \na	& \na	\\\hline
\multicolumn{4}{|c|}{\tabhead{Miscellaneous Operators}} \\\hline
ExpSeries				& ,	& \na 	& \na	\\\hline
ExpParentheses 			& ()	& ()	& ()	\\\hline
ExpSelectScope			& ::	& .	& \na	\\\hline
ExpAggregation pair		& \{\}	& \{\}	& \na	\\\hline
\end{tabular}
\caption{\label{tab:ops3}Correspondence between generation interface
routine names and language operators.
Continued from Table~\protect\ref{tab:ops2}.  }
\end{table}

\clearpage


Operators are first divided by the types of their arguments and then
by their type of function.  Calls to overloaded operators are
represented as routine calls, since user-defined operators may not fit
neatly in our categories.  

The generation interface does not have any rules for type conversion.
Therefore, user code is responsible for inserting explicit type
conversions.  Unless otherwise stated, binary operators require both
operands to be of the same type, and operators return an entity of the
same type as their operands.

%==============================================================================
\Subsection{Base Expressions}

A base expression is an expression without subexpressions.  Some base
expressions are listed in following sections with related operators.

%------------------------------------------------------------------------------
\subsubsection{Identifier Reference}
\interface{void}{ExpIdReference}{NameID entity}{}{Expression}
\begin{functionality}
This routine represents the use of a declared entity.
\end{functionality}

\partitle{Scope operators}
The scope operator allows programmers to access entities which are
hidden by other uses of the same identifier.  Resolving which
declared entity is intended by a particular reference is part of semantic
resolution, and therefore handled by the language parser.  Hence,
these routines are superfluous and only exist for providing additional
information to client code.  Note that ultimately user code must call the
\method{ExpIdReference} routine to gain access the entity's value.

\interface{NameID}{SelectScope}{NameID entity1, NameID entity2}
	{}{Reference}
\begin{functionality}
This operator searches the scope named by \code{entity1} for
\code{entity2}.  Note that if \code{entity2} is a type, this returns a
reference to the type's declaration, not the actual type.
\end{functionality}

\interface{NameID}{SelectGlobalScope}{NameID entity}{}{Reference}
\begin{functionality}
This operator searches the global scope named for identifier \code{entity}.
Note that if \code{entity2} is a type, this returns a
reference to the type's declaration, not the actual type.
\end{functionality}

%------------------------------------------------------------------------------
\subsubsection{Literals}
This section describes routines for expressing literals.  

\interface{void}{ExpLiteral}{TypeID type, String value}{}{Expression}
\begin{functionality}
Literals are difficult to handle since we may be cross compiling.
Hence, all literals are transfered across the interface as a String
constant.  Implementations of the interface must be able to convert
numeric constants from strings to a numeric value.  Equality of
literals should be determined by comparing numeric values, rather than
string values.
\end{functionality}

%%--
%\interface{void}{ExpConstruct}{}{Type t, Expression literal}{Expression}
%\begin{functionality}
%This routine constructs an unnamed entity of a given type from a
%literal value(s).  This routine handles the invocation of a type
%\end{functionality}

%------------------------------------------------------------------------------
\subsubsection{No Expression}

This section describes the special value, \node{NoExpression}.  This
node represents that an optional expression has not been specified.
This node may only replace a genuine expression node in those cases
where it is explicitly permitted.  This node does not correspond to
any source language construct.  

%--
\interface{void}{ExpNoexpression}{}{}{Expression}
\begin{functionality}
This routine pushes a special expression node onto the stack.  The
implementation is capable of distinguishing this special expression
node as not representing a valid expression.
\end{functionality}

%==============================================================================
\Subsection{Expression Ordering Operators}
The operators represented by routines in this section serve only order
and structure other expressions.  The functions of these operators are
independent of they type of their arguments.

\interface{void}{ExpSeries}{}{Expression e1, Expression e2}{Expression}
\begin{functionality}
This routine implementes C/C++'s comma operator.
\end{functionality}

\interface{void}{ExpParentheses}{}{Expression e}{Expression}
\begin{functionality}
This routine indicates that the programmer enclosed the associated
expression in parentheses.  The generation interface does not need
parentheses to override precendence rules.  However also specify the
order in which computations are performed.  This information may be
useful to optimizations which affect expression ordering.  
\end{functionality}

\partitle{Aggregation} 
Some languages allow instances of aggregate and array types to be
assigned to by aggrregate values.  These routines allow the
structuring of expressions into aggregate expressions.  Every
aggregate element has a position associated with it.  The generation
interface only supports positional specification of aggregate
elements.  User code must convert keyword specifications to positional
specifications.  

%--
\interface{void}{ExpPositionSingle}{}{Expression e}{Position}
\begin{functionality}
This routine builds a \node{Position} node that represents a single
position.  
\end{functionality}
%--
\interface{void}{ExpPositionRange}{}{Bounds b}{Position}
\begin{functionality}
This routine builds a \node{Position} node that represents a range of
positions.  
\end{functionality}
%--
\interface{void}{ExpPositionAny}{}{Bounds b}{Position}
\begin{functionality}
This routine builds a \node{Position} node that represents any
position.  This value corresponds to Ada's \key{others} construct in
an aggregate.  Any has lower priority than other positions; it fills
in those positions that no other aggregation element does.
\end{functionality}

%--
\interface{void}{ExpAggregationElement}{}
	{Position p, Expression e}{AggregationElement}
\begin{functionality}
This routine builds an element of an aggregation.  Each aggregation
element must specify its position in the final data type, but its
position may be a range or \emph{any}.
\end{functionality}

%--
\interface{void}{ExpAggregationBegin}{}{Type t}{Type t}
\interface{void}{ExpAggregationEnd}{}{Type t, \manyPops}{Expression}
\begin{functionality}
\befunc{ExpAggregation}{\node{AggregationElement}}  The resulting expression is
of type \args{t}.  Note that the type can be restricted to nodes of
\node{CompositeType} type.  
\end{functionality}

%==============================================================================
\Subsection{Assignment Operators}\label{sec:assignOp}

%The assignment operator works on all types.  For singleton types, it
%performs a simple copy, and for aggregate types, it does a component
%by component copy.  C/C++ does not allow arrays to be assigned as a
%whole, so the assignment operator uses Modula-3 semantics for array
%assignment.  

%--
\interface{void}{ExpAssignSimple}{}
	{Expression e1, Expression e2}{Expression}
\begin{functionality}
This routine works for all datatypes.  It performs a simple bit copy
of \code{e1} (rvalue) into \code{e2} (lvalue) for
\method{bitsize}\code{(e2)} bits.  This routine is best used for
singleton values, and the other assignment routines for aggregates and
arrays.  Note that for all assignment operators, we push the rvalue
before lvalue.
\end{functionality}
%--
\interface{void}{ExpAssignComponents}{}
	{Expression e1, Expression e2}{Expression}
\begin{functionality}
This routine does a component-wise copy for aggregates and arrays.
Hence, this routine is sensitive to the types of its arguments.  This
routine requires array arguments to be of the same shape and size and
will insert code to ensure run-time compliance, which is to say that
it follows Modula-3 semantics.
\end{functionality}
%--
\interface{void}{ExpAssignFixedString}{LengthFunction l}
	{Expression e1, Expression e2, Expression padding}{Expression}
\begin{functionality}
This routine provides string-like assignment for fixed sized arrays
(\eg static and open arrays).  This function allows
assignment of arrays of unequal lengths.  If the source expression is
longer than the target, only enough elements are copied to fill the
target.  If the target expression is longer than the source
expression, then the target expression is padded.
\begin{Parameters}
\Param{l} Indicates how to determine the the dynamic length of a
string.  The value is an element of an enumerated type:
\EnumOptions{LengthFunction}{cFixedLength, cTerminated}
\begin{Description}
\item [cFixedLength] Use the fixed length of \code{e1}.  This option is
for use with arrays with a fixed size at the point of assignment.  For
example, Modula-3 open arrays always have their length fixed before
any assignments may be done to the array.
\item [cTerminated] The array size is determined by an embedded
termination value, which is assumed to be zero.  This value would be
appropriate for C/C++, if they had this type of assignment.
\end{Description}
\Param{e1} Target of the assignment.
\Param{e2} Source value for the assignment.
\Param{padding} Value with which to pad \code{e1} if \code{e1} is
longer than \code{e2}. 
\end{Parameters}

\end{functionality}

%==============================================================================
\Subsection{Numeric Operators}
\oparg{Numeric}{integer, float, or fixed point} To provide support for
C/C++, the generation interface allows numeric operators to be applied
to pointers as well, in which case their bit patterns are
interpreted as integers.

Some routines which operate on numeric types behave differently
depending on its arguments' types.  The generation interface has
several options in how to discriminate between these behaviors.  One
solution is to provide provide a separate version of the routine for
each possible argument type, but this approach creates needless
additional routines.  Another solution is to give appropriate routines
a type parameter, but this approach provides redundant information to
the implementation, since it needs type information to handle
expressions (\eg declare temporaries).  A third solution is to require
implementations to extract type information from an operator's
arguments.  Though this approach requires implementations to extract
information user code already has, it is unlikely to create additional
work and greatly simplifies the interface.  Note, that the interface
requires user code to do complete type conversion; therefore, unless
otherwise stated both arguments must be of the same type.

Different languages have different semantics for overflow, underflow,
and divide-by-zero error conditions.  Both C/C++ and Modula-3 leave
error handling for these conditions implementation dependent, but Ada
requires them to be caught.  The routines representing operators with
possible error conditions accept an extra argument which indicates the
desired semantics.  Its value comes the following enumerated type:
\EnumOptions{ErrorHandling}{cImplementationDefined, cCatchError}
\begin{Description}
\item[cImplementationDefined] This value indicates that the
implementation is free to do as it chooses.
\item[cCatchError] This value indicates that these errors should be caught.
\end{Description}

\subsubsection{Arithmetic Operators}

\interface{void}{ExpPositive}{}{Expression e}{Expression}
\interface{void}{ExpNegative}{}{Expression e}{Expression}
\interface{void}{ExpAbsoluteValue}{}{Expression e}{Expression}
\interface{void}{ExpMinimum}{}{Expression e1, Expression e2}{Expression}
\interface{void}{ExpMaximum}{}{Expression e1, Expression e2}{Expression}

\interface{void}{ExpAddition}{ErrorHandling eh}
	{Expression e1, Expression e2}{Expression}
\interface{void}{ExpSubtraction}{ErrorHandling eh}
	{Expression e1, Expression e2}{Expression}
\interface{void}{ExpMultiplication}{ErrorHandling eh}
	{Expression e1, Expression e2}{Expression}
\interface{void}{ExpDivision}{ErrorHandling eh}
	{Expression e1, Expression e2}{Expression}
\interface{void}{ExpModulus}{ErrorHandling eh}
	{Expression e1, Expression e2}{Expression}
\interface{void}{ExpRemainder}{ErrorHandling eh}
	{Expression e1, Expression e2}{Expression}

\begin{functionality}
The generation interface defines remainder and modulus as follows:
\begin{itemize}
\item Remainder truncates towards zero, and the sign of
its result equals the sign of its right operand.
\item Modulus truncates towards negative infinity, and the sign of its
result equals the sign of its left operand. 
\end{itemize}
C/C++ (\%) and Modula-3's (\key{MOD}) modulus operators actually
implement the interface's remainder function.  The generation
interface's modulus supports Ada semantics.
\end{functionality}

\interface{void}{ExpExponentiation}{ErrorHandling eh}
	{Expression e1, Expression e2}{Expression}
\interface{void}{ExpPreDecrement}{ErrorHandling eh}
	{Expression e}{Expression}
\interface{void}{ExpPreIncrement}{ErrorHandling eh}
	{Expression e}{Expression}
\interface{void}{ExpPostDecrement}{ErrorHandling eh}
	{Expression e}{Expression}
\interface{void}{ExpPostIncrement}{ErrorHandling eh}
	{Expression e}{Expression}

\subsubsection{Relational Operators}
The operands to these operators are of Numeric type, but the resulting
value is of Boolean type.

\interface{void}{ExpEquality}{}{Expression e1, Expression e2}{Expression}
\interface{void}{ExpNotEqual}{}{Expression e1, Expression e2}{Expression}
\interface{void}{ExpGreater}{}{Expression e1, Expression e2}{Expression}
\interface{void}{ExpGreaterEqual}{}{Expression e1, Expression e2}{Expression}
\interface{void}{ExpLess}{}{Expression e1, Expression e2}{Expression}
\interface{void}{ExpLessEqual}{}{Expression e1, Expression e2}{Expression}

\subsubsection{Bitwise Operators}
Bit operators (see also Section~\ref{sec:compoundAssignOp}) interpret their
arguments as a bit pattern.  Hence, no special handling exists for
real types. 

%--
\interface{void}{ExpBitComplement}{}{Expression e}{Expression}
\begin{functionality}
\cxxOp
\end{functionality}
%--
\interface{void}{ExpBitAnd}{}{Expression e1, Expression e2}{Expression}
\begin{functionality}
\cxxOp
\end{functionality}
%--
\interface{void}{ExpBitXor}{}{Expression e1, Expression e2}{Expression}
\begin{functionality}
\cxxOp
\end{functionality}
%--
\interface{void}{ExpBitOr}{}{Expression e1, Expression e2}{Expression}
\begin{functionality}
\cxxOp
\end{functionality}
%--
\interface{void}{ExpBitShiftLeft}{}{Expression e1, Expression e2}{Expression}
\begin{functionality}
\cxxOp
\end{functionality}
%--
\interface{void}{ExpBitShiftRight}{}{Expression e1, Expression e2}{Expression}
\begin{functionality}
\cxxOp
\end{functionality}

%------------------------------------------------------------------------------
\subsubsection{Compound Assignment Operators}\label{sec:compoundAssignOp}
These operators perform two simpler operations in a single step.  The
second operation is a simple assignment.  

%--
\interface{void}{ExpMultiplicationAssignment}{ErrorHandling eh}
	{Expression e1, Expression e2}{Expression}
\begin{functionality}
\cxxOp
\end{functionality}
%--
\interface{void}{ExpDivisionAssignment}{ErrorHandling eh}
	{Expression e1, Expression e2}{Expression}
\begin{functionality}
\cxxOp
\end{functionality}
%--
\interface{void}{ExpRemainderAssignment}{ErrorHandling eh}
	{Expression e1, Expression e2}{Expression}
\begin{functionality}
\cxxOp
\end{functionality}
%--
\interface{void}{ExpAdditionAssignment}{ErrorHandling eh}
	{Expression e1, Expression e2}{Expression}
\begin{functionality}
\cxxOp
\end{functionality}
%--
\interface{void}{ExpSubtractionAssignment}{ErrorHandling eh}
	{Expression e1, Expression e2}{Expression}
\begin{functionality}
\cxxOp
\end{functionality}
%--
\interface{void}{ExpBitShiftRightAssignment}{}
	{Expression e1, Expression e2}{Expression}
\begin{functionality}
\cxxOp
\end{functionality}
%--
\interface{void}{ExpBitShiftLeftAssignment}{}
	{Expression e1, Expression e2}{Expression}
\begin{functionality}
\cxxOp
\end{functionality}
%--
\interface{void}{ExpBitAndAssignment}{}
	{Expression e1, Expression e2}{Expression}
\begin{functionality}
\cxxOp
\end{functionality}
%--
\interface{void}{ExpBitXorAssignment}{}
	{Expression e1, Expression e2}{Expression}
\begin{functionality}
\cxxOp
\end{functionality}
%--
\interface{void}{ExpBitOrAssignment}{}
{Expression e1, Expression e2}{Expression}
\begin{functionality}
\cxxOp
\end{functionality}

%==============================================================================
\subsection{Boolean Operators}\label{sec:logicOp}
\oparg{Boolean}{boolean} C/C++ specify that logical operators can
accept arguments of integral type but still return arguments of
logical type.  Since the generation interface does not permit logical
operators to have integral operands, user code must type convert
integral operands to boolean type.

Nevertheless, boolean values are considered to be an integer subrange,
so that Numeric \method{equality} and \method{inequality} apply to
boolean types.

\interface{void}{ExpTrue}{}{}{Expression}
\interface{void}{ExpFalse}{}{}{Expression}
\begin{functionality}
\method{True} and {false} act as through they are an enumerated type
with false equal to zero and true equal to one.  Hence, the type
conversion routines for enumerated types may be used on these values.
\end{functionality}

\interface{void}{ExpNot}{}{Expression e}{Expression}
\interface{void}{ExpAnd}{}{Expression e1, Expression e2}{Expression}
\interface{void}{ExpOr}{}{Expression e1, Expression e2}{Expression}
\interface{void}{ExpAndConditional}{}{Expression e1, Expression e2}{Expression}
\interface{void}{ExpOrConditional}{}{Expression e1, Expression e2}{Expression}
\begin{functionality}
The \method{*Conditional} forms implement short circuit semantics.
Hence, C and C++ should use these forms.
\end{functionality}

\interface{void}{ExpExpressionIf}{}{Expression e1, Expression e2, Expression e3}
	{Expression}
\begin{functionality}
\cxxOp
\end{functionality}

%==============================================================================
\Subsection{Pointer Operators}
\oparg{Pointer}{pointer}  For equality and inequality, pointers are
treated as bit patterns (\ie a Numeric type).

\interface{void}{ExpAddress}{}{Expression e}{Expression}
\interface{void}{ExpDereference}{}{Expression e}{Expression}
\interface{void}{ExpNil}{}{}{Expression}
\begin{functionality}
This routine generates a \emph{nil} pointer value.  Modula-3 has an
explicit nil value.  C++ specifies that when the integer value zero is
converted to a pointer it becomes the nil pointer value (regardless of
bit representation).
\end{functionality}

%==============================================================================
\Subsection{Aggregate Operators}
In each of the routines with parameters, the first argument is of
aggregate type, and the second argument is of field or routine type.

\interface{void}{ExpThis}{}{}{Expression}
\begin{functionality}
This routine returns a pointer to the current object.  It is only
valid inside of a method.
\end{functionality}

%--
\interface{void}{ExpSelect}{}{Expression e1, Expression e2}{Expression}
\begin{functionality}
\cxxOp
\end{functionality}
%--
\interface{void}{ExpSelectIndirect}{}{Expression e1, Expression e2}{Expression}
\begin{functionality}
\cxxOp
\end{functionality}
%--
\interface{void}{ExpSelectRelative}{}{Expression e1, Expression e2}{Expression}
\begin{functionality}
\cxxOp
\end{functionality}
%--
\interface{void}{ExpSelectRelativeIndirect}{}{Expression e1, Expression e2}
	{Expression}
\begin{functionality}
\cxxOp
\end{functionality}

%==============================================================================
\Subsection{Array Operators}
This section describes the operators available for manipulating
arrays.  Some array operators (\eg \method{assignment} and
\method{deleteArray}) may be found in other sections.

%------------------------------------------------------------------------------
\subsubsection{Subscripting}
Some source languages supply multiple dimension subscripts, while
others, such as C++, require repeated application of a one-dimensional
subscript operator.  The generation interface supports both
approaches.  Multiple dimension subscripts may be composed using the
\method{index} routines.  For single dimensional subscripts, either
of the \method{subscript} routines may be used.

\interface{void}{ExpIndexBegin}{}{}{}
\interface{void}{ExpIndexEnd}{}{\manyPops}{Indicies}
\begin{functionality}
\befunc{ExpIndex}{\node{Expression}}
\end{functionality}
\interface{void}{ExpSubscript}{bool boundsChecking}
	{Expression e, Indicies i}{Expression}
\begin{functionality}
This routine represents a subscript operation.  Some languages require
subscript bound checks while other languages do not.  The
\code{boundsChecking} parameter allows user code to select between
these two choices.
\end{functionality}

\interface{void}{ExpSubscript1d}{bool boundsChecking}
	{Expression e1, Expression e2}{Expression}
\begin{functionality}
This routine is a short-cut for when only one dimension is specified.
Some languages require subscript bound checks while other languages do
not.  The \code{boundsChecking} parameter allows user code to select
between these two choices.
\end{functionality}

%------------------------------------------------------------------------------
\subsubsection{String Array Operators}

%--
\interface{void}{ExpArrayEquality}{}{Expression e1, Expression e2}{Expression}
\begin{functionality}
This routine returns true if the two arrays (\code{e1} and \code{e2})
have the same size and their elements are equal.  
\end{functionality}

%--
\interface{void}{ExpArrayInequality}{}{Expression e1, Expression e2}{Expression}
\begin{functionality}
This routine is the complement of \method{arrayEquality}.  
\end{functionality}

%--
\interface{void}{ExpArrayGreater}{}{Expression e1, Expression e2}{Expression}
\begin{functionality}
\fortranOp
\end{functionality}
%--
\interface{void}{ExpArrayGreaterEqual}{}
	{Expression e1, Expression e2}{Expression}
\begin{functionality}
\fortranOp
\end{functionality}
%--
\interface{void}{ExpArrayLess}{}{Expression e1, Expression e2}{Expression}
\begin{functionality}
\fortranOp
\end{functionality}
%--
\interface{void}{ExpArrayLessEqual}{}
	{Expression e1, Expression e2}{Expression}
\begin{functionality}
\fortranOp
\end{functionality}
%--
\interface{void}{ExpSlice}{}
	{Expression s, Expression b, Expression e}{Expression}
\begin{functionality}
This routine extracts a contiguous section (\ie substring) of a one
dimensional array.  
\begin{Parameters}
\Param{s} The array to be operated on.
\Param{l} The beginning position of the substring.
\Param{e} The ending position of the substring.
\end{Parameters}
\end{functionality}
%--
\interface{void}{ExpRemainingSlice}{}{Expression s, Expression l}{Expression}
\begin{functionality}
This routine extracts the substring of elements from position \code{l}
to the end of the one dimensional array.
\begin{Parameters}
\Param{s} The array to be operated on.
\Param{l} The beginning position of the substring.
\end{Parameters}
\end{functionality}
%--
\interface{void}{ExpConcatenation}{}{Expression s1, Expression s2}{Expression}
\begin{functionality}
\fortranOp
Concatentation is performed into a temporary variable.
\end{functionality}

%==============================================================================
\Subsection{Set Operators}

%--
\interface{void}{ExpSetEquality}{}{Expression se}{Expression}
\begin{functionality}
\modulaOp
\end{functionality}
%--
\interface{void}{ExpUnion}{}{Expression se1, Expression se2}{Expression}
\begin{functionality}
\modulaOp
\end{functionality}
%--
\interface{void}{ExpDifference}{}{Expression se1, Expression se2}{Expression}
\begin{functionality}
\modulaOp
\end{functionality}
%--
\interface{void}{ExpIntersection}{}{Expression se1, Expression se1}{Expression}
\begin{functionality}
\modulaOp
\end{functionality}
%--
\interface{void}{ExpSymmetricDifference}{}{Expression se1, Expression se2}
	{Expression}
\begin{functionality}
\modulaOp
\end{functionality}
%--
\interface{void}{ExpSubset}{}{Expression se1, Expression se2}{Expression}
\begin{functionality}
\modulaOp
\end{functionality}
%--
\interface{void}{ExpSuperset}{}{Expression se1, Expression se2}{Expression}
\begin{functionality}
\modulaOp
\end{functionality}
%--
\interface{void}{ExpElement}{}{Expression e, Expression se}{Expression}
\begin{functionality}
\modulaOp
\end{functionality}

%==============================================================================
\Subsection{Call Operators}
Call operators represent calls to routines.  These calls may
be either function calls or method calls.

\interface{void}{ExpPositionalArgument}{}{Expression e}{Argument}
\interface{void}{ExpNamedArgument}{Identifier name}{Expression e}{Argument}
\begin{functionality}
Arguments may be either positional or named.  C++ uses only positional
arguments, but Modula-3 and Ada use both.  Note that
\method{positionalArgument} really only type converts from
\node{Expression} to \node{Argument}.  
\end{functionality}

%--
\interface{void}{ArgumentsBegin}{}{}{}
%--
\interface{void}{ArgumentsEnd}{}{\manyPops}{ArgumentList}
\begin{functionality}
\befunc{ExpArguments}{\node{Argument}}
\end{functionality}

\interface{void}{ExpCallFunction}{}{Expression function, ArgumentList al}
	{Expression}
\interface{void}{ExpCallMethod}{}{Expression object, Expression
	method, ArgumentList al}{Expression}
\begin{functionality}
This routine generates the representation for a method invocation.
The \code{object} parameter represents the object associate with
the method.  The \code{method} formal is the name of the method
being called (It should be an \node{IdReference} node).

\end{functionality}

%==============================================================================
\Subsection{Heap Operators}
The behavior of heap operators are affected by several compilation
unit attributes (see Section~\ref{sec:compilationUnit}).  The
allocation operators call constructors, if appropriate.  The
delete operators call destructors, if appropriate.  In addition,
implementations of the interface must be aware of whether or not
garbage collection is being used.

\interface{void}{ExpAllocate}{TypeID type}{}{Expression}
\interface{void}{ExpAllocateDefault}{TypeID type}{Expression e}
	{Expression}
\begin{functionality}
This routine is used when the allocation has a default value.
For a class object with an initializer, the expression \code{e} is a
function call to the class' initializer.
\end{functionality}
\interface{void}{ExpAllocatePlacement}{TypeID type}{Expression e,
	Expression p}{Expression}
\begin{functionality}
This routine allows both a default and a placement parameter (for C++).
The argument \code{p} is a placement list parameter.
\end{functionality}
\interface{void}{ExpAllocateOpenArray}{TypeID type}{Indicies i}
	{Expression}
\begin{functionality}
This routine allocates an open array (as found in Modula-3), and sets
the size of the array in each dimension.
\end{functionality}
\interface{void}{ExpAllocateSettingFields}{TypeID type}
	{ArgumentList al}{Expression}
\begin{functionality}
This routine allocates an aggregate entity and then uses the
positional arguments specified in \code{al} to initialize the entity.
\end{functionality}

\interface{void}{ExpDelete}{}{}{Expression e}
\interface{void}{ExpDeleteArray}{}{}{Expression e}
\begin{functionality}
This routine implements the semantics of C++'s delete array operator.
Note that we do not support C++'s archaic number of elements parameter
to the delete array operator.
\end{functionality}

%==============================================================================
\Subsection{Type Operators}

\subsubsection{Type Query Operators}
%--
\interface{void}{ExpBytesizeVariable}{}{Expression variable}{Expression}
\begin{functionality}
This routine implements the Modula-3 \key{bytesize} operator as well as the
C/C++ \key{sizeof} operator.
\end{functionality}
%--
\interface{void}{ExpBytesizeType}{TypeID type}{}{Expression}
\begin{functionality}
This routine implements the Modula-3 \key{bytesize} operator as well as the
C/C++ \key{sizeof} operator.
\end{functionality}
%--
\interface{void}{ExpBitsizeVariable}{}{Expression variable}{Expression}
\begin{functionality}
\modulaOp
\end{functionality}
%--
\interface{void}{ExpBitsizeType}{TypeID type}{}{Expression}
\begin{functionality}
\modulaOp
\end{functionality}
%--
\interface{void}{ExpAdrsizeVariable}{}{Expression variable}{Expression}
\begin{functionality}
\modulaOp
\end{functionality}
%--
\interface{void}{ExpAdrsizeType}{TypeID type}{}{Expression}
\begin{functionality}
\modulaOp
\end{functionality}

%--
\interface{void}{ExpIstype}{TypeID type}{Type t}{Expression}
\begin{functionality}
\modulaOp
\end{functionality}
%--
\interface{void}{ExpNarrow}{TypeID type}{Type t}{Expression}
\begin{functionality}
\modulaOp
\end{functionality}
%--
\interface{void}{ExpTypecode}{}{Expression e}{Expression}
\begin{functionality}
\modulaOp
\end{functionality}
%--
\interface{void}{ExpNumber}{}{Expression e}{Expression}
\begin{functionality}
\modulaOp
\end{functionality}
%--
\interface{void}{ExpFirst}{}{Expression e}{Expression}
\begin{functionality}
\modulaOp
\end{functionality}
%--
\interface{void}{ExpLast}{}{Expression e}{Expression}
\begin{functionality}
\modulaOp
\end{functionality}

%------------------------------------------------------------------------------
\subsubsection{Type Conversion Operators}

Type conversion rules vary substantially between languages.  Hence,
user code is responsible for ensuring that all implicit conversions
are made explicit.  

Some languages provide a fixed set of type conversions, but C++ allows
users to define conversion routines for classes.  Hence in the
generation interface, a type conversion is a triple consisting of the
expression to be converted, the type to which it is to be converted,
and a routine for performing the conversion.  For language defined
conversions, the generation interface provides an enumerated list of
recognized conversions.  The elements of the following enumeration
follow Modula-3 semantics:
\EnumOptions{ConversionRoutines}{cReal, cFloor, cCeiling, cRound,
	cTruncate, cOrdinal, cEnumerationValue, cLoophole, cCast, cComplex}

\interface{void}{ExpTypeConversion}{TypeID type, NameID routine}
	{Expression e}{Expression}
\interface{void}{ExpTypeConversion}{TypeID type, ConversionRoutines cr}
	{Expression e}{Expression}

\todo{
(*----------------------------------------------------------- expressions ---*)

(*  The code to evaluate expressions is generated by calling the
    procedures listed below.  Each procedure corresponds to an
    instruction for a simple stack machine.  Values in the stack
    have a type and a size.  Operations on the stack values are
    also typed.  Type mismatches may cause bad code to be generated
    or an error to be signaled.  Explicit type conversions must be used.

    Integer values on the stack, regardless of how they are loaded,
    are sign-extended to full-width values.  Similarly, word values
    on the stack are always zero-extened to full-width values.
    
    The expression stack must be empty at each label, jump, or call.
    The stack must contain exactly one value prior to a conditional
    or indexed jump.

    All addresses are byte addresses.  There is no boolean type;  boolean
    operators yield [0..1].

    Operations on word values are performed MOD the word size and are
    not checked for overflow.  Operations on integer values may or may not
    cause checked runtime errors depending on the particular code generator.

    The operators are declared below with a definition in terms of
    what they do to the execution stack.  For example,  ceiling(Reel)
    returns the ceiling, an integer, of the top value on the stack,
    a real:  s0.I := CEILING (s0.R).

    Unless otherwise indicated, operators have the same meaning as in
    the Modula-3 report.
*)

(*-------------------------------------------------------------- literals ---*)

load_nil     ();                         (*push; s0.A := NIL*)
load_integer (READONLY i: Target.Int);   (*push; s0.I := i *)
load_float   (READONLY f: Target.Float); (*push; s0.t := f *)

(*------------------------------------------------------------ arithmetic ---*)

(* when any of these operators is passed t=Type.Word, the operator
   does the unsigned comparison or arithmetic, but the operands
   and the result are of type Integer *)
   
eq       (t: ZType);   (* s1.I := (s1.t = s0.t); pop *)
ne       (t: ZType);   (* s1.I := (s1.t # s0.t); pop *)
gt       (t: ZType);   (* s1.I := (s1.t > s0.t); pop *)
ge       (t: ZType);   (* s1.I := (s1.t >= s0.t); pop *)
lt       (t: ZType);   (* s1.I := (s1.t < s0.t); pop *)
le       (t: ZType);   (* s1.I := (s1.t <= s0.t); pop *)
add      (t: AType);   (* s1.t := s1.t + s0.t; pop *)
subtract (t: AType);   (* s1.t := s1.t - s0.t; pop *)
multiply (t: AType);   (* s1.t := s1.t * s0.t; pop *)
divide   (t: RType);   (* s1.t := s1.t / s0.t; pop *)
negate   (t: AType);   (* s0.t := - s0.t *)
abs      (t: AType);   (* s0.t := ABS (s0.t) (noop on Words) *)
max      (t: ZType);   (* s1.t := MAX (s1.t, s0.t); pop *)
min      (t: ZType);   (* s1.t := MIN (s1.t, s0.t); pop *)
round    (t: RType);   (* s0.I := ROUND (s0.t) *)
trunc    (t: RType);   (* s0.I := TRUNC (s0.t) *)
floor    (t: RType);   (* s0.I := FLOOR (s0.t) *)
ceiling  (t: RType);   (* s0.I := CEILING (s0.t) *)
cvt_float(t: AType;  u: RType);     (* s0.u := FLOAT (s0.t, u) *)
div      (t: IType;  a, b: Sign);   (* s1.t := s1.t DIV s0.t;pop*)
mod      (t: IType;  a, b: Sign);   (* s1.t := s1.t MOD s0.t;pop*)

(* In div and mod, "a" is the sign of s1 and "b" is the sign of s0. *)

(*------------------------------------------------------------------ sets ---*)

(* Sets are represented on the stack as addresses. *)

set_union (s: ByteSize);          (* s2.B := s1.B + s0.B; pop(3) *)
set_difference (s: ByteSize);     (* s2.B := s1.B - s0.B; pop(3) *)
set_intersection (s: ByteSize);   (* s2.B := s1.B * s0.B; pop(3) *)
set_sym_difference (s: ByteSize); (* s2.B := s1.B / s0.B; pop(3) *)
set_member (s: ByteSize);         (* s1.I := (s0.I IN s1.B); pop *)
set_eq (s: ByteSize);             (* s1.I := (s1.B = s0.B); pop *)
set_ne (s: ByteSize);             (* s1.I := (s1.B # s0.B); pop *)
set_lt (s: ByteSize);             (* s1.I := (s1.B < s0.B); pop *)
set_le (s: ByteSize);             (* s1.I := (s1.B <= s0.B); pop *)
set_gt (s: ByteSize);             (* s1.I := (s1.B > s0.B); pop *)
set_ge (s: ByteSize);             (* s1.I := (s1.B >= s0.B); pop *)
set_range (s: ByteSize);          (* s2.A[s1.I..s0.I] := 1; pop(3) *)
set_singleton (s: ByteSize);      (* s1.A [s0.I] := 1; pop(2) *)

(*------------------------------------------------- Word.T bit operations ---*)

not ();  (* s0.I := Word.Not (s0.I) *)
and ();  (* s1.I := Word.And (s1.I, s0.I); pop *)
or  ();  (* s1.I := Word.Or  (s1.I, s0.I); pop *)
xor ();  (* s1.I := Word.Xor (s1.I, s0.I); pop *)

shift        ();  (* s1.I := Word.Shift  (s1.I, s0.I); pop *)
shift_left   ();  (* s1.I := Word.Shift  (s1.I, s0.I); pop *)
shift_right  ();  (* s1.I := Word.Shift  (s1.I, -s0.I); pop *)
rotate       ();  (* s1.I := Word.Rotate (s1.I, s0.I); pop *)
rotate_left  ();  (* s1.I := Word.Rotate (s1.I, s0.I); pop *)
rotate_right ();  (* s1.I := Word.Rotate (s1.I, -s0.I); pop *)

extract (sign: BOOLEAN);
(* s2.I := Word.Extract(s2.I, s1.I, s0.I);
   IF sign THEN SignExtend s2; pop(2) *)

extract_n (sign: BOOLEAN;  n: INTEGER);
(* s1.I := Word.Extract(s1.I, s0.I, n);
   IF sign THEN SignExtend s1; pop(1) *)

extract_mn (sign: BOOLEAN;  m, n: INTEGER);
(* s0.I := Word.Extract(s0.I, m, n);
   IF sign THEN SignExtend s0 *)

extract_typecode (equiv: Var);
(* s0.I := typecode of object pointed to by S0 *)

istype (type: TypeUID; t, f: Label);

insert ();
(* s3.I := Word.Insert (s3.I, s2.I, s1.I, s0.I); pop(3) *)

insert_n (n: INTEGER);
(* s2.I := Word.Insert (s2.I, s1.I, s0.I, n); pop(2) *)

insert_mn (m, n: INTEGER);
(* s1.I := Word.Insert (s1.I, s0.I, m, n); pop(1) *)

(*------------------------------------------------ misc. stack/memory ops ---*)

swap (a, b: Type);
(* tmp := s1.a; s1.b := s0.b; s0.a := tmp *)

pop (t: Type);
(* pop(1) discard s0, not its side effects *)

copy_n (t: MType;  overlap: BOOLEAN);
(* copy s0.I units with 't's size and alignment from s1.A to s2.A; pop(3).
   'overlap' is true if the source and destination may partially overlap
   (ie. you need memmove, not just memcpy). *)

copy (n: INTEGER;  t: MType;  overlap: BOOLEAN);
(* copy 'n' units with 't's size and alignment from s0.A to s1.A; pop(2).
   'overlap' is true if the source and destination may partially overlap
   (ie. you need memmove, not just memcpy). *)

zero_n (t: MType);
(* zero s0.I units with 't's size and alignment starting at s1.A; pop(2) *)

zero (n: INTEGER;  t: MType);
(* zero 'n' units with 't's size and alignment starting at s0.A; pop(1) *)

(*----------------------------------------------------------- conversions ---*)

loophole (from, two: ZType);
(* s0.two := LOOPHOLE(s0.from, two) *)

(*---------------------------------------------------- address arithmetic ---*)

add_offset (i: INTEGER);
(* s0.A := s0.A + i bytes *)

index_address (size: INTEGER);
(* s1.A := s1.A + s0.I * size; pop  -- where 'size' is in bytes *)

(*------------------------------------------------------- procedure calls ---*)

(* To generate a direct procedure call:
    
      cg.start_call_direct (proc, level, t);
    
      for each actual parameter i
          <generate value for parameter i>
          cg.pop_param ();  -or-  cg.pop_struct();
        
      cg.call_direct (proc, t);

   or to generate an indirect call:

      cg.start_call_indirect (t);
    
      for each actual parameter i
          <generate value for parameter i>
          cg.pop_param ();  -or-  cg.pop_struct();

      If the target is a nested procedure,
          <evaluate the static link to be used>
          cg.pop_static_link ();

      <evaluate the address of the procedure to call>
      cg.call_indirect (t);
*)

start_call_direct (p: Proc;  lev: INTEGER;  t: Type);
(* begin a procedure call to procedure 'p' at static level 'lev' that
   will return a value of type 't'. *)

call_direct (p: Proc;  t: Type);
(* call the procedure 'p'.  It returns a value of type t. *)

start_call_indirect (t: Type;  cc: CallingConvention);
(* begin an indirect procedure call that will return a value of type 't'. *)

call_indirect (t: Type;  cc: CallingConvention);
(* call the procedure whose address is in s0.A and pop s0.  The
   procedure returns a value of type t. *)

(* Allocate something of type alloc_type *)
start_alloc_call (t: Type; alloc_type: TypeUID);

(* l_to_r = args are passed from left to right *)
method_expr (holder, basetype: TypeUID; method: Name);

call_closure (proc: Var; t: Type);
(* Call closure s0.  Pop s0 *)

pop_param (t: ZType);
(* pop s0.t and make it the "next" parameter in the current call. *)

pop_struct (s: ByteSize;  a: Alignment);
(* pop s0.A, it's a pointer to a structure occupying 's' bytes that's
  'a' byte aligned; pass the structure by value as the "next" parameter
  in the current call. *)

pop_static_link ();
(* pop s0.A for the current indirect procedure call's static link  *)

(*------------------------------------------- procedure and closure types ---*)

load_procedure (p: Proc);
(* push; s0.A := ADDR (p's body) *)

load_static_link (p: Proc);
(* push; s0.A := (static link needed to call p, NIL for top-level procs) *)
}