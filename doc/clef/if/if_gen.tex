% -*- Mode: latex; Mode: auto-fill; -*-

\chapter{Generation Interface}\label{chap:genIf}

This chapter describes the \textit{generation interface}.  The
generation interface provides an Application Programming Interface
which supports the generation of an intermediate representation.  The
interface hides the structure of the intermediate representation from
the code which is calling the interface.  This separation of concerns
allows user code (code which calls the interface) to handle source
language dependent issues, and client code (which uses the generated
representation) to handle language independent or machine dependent
issues.  We use the term \emph{language parser} to refer to code that
handles source language issues, and \emph{user code} to refer to code
that directly calls the generation interface.  For this interface to
generate a particular representation, programmers must supply an
\emph{implementation} which provides the bodies for the
routines defined in the interface.  The code which uses the
resulting representation is called a \emph{client}.

\begin{figure}
  \centering
  %\setlength{\epsfxsize}{\textwidth}
  \epsfbox{scaleFlow_if.eps}
\caption{\label{fig:ifRole} Generation interface's role in Scale
compiler framework.}
\end{figure}

We designed the generation interface with the following goals:
\begin{itemize}
\item Souce language independent
\item Target architecture independent
\item Extensible
\item Support generation of high-level representations 
\item Fully describe original program
\item Prevent user errors where possible
\item Support C, C++, Modula-3, and Fortran~77, and consider
Java, Ada, and parallel Fortran variants
\end{itemize}
Though our design permits the detection of many invalid sequences of
routine calls, it cannot detect all such sequences.  In particular, the
generation interface is not intended to support syntax or semantic
error checking.  Source language issues must be handled before the
generation interface may be used.  The interface provides some
routines that allow the original program to be fully described, even
though the information provided by the routines is not strictly
necessary.  

\begin{table}
\[\begin{array}{cc}
\begin{minipage}[t]{3.0in}
User code:
\begin{itemize}
\item Syntactic and semantic error checking
\item Semantic resolution
\item Enumeration constant assignment
\item Insertion of explicit type conversions
\item Generation of unspecified brands
\end{itemize}
\end{minipage} &
\begin{minipage}[t]{3.0in}
Implementation code:
\begin{itemize}
\item Memory layout
\item Initialization code generation
\end{itemize}
\end{minipage}
\end{array}\]
\caption{\label{tab:assumptions}Division of responsibilities between
user code and implementation code.}
\end{table}

To support our goal of source language independence, we have divided
responsibility between the language parser and the client as shown in 
Figure~\ref{tab:assumptions}.  

The generation interface derives its language independence by
supersetting the capabilities of the languages which we wish to
support.  Hence, no one language is likely to need all the routines
defined in the interface.  However, implementations should implement
all the routines if possible.  If not, they are required to issue
appropriate error messages.

%Our design of the generation interface strives to be high-level and
%source language independent.  The interface definition attempts to
%prevent user (\ie programmers who use the interface) errors, but
%language independence necessarily forces the interface to accept input
%which is nonsense for any language.  In other words, the generation
%interface does not try to support syntax or semantic error checking.
%Ideally, the generation interface would be sufficient to handle any
%programming language, but our design focuses on C, C++, Modula-3, and
%Fortran, with some deference to Ada and Java.  The language used in
%naming interface methods is a half-breed of C++ and Modula-3.


\partitle{\clef\ Implementation}
Our implementation of the generation interface builds the \clef\
representation.  Our implemenation realizes the generation interface
as a C++ class.  The \clef\ class hierarchy does not include the
interface's class hierarchy.  This design point highlights the
separation of the interface from the representation.  Nevertheless,
many of the interface's routines end up being a call to a single
constructor in the \clef\ class hierarchy.

%The programmer should use a separate class hierarchy to define the
%node set for the generated IR.  The programmer must then provide an
%implementation of the generation interface in which the interface
%methods assemble an IR graph.  Many of the interface's methods may be
%simple invocations of a single constructor from the IR node set class
%hierarchy.  Nevertheless, the goal of this interface is to separate
%the code which generates an intermediate representation from the
%details of the represenation.

%The rest of this document lists the method calls in the generation
%interface's API.  Section~\ref{sec:decl} describes how declarations
%are handled.  Section~\ref{sec:stmt} covers statements, and
%section~\ref{sec:op} lists the supported operations.
%Section~\ref{sec:unit} shows how compilation units are
%handled. Finally, interface face options are listed in
%section~\ref{sec:opts}.  

% -*- Mode: latex; Mode: auto-fill; -*-
\Section{Supporting Components}\label{sec:aux}

This section describes the support that the generation interface
requires from an implementation.  

%==============================================================================
\Subsection{Exception Handling}\label{sec:exceptionHandling}
The generation interface requires that the implementation language
provide an exception handling mechanism.  Individual sections and
routines specify exceptions that are unique to them.  The exceptions
listed below may be returned by any interface routine:
\begin{ErrorList}
\ErrorItem{InternalError} The implementation has detected an error
which should not be able to occur.  This indicates a error in the
implementation itself.
\ErrorItem{InvalidParameters} The top most elements on the stack do
not match the types of elements required by the current procedure.
\ErrorItem{InvalidSemantics} The called routine is not valid within
the context in which it is called.  This error is most likely to be
generated by attempting to push an element on the stack (by calling
the current routine) that is not valid between the current begin/end
pair.
\ErrorItem{OutOfMemory} The implementation is not able to allocate
sufficient memory to complete the routine.
\ErrorItem{Unimplemented} The implementation has chosen to not
implement the called routine.
\end{ErrorList}

Implementations may handle exceptions directly or rely on a default
error handler to process all errors.  The interface provides a routine
for setting the default error handler to a user defined procedure.

\interfaceNS{void}{SetErrorHandler}{void (*er)()}
	%set_error_handler (p: ErrorHandler);
	%(*'p' is called to communicate failures (ie creating a stack or module
	%   that's too big) back to the front-end.  Client or implementation
	%   programming errors (bugs) result in crashes. *)
\begin{functionality}
This routine makes the procedure pointed to by \code{er} the default
error handling procedure.  The procedure \code{er} has no arguments
and returns \key{void}.
\end{functionality}


%==============================================================================
\Subsection{The Implementation Stack}\label{sec:implementationStack}

We have modeled the generation interface after the stack IR used in
the SRC Modula-3 compiler.  The two stacks in the stack IR serve very
different purposes.  The first stack, the implementation stack, allows
different implementations of the stack IR to be connected together.
Beginning at the top implementation in the implementation stack, each
called routine first executes its version of the routine, and then
calls the routine of the same name in the next lower implementation.
This process allows user code to generate several layers of
representation (or perform several functions) with a single call.

In the generation interface, the implementation stack has a limited
set of manipulation routines.  Unlike the other routines in the
interface, the routines which operate on the implementation stack do
not recursively execute on lower representations.  Moreover, user code
can only directly call routines in the top most implementation.
Therefore, the implementation stack must be built bottom-up.  User
code is unlikely to want to change the implementation stack
dynamically, so this capability is not supported.

Note for the implementation stack concept to work, the program must be
able to distinguish between different implementations of the same
routine.  Therfore, the generation interface is implemented as an
abstract base class, \code{GenerationInterface}.  Implementations of
this abstract base class inherit the routine names and provide routine
bodies.  The implementation stack contains instances of these
implementation classes (generally one instance per implementation class).

%--
\interfaceNS{void}{AppendImplementation}{GenerationInterface gi}
\begin{functionality}
This routine appends the implementation \code{gi} at the bottom of the
stack. 
\end{functionality}

%==============================================================================
\Subsection{The Operand Stack}\label{sec:stack}

Section~\ref{sec:implementationStack} discusses one use of a stack
data structure in the generation interface.  This section discusses
the other, more important, use of a stack, the operand stack.  User
code uses the generation interface to generate a high-level
representation of a program by calling interface routines.  Each
routine builds a small piece of the final representation.  Between
calls, the interface must store these pieces somewhere.  The
generation interface's solution is to use the operand stack to pass
information between routines.

The generation interface uses a stack in order to allow a cleaner
division between interface and implementation.  The generation
interface seeks to catch as many errors as possible, but this requires
making some assumptions about the represention being generated (\eg
the representation has an expression node).  These assumed nodes (see
Figures~\ref{fig:if_hier_top}--\ref{fig:if_hier_exp}\footnote{These figures
currently show more detail than the interface needs to know.  This
situation will be fixed when Chapter~\ref{chap:nodes} is written.})
are the elements placed on the operand stack, according to the
interface definition.  However, user code is not permitted to directly
manipulate the stack (only implementation code may do so), so
implementations are free to use whatever nodes they want on the stack,
as long as error checking is performed according to the type
information used in this document.  This frees implementations to use
a different class hierarchy than the one shown (or no hierarchy at
all).  Without a stack (or some similar entity), the interface would
have to return intermediate results to the user code, and so the an
implementation would have to create objects with the type name
specified for each routine.

Therefore, the generation interface uses an abstract stack machine as
its execution model.  Before programmers invoke a routine, they must
ensure that the top of the stack contains the necessary arguments.
A \method{StackPop} routine is provided for removing unwanted elements from
the type of the stack.  Adding elements to the top of the stack
requires calling appropriate interface routines.
%--
\interface{void}{StackPop}{}{Node}{}
\begin{functionality}
This routine removes an element from the top of stack.  Since stack
elements are not accessible to user code, no value is returned and the
stack element is simply no longer on the stack.
\end{functionality}


Each interface routine can accept input both from its argument list
and from the operand stack.  Routines can return results through both
their return value and the operand stack.  Most Interface routines
obtain input from the operand stack and push a single return value on
the stack.  Only a few routines use a \key{return} statement to return
a value.  Argument lists are used to pass leaf information to the
interface.  Throughout this chapter, interface routines are shown in
the following form:
\begin{center}
\interface{return-type}{routine-name}{routine-arguments}
	{arguments-on-stack}{type-of-result-on-stack}
\end{center}
In \node{arguments-on-stack}, the rightmost stack element is the top
of stack.  Though this chapter gives names to the arguments, these
names are only for clarity.  The \manyPops\ symbol, which is usually
found in an end routine, indicates that a routine removes an unknown
number of elements from the stack.  All the elements specified in
\node{arguments-on-stack} are replaced by an element of type
\node{type-of-result-on-stack}.  Currently, no routine pushes more
than one value on the operand stack, but this is permissible.

Not all invocations of a routine will need all the routine's
parameters.  To handle the case when the unnecessary parameters are
expressions, the generation interface defines a special value,
\node{NoExpression}, to represent no value.  

\partitle{Unbounded Input} 
When a construct allows a dynamic number of repeated items (\eg fields
in a record, statements in a block, arguments in a list), the
interface uses a begin/end pair to bracket the repeated elements.  For
example, calls to \method{FormalsBegin} and \method{FormalsEnd}
surround calls to \method{DeclFormal}.  The begin routine does not
have a visible effect on the operand stack, but the end routine
removes all the intervening stack elements, forms a list of these
elements, and pushes a new stack element on the top of the stack which
represents this list.  Unless otherwise stated, user code may build an
empty list by calling the end routine when the stack is identical to
when the begin routine was called.

The generation interface design requires unbounded lists to be
completely pushed on the operand stack before any of the list elements
can be removed.  For some lists (\eg statements in a procedure or
statements in a file), this approach appears wasteful of space.
However, implementations are most likely pushing the actual program
representation on the stack, and this representation must exist
regardless of whether or not it is on the stack.  So, no extra space
may be required.  If space is still a concern, then implementations
may choose to devise a scheme by which appropriate lists are
incrementally compressed on the operand stack, as long as the
semantics of the begin/end pair are preserved.

\partitle{Implementation suggestion} 
Implementations may choose to use an auxillary stack, the list stack,
for begin/end pairs.  Each time a begin occurs, an element is pushed
on the list stack that records the top of stack when the begin routine
was called.  Therefore, when the corresponding end routine is called,
all the elements of the operand stack down to the recorded location
may be removed and formed into a list.

%--
\begin{figure}
\caption{\label{fig:if_hier_top}
  Top level of class hierarchy model used in design of generation interface.}
  \begin{center}
  %\setlength{\epsfxsize}{\textwidth}
  \epsfbox{if/if_hier_node.eps}
  \end{center}
\end{figure}
%--
\begin{figure}
\centering
\caption{\label{fig:if_hier_type}
  Class hierarchy for Types.}
  %\setlength{\epsfxsize}{\textwidth}
  \epsfbox{if/if_hier_type.eps}
\end{figure}
%--
\begin{figure}
\centering
\caption{\label{fig:if_hier_decl}
  Class hierarchy  for Declarations.}
%  \setlength{\epsfxsize}{\textwidth}
  \epsfbox{if/if_hier_decl.eps}
\end{figure}
%--
\begin{figure}
\centering
\caption{\label{fig:if_hier_stmt}
  Class hierarchy for Statements.}
  %\setlength{\epsfxsize}{\textwidth}
  \epsfbox{if/if_hier_stmt.eps}
\end{figure}
%--
\begin{figure}
\centering
\caption{\label{fig:if_hier_exp}
  Class hierarchy for Expressions.}
  \setlength{\epsfxsize}{\textwidth}
  \epsfbox{if/if_hier_exp.eps}
\end{figure}
%--
%\begin{figure}
%\centering
%\caption{\label{fig:if_hier_unit}
%  Class hierarchy for compilation units.}
%  %\setlength{\epsfxsize}{\textwidth}
%  \epsfbox{if/if_hier_unit.eps}
%\end{figure}

\partitle{Implementation Suggestion}
We strongly recommend creating two global variables to aid in
debugging.  These variables are intended to help catch errors with
begin/end pairs.  One variable, \code{listType}, holds an
indication of the type of an arbitrary element popped off the stack by
an end routine.  The other variable, \code{listLength}, holds a count
of how many elements were popped.  These variables' values are valid
until the next end routine.  The initial values of these variables are
\code{NotAType} and zero, respectively.

%==============================================================================
\Subsection{The Symbol and Type Tables}\label{sec:tables}

The generation interface assumes that implementations maintain two
tables related to declarations: a symbol table and a type table.  A
symbol table maintains correspondences between programmer introduced
names (and some language defined names) and their value.  A type table
maintains a representation of each type's structure.  Since the
language parser is responsible for semantic resolution, it already
has all the information contained in these two tables, and indeed most
likely has its own version of these tables.  However, to maintain its
independence from an particular language parser, the generation
interface cannot simply accept the parser's tables directly.  Instead,
these two tables must be built by an implemenation by repeated calls
to type constructor (see Section~\ref{sec:type}) and declaration
routines (see Section~\ref{sec:decl}).  

Since the interface does not know anything about the user code's
tables, any reference to an identifier or type must refer to entries
in the interface implementation's tables.  Hence, user code needs some
way of referring to an entry in an interface implementation's tables.
Appropriate interface routines pass back an entry identifier.
Whenever the implementation makes an entry in one of its tables, it
passes back either a \ret{NameID} which refers to an entry in the
symbol table, or a \ret{TypeID} which refers to an entry in the type
table.  User code must retain a mapping of \ret{NameID}s and
\ret{TypeID}s to entries in its own tables.  Whenever an identifier
is used, the user code can than translate from the entry in its symbol
table to the \ret{NameID} or \ret{TypeID} which identifies the entry
in the implementation's tables.  Implementations should provide
\code{NoName} and \code{NoType} to represent that the corresponding
item does not exist.

%Constructing a symbol table requires some knowledge of the source
%language.  Since the structure of the front end's symbol table is
%unknown to the interface, the generation interface (actually
%implementations of the interface) must build its own symbol table.  As
%long as the correspondences between declarations and uses of
%identifiers are maintained, building a symbol table should be easy
%because the interface does not need to worry about the details of how
%scoping works in the source language.  Unfortunately, passing
%correspondences requires extra work on the part of the interface user
%(\ie calling software).  The user must maintain a correspondence
%between an identifer's \key{Declaration} assigned by the generation
%interface and the user's own symbol table index.  A dictionary data
%structure should be sufficient to mainatin this correspondence.

%The generation interface supports the construction of a simple symbol
%table by implementations of the interface.  Users may open scopes with 
%%the \method{BeginScope} and \method{BeginCompound} methods.  
%a variety of methods.  Compound
%entries in a symbol table (such as a type with subcomponents) are
%built via \method{desc*} methods, and then entered into the symbol
%table with a \method{decl*} method.  The generation interface allows
%users to seed the symbol table with language defined identifiers by
%using the \method{decl\_builtin\_*} methods.  

%------------------------------------------------------------------------------
\subsubsection{The Symbol Table}\label{sec:symbolTable}

This section describes the function of the symbol table in more detail
and lists the routines used to control and access the symbol table.  A
symbol is a name.  Most symbols are created by the programmer (\eg
variable names and procedure names).  Some names are introduced by the
compiler either for compiler generated entities such as compiler
generated labels or as compiler generated names for anonymous entities.
User code also generates a few names with calles to the primitive type
declaration routines\footnote{Primitive types would not have to be
named in order to representat a program.  However, it may be useful in
debugging and quite easy to provide.}.

Many program elements have names: variables, procedures, exceptions
(in some source languages), etc.  The generation interface refers to
any nameable program element as an \emph{entity}.  Every entity in a
program must have a corresponding \emph{entry} in the symbol table,
which maps the name to its value (where a value can have several
parts).  The value of an entity depends on what kind of entity it is.
A value of a type entity is a type; the value of a procedure entity is
its signature and body.  In order for user code to refer to a particular
entity (or equivalently, entry), interface routines return a
\ret{NameID} whenever a new entry is created in the symbol table.
Each implementation is free to define \ret{NameID}'s structure, but it
is conceptually an index into the implementation's symbol table.

After an entity is declared, user code always uses the \ret{NameID} to
refer to the entity.  By not using identifiers, the generation
interface avoids having to understand the source language's scoping
and name conflict resolution rules.  In other words, the symbol table
does not limit the number of times an identifier may appear in a
single scope.  On the other hand, the generation interface cannot
provide name lookup.

%- - - - - - - - - - - - - - - - - - - - - - - - - - - - - - - - - - - - - - -
\paragraph{Specifying Scopes}
Client code needs to understand scoping, in order to transform code
and generate object code.  Unfortunately, different source languages
may have slightly different scoping rules.  The generation interface
handles this issue by requiring user code to explicitly mark where a
scope begins and ends by using the \method{ScopeBegin} and
\method{ScopeEnd} routines.  These routines do not visibly affect
the contents of the stack.  

Since user code is responsible for semantically resolving the program
before using the generation interface to build an new intermediate
representation, the interface can use a generic block structured
symbol table.  Declarations may appear anywhere within a scope, and
the declared entity is useable anywhere from the point of declaration
to the end of the enclosing scope.  In addition, entities in enclosing
scopes are accessible from an inner scope (note that \emph{visibility}
is a language parser issue).

Note that namespaces are essentially named scopes (see
Section~\ref{sec:namespace}).
%--
\interface{void}{ScopeBegin}{}{}{}
%--
\interface{void}{ScopeEnd}{}{}{}

%- - - - - - - - - - - - - - - - - - - - - - - - - - - - - - - - - - - - - - -
\paragraph{Entering Entities into the Symbol Table}
The generation interface specifies that all programmer defined
entities (\eg types, labels, procedures) are stored in the symbol
table.  Many entities have names, but some may not.  Entities are
entered into the symbol table either by \emph{declarations} or
\emph{specifications}.  A declaration defines an entities; whereas, a
specification provides partial information about an entity.

%- - - - - - - - - - - - - - - - - - - - - - - - - - - - - - - - - - - - - - -
\paragraph{Entity References}
In addition to the notion of declarations, the generation interface
maintains the idea of an entity \emph{reference}.  An entity reference
represents an entity independently of its name.  Uses of an entity in
a program are represented by its reference node, rather than its
declaration.  In most cases, a reference is indistinguishable from a
declaration, but not always.  For example consider two variables which
are declared to be aliases of each other.  They have separate
declarations but a single reference.  Because of the similarity
between references and declarations, their class hierarchies are
identical and only the declaration hierarchy is shown (see
Figure~\ref{fig:if_hier_decl}).  \ret{NameID} maps into declarations,
but a routine is provided for mapping from \ret{NameID} to references.

%- - - - - - - - - - - - - - - - - - - - - - - - - - - - - - - - - - - - - - -
\paragraph{Accessing the Symbol Table}

To avoid having to know the details of how a particular language
manages its symbol table, the generation interface relies on the
requirement that user code semantically resolves identifiers before
using the generation interface.  Hence, user code knows to which
declaration each use of an identifier refers (\ie user code does its
own name lookup).  User code passes this knowledge to the generation
interface by using a \ret{NameID}.  It is a simple type which uniquely
identifies a declaration.  Every declaration routine passes a
\ret{NameID} back.  User code is responsible for maintaining a mapping
between entries in its symbol table and NameIDs.

User code is only permitted to lookup entities which it has already
entered into the symbol table.  Hence, few errors should occur.
Nevertheless, implementations ought to check the validity of
\node{NameID}s. 
% and if possible, verify that the type of the retrieved
% entity matches the expected result type of the routine.

%--
\interface{void}{LookupDecl}{NameID name}{}{Declaration}
%--
\interface{void}{LookupRef}{NameID name}{}{Reference}
%--
\interface{void}{LookupType}{TypeID type}{}{Type}
%--
\interface{void}{LookupTypeDecl}{NameID type}{}{Type}

\subsubsection{Type Table}

The most complicated values in the symbol table are type descriptions.
In fact, type information is maintained in a separate table.  The type
table has its own index, \ret{TypeID}.  If the only use of the type
table were to describe types, the design of the table would be
straightforward.  However, the generation interface must also capture
sufficient semantic information for client code to perform type
equivalence checks.  Determining type equivalence is difficult because
each source language defines its own type equivalence rules.  The
generation interface's approach is to include as much information as
possible in the type table, and then define a general routine which
traverses the type table using a subset of that information to compare
two types.  

\partitle{Adding Information}
Besides the normal information required by type constructors, the
generation interface places two other kinds of information in the type
table.  The first is type names, which are useful for type
equivalence.  To simplify the specification of type names across the
interface, the generation interface takes the name from the type
declaration routine and inserts it into the type table.  The name is
added via a \node{nameBrand} node, which is distinct from a
\node{UserBrand} node.  Note that a \node{NameBrand} node may
include multiple names to handle types with multiple names.  The other
kind of information is language defined type matching rules.  These
rules are captured at the compilation unit level and include such
information as whether or not the source language uses field order to
distinguish record types.

\partitle{Type Equivalence} A general routine is used to perform all
type equivalence checks.  This routine accepts as input an indication
of which of the possible features that distinguish types should be
used for the current comparison.

\partitle{Multiple Type Names}
Source languages generally allow programmers to declare types, which
is naming a type.  Source languages generally allow multiple names to
be associated with the same type.  For example in the following C
code, both \code{gew} and \code{GEW} are names for the same type:
\begin{center}
\begin{minipage}{2.0in}\code{
typedef struct gew \{
\ \ \ int a;
\} GEW;
}\end{minipage}
\end{center}

The generation interface does not guarantee that two equivalent types
will have the same \ret{TypeID}.  Hence, user code cannot use
\ret{TypeID}s to determine type equivalence, but user code should
already have its own way of finding type equivalence.  Implementations
may wish to provide a way for client code to compute type equivalence.

\partitle{Decorated Types}
One reason for not guaranteeing a unique \ret{TypeID} for each type is
because types can be decorated with attributes.
Section~\ref{sec:typeAttributes} discusses the different type
attributes in more detail.  Type attributes are associated with an
individual type node, and any type node may be annotated though the
generation interface does not guarantee that all attributes make
sense.  Attributes do not affect the type of an object, but do
represent information about the object that must be maintained.
Hence, the \ret{TypeID} carries the attribute information as well as
the type.

\partitle{Recursive Types}
One of the hardest aspects of describing types is describing recursive
types.  To handle this situation, the generation interface provides
\method{Incomplete*} routines for types which may refer to instances
of themselves (\ie aggregate types).  Within the definition of the
aggregate type, recursive references may refer to the incomplete type.
After the definition of the aggregate type is complete, user code
should call the \method{CompleteType} routine to establish the
relationship between the incomplete and complete types.

\partitle{Branding}
The generation interface supports a branded type which is used to
distinguish two types which are otherwise type equivalent.  Branding
is a way to introduce a name into the type table, and thereby obtain
name equivalence in a structural equivalence framework.  Branding is
in fact the only way of introducing a name into the type 
table.\footnote{A name is specified when primitive types are declared,
and these are recorded in the type table.  However, these names are
source language defined (\eg int) rather than user or compiler defined,
and they are not used to discriminate types.  Indeed, their only
purpose is debugging support.}

\partitle{Separate Compilation}
Separate compilation complicates type matching because not all type
information is available for each compilation.  For separate
compilation, C/C++ uses name equivalence between files.  Hence, two
types are considered equivalent if they have the same name, regardless
of their actual type declaration.  This unfortuante rule works fine if
the two files never exchange data of this type, otherwise the
programmer has made a non-detectable error.  To support this
mis-feature of C/C++, only the brand (not the type structure) is used
to determine the equivalence of branded types.

%==============================================================================
\Subsection{Identifiers}\label{sec:identifiers}

All practical programming languages use identifiers, user selected
strings, to represent program entities such as variables, types, and
labels.  The generation interface does not use identifiers to
reference these entities, since many programming languages allow the
same identifier to have different meanings in different contexts.
However, the generation interface does permit user code to pass
identifiers through to client code.

The generation interface represents identifiers as strings.  A null
string is interpreted as representing that the programmer did not
specify an identifier.  Among other possible uses, a null string can
be used to indicate anonymous declarations.

%==============================================================================
\Subsection{Stable Storage}\label{sec:storage}

Once the user has finished generating a representation with the
generation interface, he or she is likely to want to save the
representation to some form of stable storage.  The generation
interface provides the following routine for performing this function.
The interpretation of \args{key} is implementation dependent, and the
interface does not provide a mechanism for reading back in the
representation.  This functionality should be provided by other code
which manipulates the generated representation.  

\interface{bool}{WriteRepresentation}{String key}{}{}
\begin{functionality}
This function writes the representation out to an implementation
defined stable storage.  The \args{key} argument supplies an
identifier, most likely a file name, for the written representation.
When multiple implementations are stacked, calling this routine will
write out all the different representations.  Therefore, we suggest
that each implementation modify the key somehow to uniquely identify
its output (\eg a file name extension).  The return value indicates
success or failure, with the value from multiple implementations
combined with a logical or function.
\end{functionality}


% -*- Mode: latex; Mode: auto-fill; -*-
\Section{Source Language Specific Comments}\label{sec:lang}

This section records suggestions of how to translate specific source
languages using the generation interface.  The generation interface
directly reflects the syntax and semantics of C/C++ and Modula-3, so
these languages are likely to have fewer comments.  

%==============================================================================
\Subsection{Modula-3}

\begin{itemize}
\item inc/dec 

Modula-3's \key{inc} and \key{dec} statements are not
directly supported in the generation interface.  User code should
represent them as \method{eval}ed expressions.

\item assignment 

In Modula-3 assignment is a statement, but the
generation interface provides only an expression form of assignment.
User code should represent an assignment statement as an
\method{eval}ed assignment expression.

\item Refany and Address

Refany is represented as an unattributed pointer to void type.
Address is represented as a pointer to void type with the untraced
attribute.

\item Safe modules

The language parser and user code is responsible for checking whether
or not modules are safe.  This information is not passed through the
generation interface.

\end{itemize}

%==============================================================================
\Subsection{Fortran~77}

This section consists of a list of issues and our recommendations for
handling them.

\begin{itemize}
\item Generic functions

Fortran~77 allows a small amount of function overloading by permitting
library writers to create generic functions.  The identification of
generic functions is considered a front end issue, so the generation
interface only provides specific function names.

\item Declarations

Declarations in Fortran~77 are distributed across several statements
rather than collected together as expected by the generation interface.  
Basically, user code is responsible for collecting the distributed
information together and making the appropriate generation interface
call.  The following lists some of the declaration statements that
must be transformed:
\begin{itemize}
\item Implicit
\end{itemize}

\item Strings

User code should transform Fortran~77 strings into character arrays.

\item Intrinsic Functions

Some intrinsic functions are mapped to primitive operators, while
others will have to be implemented in a library.  See
Tables~\ref{tab:ops1}--\ref{tab:ops2}.  
[<
\item At least once \key{do} loops.

Some older version of Fortran have \emph{at least once} semantics for
their \key{do} loops.  The generation interface does not directly
provide these semantics, so user code will have to build equivalent
code from the available routines.

\end{itemize}

%==============================================================================
\Subsection{C++}

\begin{itemize}
\item Declarations in conditions and for loop initialization

C++ allows programmers to declare variables in conditions (\ie the
test in loops and in conditional statements).  The value of the
variable comes from its initialization.
\end{itemize}


% -*- Mode: latex; Mode: auto-fill; -*-
\Section{Specifying Types}\label{sec:type}

This section describes the routines which user code can use to pass
type information to clients (see Table~\ref{tab:typeConstructors}.
The routines for declaring type names can be found in
Section~\ref{sec:decl}.  Type information is stored in the type table
(see Section~\ref{sec:tables}).

\begin{table}[b]
\centering
\begin{tabular}{|l||c|c|c|}\hline
\multicolumn{4}{|c|}{Type Constructor Correspondences} \\\hline
\textbf{Method Name} & \textbf{C++} & \textbf{Modula-3} & \textbf{Fortran~77}
\\\hline

TypePrimitiveCharacter		& 
	char & 
	char & 
	character \\\hline
TypePrimitiveInteger		
	& int	
	& integer
	& integer \\\hline
TypePrimitiveFixedPoint		
	& \na	
	& \na	
	& \na \\\hline
TypePrimitiveReal			
	& float/double	
	& real/longreal/extended
	& real \\\hline
TypePrimitiveVoid			
	& void	
	& null
	& \na \\\hline
TypePrimitiveBoolean		
	& bool	
	& boolean
	& logical \\\hline
TypeArrayFixed
	& []	
	& array-of	
	& (), $*$ \\\hline
TypeArrayOpen			
	& \na	
	& array-of	
	& \na \\\hline
TypeArrayUnconstrained		
	& \na	
	& \na	
	& \na \\\hline
TypeEnum begin/end
	& enum	
	& \{\}	
	& \na  \\\hline
TypeRecord begin/end
	& struct
	& record	
	& \na \\\hline
TypeIncompleteRecord		
	& struct
	& \na
	& \na \\\hline
TypeUnion begin/end
	& union	
	& \na
	& equivalence \\\hline
TypeIncompleteUnion		
	& union	
	& \na	
	& \na \\\hline
TypeClass begin/end
	& class	
	& object
	& \na \\\hline
TypeIncompleteClass		
	& class	
	& \na
	& \na \\\hline
TypePointer				
	& $*$	
	& \na	
	& \na \\\hline
TypeIndirect			
	& \&
	& ref
	& \na \\\hline
TypeOffset
	& ::* 
	& \na
	& \na \\\hline
TypeProcedureType			
	& ()	
	& procedure	
	& \na \\\hline
TypeSet				
	& \na	
	& set-of
	& \na \\\hline
TypeRange				
	& \na	
	& [..]	
	& \na \\\hline
TypeBrand				
	& \na	
	& branded
	& \na \\\hline
TypePacked				
	& :	
	& bits-for
	& \na \\\hline

\end{tabular}
\caption{\label{tab:typeConstructors}Correspondence between generation 
interface routine names and language type constructors.}
\end{table}
\clearpage


Different languages employ slightly different versions of derived
types and subtypes.  A derived type has all the associated operators
of the parent type, but is not type compatible with other types
derived from the same parent.  A subtype is much like a derived type,
except that a subtype may have additional fields and operators to
those of the parent type.  The generation interface provides type
attributes and branding for representing derived types.  Subtyping is
only available for classes.

%==============================================================================
\Subsection{Primitive Type Constructors}

This section describes routines for declaring primitive language
types.  In term of type representation, these routines specify the
leaves of type DAGs.  Different languages use different names for the
same primitive type, and some languages do not completely specify what
their primitive types mean.  The generation interface requires user
code to indicate what name and format their source code needs.  The
type name is only useful for debugging.\footnote{The name does not
distinguish primitive types.  Hence, implementations should be able to
handle having the same primitive type with two different names.
Without a name, a debugger could not identify (to the user) the
primitive types.}  These routines may be called multiple times with
different values to create different primitive types (\eg C's
\key{short}, \key{int}, and \key{long} are all integers).  

The generation interface offers two mechanisms for defining primitive
types.  User code may either specify the type with constraints on the
number of bits used in the representation, or the user code may use
purely symbolic names for the size of the types.  Symbolic names will
be mapped to the natural size supported by the target architecture.
For symbolic types, their size is specified with a symbolic value from
the following enumerated list:
\EnumOptions{SymbolicSize}{cVeryShort, cShort, cNormal, cLong, cVeryLong}

%------------------------------------------------------------------------------
\subsubsection{Sized Primitive Types}

%--
\interface{TypeID}{TypePrimitiveCharacter}
	{Identifier name, CharacterFormat cf}{}{Type} 
\begin{functionality}
Ideally, the language should not specify a particular character
format.  However, many C programs rely on the ANSI character set, and
Java prescribes the use of Unicode.  Hence, user code may use this
routine to specify the format (and implicitly the size) of the
character set.  If the language does not require a particular format
it may specify \code{Any}.  The generation interface assumes a default
of \code{Any}.

The generation interface provides the following enumeration:
\EnumOptions{CharacterFormat}{cAny, cAnsi, cEbcdic, cUnicode}
\end{functionality}

%--
\interface{TypeID}{TypePrimitiveInteger}{Identifier name, int minBitSize,
	IntegerRepresentation rep}{}{Type}
\begin{functionality}
This routine creates a primitive integer type in the symbol table.
\begin{Parameters}
\Param{name} The name of the type.
\Param{minBitSize} The minimum number of bits required to represent 
this type.
\Param{rep} Machine representation for this integer type.  Its value
comes from the following enumerated type.
\EnumOptions{IntegerRepresentation}{cUnsigned, cTwosComplement}
\end{Parameters}
\end{functionality}

%--
\interface{TypeID}{TypePrimitiveSymbolicInteger}{Identifier name, 
	SymbolicSize size, IntegerRepresentation rep}{}{Type}
\begin{functionality}
This routine creates a primitive integer type in the symbol table.
\begin{Parameters}
\Param{name} The name of the type.
\Param{size} The symbolic size of the integer type.
Our expectation is that for current machines, \code{cVeryShort} equals
eight bits, \code{cShort} equals sixteen bits, \code{cNormal} equals 32
bits, \code{cLong} equals 64 bits, and \code{cVeryLong} equals 128 bits.
\Param{rep} Machine representation for this integer type.  Its value
comes from the following enumerated type.
\EnumOptions{IntegerRepresentation}{cUnsigned, cTwosComplement}
\end{Parameters}
\end{functionality}

%--
\interface{TypeID}{TypePrimitiveFixedPoint}
	{Identifier name, int minBitSize, int minScaleBits}{}{Type}
\begin{functionality}
This routine creates a primitive fixed point type in the symbol table.
\begin{Parameters}
\Param{name} The name of the type.
\Param{minBitSize} The minimum number of bits required to represent 
this type.
\Param{minScaleBits} Minimum number of bits by which to scale the
representation.  This parameter may be negative.  \code{minBitSize}
includes \code{minScaleBits}.
\end{Parameters}
\end{functionality}

%--
\interface{TypeID}{TypePrimitiveReal}{Identifier name, int minBitSize}
	{}{Type}
\begin{functionality}
\begin{Parameters}
\Param{name} The name of the type.
\Param{minBitSize} The minimum number of bits required to represent this
type.
\end{Parameters}
Note that the format of real numbers is considered a back end issue.  
\end{functionality}

%--
\interface{TypeID}{TypePrimitiveSymbolicReal}
	{Identifier name, SymbolicSize size}{}{Type}
\begin{functionality}
\begin{Parameters}
\Param{name} The name of the type.
\Param{size} The symbolic size of the real type.  
\end{Parameters}
Note that the format of real numbers is considered a back end issue.  
\end{functionality}

%--
\interface{TypeID}{TypePrimitiveComplex}
	{Identifier name, int realMinBitSize, int imaginaryMinBitSize}
	{}{Type}
\begin{functionality}
\begin{Parameters}
\Param{name} The name of the type.
\Param{realMinBitSize} The minimum number of bits required to
represent the real part of this type.
\Param{imaginaryMinBitSize} The minimum number of bits required to
represent the imaginary part of this type.
\end{Parameters}
Note that the format of real numbers is considered a back end issue.  
\end{functionality}

%--
\interface{TypeID}{TypePrimitiveVoid}{Identifier name}{}{Type}
\begin{functionality}
\node{Void} is the null type.  
\end{functionality}

%--
\interface{TypeID}{TypePrimitiveBoolean}{Identifier name}{}{Type}
\begin{functionality}
Boolean values are defined in Section~\ref{sec:logicOp}.
\end{functionality}

%An important open issue is how will basic data types be handled in a
%language independent fashion.  Each interface user could declare all
%of their source languages primitive types; however, how will interface
%implementation recognize identical types with different names?  The
%interface could require implementations to seed their symbol tables
%with basic types, but how will users get a handle to these type
%declarations? 

	%(*The debugging information for a type is identified by a gloablly unique
	% 32-bit id generated by the front-end.  The following methods generate
	%   the symbol table entries needed to describe Modula-3 types to the
	%   debugger. *)
		
	%declare_typename (t: TypeUID;  n: Name);
	%(* associate the name 'n' with type 't' *)

	%declare_builtin (t: TypeUID; n: TEXT);
	%(* Declare a builtin type with uid t and name n *)

	%declare_builtin_object (t: TypeUID; n: TEXT; super: TypeUID);
	%(* Declare a builtin object type with uid t and name n *)

%==============================================================================
\Subsection{Type constructors}

Languages such as C and C++ which provide weak type naming
capabilities encourage programmers to use type constructors when
describing the type of non-type entities (\eg variables) instead of
type names.  Implementations may choose to handle such circumstances
by internally creating an anonymous type for the entity.  

%------------------------------------------------------------------------------
\subsubsection{Array Type Constructors}
These routines support the construction of arrays.  In some languages,
arrays carry knowledge of their length, and in other languages (\eg C
and C++) they do not.  Though knowing their length does impact data
layout, it does not affect type equivalence.  Hence, we mark each
array as to whether or not it must carry a length:
%--
\interface{TypeID}{TypeArrayFixed}{bool lengthField}
	{RangeTypes indexType, Type elementType}{Type}
	%declare_array (t, index, elt: TypeUID;  s: BitSize);
\begin{functionality}
This routine constructs a fixed length array.
\begin{Parameters}
\Param{indexType} Indicates the type of the index expression 
(a list of ranges - for a single dimension array it will be a list
containing one range).
\Param{elementType} Indicates the type of the array elements.
\end{Parameters}
\end{functionality}

%--
\interface{TypeID}{TypeArrayUnconstrained}{bool lengthField}
	{RangeTypes indexType, Type elementType}{Type}
\begin{functionality}
This routine represents an unconstrained array type, as found in Ada.
An unconstrained array is an incomplete type, because it lacks an
index range.  Therefore, the type of \node{indexType} should not
include range information (\ie the bounds should be \node{NoBounds}).  
\end{functionality}

%--
\interface{TypeID}{TypeArrayOpen}{bool lengthField}{Type elementType}
	{Type}
	%declare_open_array (t, elt: TypeUID;  s: BitSize);
	%(* s describes the dope vector *)
\begin{functionality}
This routine constructs an open array.  The size of an open array is
determined at runtime but cannot change once set.  An open array acts
like an unconstrained array, where the only unknown entity is the
maximum index value.  
\end{functionality}

%------------------------------------------------------------------------------
\subsubsection{Enumeration Type Constructors}
The generation interface assumes that the front end has assigned a
value to each enumeration item.  This approach frees 
implemenations of the interface from concern about source language
specific vagaries in enumeration element assigment.

%--
\interface{NameID}{DeclEnumElement}{Identifier id}{Expression e}
	{EnumElementDecl}

%--
\interface{void}{TypeEnumBegin}{}{}{}
%--
\interface{TypeID}{TypeEnumEnd}{Identifier id}{\manyPops}{Type}
\begin{functionality}
\befunc{TypeEnum}{\node{EnumElementDecl}}
\end{functionality}

%--
%\interface{void}{Enum}{}{EnumElements elts}{Type}
%	%declare_enum (t: TypeUID; n_elts: INTEGER;  s: BitSize);
%	%declare_enum_elt (n: Name);

%------------------------------------------------------------------------------
\subsubsection{Incomplete Type Constructors}

The generation interface requires that all entities be defined before
being used.  This restriction keeps the interface free of language
specific name resolution rules.  Unfortunately for recursive types,
this restriction implies that user code needs a mechanism for
specifying a type before definining it completely.  The generation
interface uses a mechanism similar to forward declarations found in
many languages.  

The generation interface allows the creation of \emph{incomplete
types}.  An incomplete type does not carry any additional type
information, and therefore may be used anywhere a type is valid.  User
code should build recursive type structures with recursive references
pointing to the incomplete type.  After the recursive type is
completely built, user code should use \method{CompleteType} to
allow implementations of the interface to patch up its representation
to reflect the true recursive structure.  Incomplete types are only to
aid in conveying the actual type structure through the interface.
Hence, user code must associate complete type with each incomplete
type.

For languages such as Modula-3 which do not require forward
declarations for mutually recursive types, user code must generate
incomplete types.  

Incomplete types can be created at any time.  Hence, user code can
begin building an aggregate structure and generate an incomplete type
only if the aggregate type turns out to be recursive.  However, the
user code must be careful to clean up the \node{Type} node on the stack.

%--
\interface{TypeID}{TypeIncompleteType}{Identifier name}{}{Type}
%--
\interface{TypeID}{TypeCompleteType}{TypeID tic, TypeID tc}{}{Type}
\begin{functionality}
This routine completes the declaration of an incomplete type.  It
informs the implementation that incomplete type \code{tic} is really
the completed type \code{tc}.  The \node{Type} node left on the stack
corresponds to \code{tc}.
\begin{Parameters}
\Param{tic} \ret{TypeID} identifying incomplete type.
\Param{tc} \ret{TypeID} identifying complete type.
\end{Parameters}
\end{functionality}

%------------------------------------------------------------------------------
\subsubsection{Aggregate Type Constructors}

%--
\interface{void}{TypeRecordBegin}{}{}{}
%--
\interface{TypeID}{TypeRecordEnd}{}{\manyPops}{Type}
\begin{functionality}
\befunc{TypeRecord}{\node{FieldDecl}}
In C++, a class struct is represented as a \node{ClassType}.
\end{functionality}

%--
\interface{void}{TypeUnionBegin}{}{}{}
%--
\interface{TypeID}{TypeUnionEnd}{}{\manyPops}{Type}
\begin{functionality}
\befunc{TypeUnion}{\node{FieldDecl}}
In C++, a class union is represented as a \node{ClassType}	
\end{functionality}

%--
\interface{void}{Superclass}{NameID class, AccessSpecifier as}{}{SuperClass}
\begin{functionality}
\begin{Parameters}
\Param{class} Indicates the super class.
\Param{as} Indicates the access specifier (as defined by C++).  For
other languages, a suitable value for \code{as} should be selected.
The value for \code{as} comes from the following:
\EnumOptions{AccessSpecifier}{cPrivateAccess, cProtectedAccess, cPublicAccess}
\end{Parameters}
\end{functionality}
%--
\interface{void}{SuperclassBegin}{}{}{}
%--
\interface{void}{SuperclassEnd}{}{\manyPops}{SuperClasses}
\begin{functionality}
\befunc{Superclass}{\node{SuperClass}}
\end{functionality}
%--
\interface{void}{SingleSuperclass}{NameID class, AccessSpecifier as}
	{}{SuperClasses}
\begin{functionality}
This routine is a shortcut for cases when only a single super class
exists (as with single inheritance).
\end{functionality}

%--
\interface{void}{TypeClassBegin}{}{}{}
%--
\interface{TypeID}{TypeClassEnd}{}{SuperClasses, \manyPops}{Type}
	%declare_object (t, super: TypeUID;  brand: TEXT;
	%        traced: BOOLEAN;  n_fields, n_methods, n_overrides: INTEGER;
	%        field_size: BitSize);
	%(* brand=NIL ==> t is unbranded *)
\begin{functionality}
The generation interface uses its own variation of C++ and Modula-3
terminology when discussing objects.  An \emph{object} is an instance
of a \emph{class} datatype.  The components of a class are called
\emph{members}, which may be either data \emph{fields}, \emph{routines},
conversion functions, constructors, and destructors.

\befunc{TypeClass}{\node{Declaration}}
\end{functionality}

%------------------------------------------------------------------------------
\subsubsection{Pointer Type Constructors}
The routines in this section allow the construction of pointer types.

%--
\interface{TypeID}{TypePointer}{}{Type t}{Type}
	%declare_pointer (t, target: TypeUID;  brand: TEXT;  traced: BOOLEAN);
	%(* brand=NIL ==> t is unbranded *)
\begin{functionality}
This routine constructs a pointer type which must be explicitly dereferenced.
\end{functionality}

%--
\interface{TypeID}{TypeIndirect}{}{Type t}{Type}
	%declare_indirect (t, target: TypeUID);
	%(*an automatically dereferenced pointer! (WITH variables, 
	%	VAR formals, ...) *)
\begin{functionality}
This routine describes automatically dereferenced pointers.  In
Modula-3 these pointers are used in implementing \key{with} aliases,
\key{var} formals, etc.  
\end{functionality}
	
%--
\interface{TypeID}{TypeOffset}{NameID aggregate}{Type t}{Type}
\begin{functionality}
This routine represents a pointer-to-member as found in C++.  A
pointer-to-member is an offset to a member of \args{aggregate}.  
\end{functionality}
	
%------------------------------------------------------------------------------
\subsubsection{Procedure Type Constructors}
A method is a procedure who is a member.

\partitle{Parameters}
%--
\interface{void}{FormalsBegin}{}{}{}
%--
\interface{void}{FormalsEnd}{}{\manyPops}{Formals}
\begin{functionality}
\befunc{Formals}{\node{FormalDecl}}
\end{functionality}

\partitle{Exceptions}
%--
\interface{void}{RaiseException}{NameID exception}{}{Raise}
	%declare_raises (n: Name);
\begin{functionality}
This routine represents a Modula-3 style exception, when specifying a
procedure's throw list.
\end{functionality}
%--
\interface{void}{RaiseType}{TypeID type}{}{Raise}
\begin{functionality}
This routine represents a C++ style exception, when specifying a
procedure's throw list.
\end{functionality}
%--
\interface{void}{RaisesAny}{}{}{Raise}
\begin{functionality}
This routine indicates that the associated procedure can raise any
exception.  By default, C++ functions may raise any exception;
Modula-3 procedures must explicitly indicate that they can generate
any exception.
\end{functionality}
%--
\interface{void}{RaisesNone}{}{}{Raises}
\begin{functionality}
This routine indicates that the associated procedure cannot raise any
exceptions.  By default, Modula-3 procedures cannot raise any exceptions;
C++ procedures must explicitly indicate that they cannot raise
exceptions.  Using this routine is equivalent to specifying
\method{RaisesBegin}/\method{RaisesEnd} without any \node{Raise}
nodes in between.
\end{functionality}
%--
\interface{void}{RaisesBegin}{}{}{}
%--
\interface{void}{RaisesEnd}{}{\manyPops}{Raises}
\begin{functionality}
\befunc{Raises}{\node{Raise}}
\end{functionality}

%--
\interface{void}{Signature}{}{Formals f, Type ret, Raises r}{Signature}
\begin{functionality}	
This function defines a routine signature.  All signatures must
indicate what exceptions it can raise.  For those languages which do
not support exceptions, they should simply call \method{RaiseNone}.
\end{functionality}	

%object_info (t: TypeUID);
%(* s1 = method offset, s0 = field offset*)

%--
\interface{TypeID}{TypeProcedure}{}{Signature s}{Type}
	%declare_proctype (t: TypeUID;  n_formals: INTEGER;
	%                  result: TypeUID;  n_raises: INTEGER;
	%                  cc: CallingConvention);
	%(* n_raises < 0 => RAISES ANY *)

	%declare_method (obj: TypeUID; n: Name;  signature: TypeUID; 
	%		offset:INTEGER);

%------------------------------------------------------------------------------
\subsubsection{Set Type Constructors}
The routines in this section permit the construction of set types.
%--
\interface{TypeID}{TypeSet}{}{Type type}{Type}
	%declare_set (t, domain: TypeUID;  s: BitSize);

%------------------------------------------------------------------------------
\subsubsection{Range Type Constructors}\label{sec:range}
\par
\partitle{Bounds} 

Bounds specify a minimum and maximum value.  Bounds are primarily used
for specifying the minimum and maximum for range types.  Support for
defining bounds varies between languages, so the minimum and maximum
values are arbitrary expressions.  We also use ranges to represent
array indicies.  For languages (such as Ada) that support true
multi-dimensional arrays, ranges may be chained together since a range
type contains a single bound.  However, the common case for C/C++ and
Modula-3 is to use a single range.

%--
\interface{void}{Bound}{}{Expression min, Expression max}{Bound}
\begin{functionality}
This routine creates a single bounds which can be chained together
with a begin/end pair.
\end{functionality}
%--
\interface{void}{BoundsBegin}{}{}{}
%--
\interface{void}{BoundsEnd}{}{\manyPops}{Bounds}
\begin{functionality}
\befunc{Bounds}{\node{Bound}}
\end{functionality}
%--
\interface{void}{Bounds}{Expression min, Expression max}{}{Bounds}
\begin{functionality}
This routine is a short cut for specifying a single dimensional bounds
specification.
\end{functionality}
%--
\interface{void}{Nobounds}{}{}{Bound}
\begin{functionality}
This routine pushes onto the stack a special \node{Bounds} that
connotes that no bounds have been specified.  This value 
may be used to construct arrays without bounds.
\end{functionality}
%--
%\interface{bool}{Isnobounds}{}{Bounds}{Bounds}
%\begin{functionality}
%This routine allows user code to test if a bounds is undefined.  The
%\node{Bounds} on the top of the stack is unmodified.
%\end{functionality}

%--
\interface{TypeID}{TypeRange}{}{Type basetype, Bound b}{Type}
	%declare_subrange (t,domain: TypeUID; READONLY min,max: Target.Int; 
	%		s: BitSize);
\begin{functionality}
This routine creates a range type with a single bound.
\end{functionality}
%--
\interface{void}{RangeBegin}{}{}{}
%--
\interface{void}{RangeEnd}{}{\manyPops}{RangeTypes}
\begin{functionality}
\befunc{RangeType}{\node{RangeType}}
\end{functionality}

%------------------------------------------------------------------------------
\subsubsection{Branded Type Constructors}
Branded type constructors are unique in that though they do indeed
introduce a new type, they do not change the structure of the type.
Branding is useful in languages with structural equivalence (\eg
Modula-3).
%--
\interface{TypeID}{TypeBrand}{}{Type t, Expression b}{Type}
\begin{functionality}
This routine creates a new type which is structurally identical to
type \code{t}.  
\end{functionality}

%------------------------------------------------------------------------------
\subsubsection{Packed Type Constructors}
The generation interface treats packed types a unique types.  For
C/C++ this distinction between types would not be necessary.  However,
other languages such as Modula-3 clearly distinguish between a base
type and its packed version.  This distinction implies that user code
must insert explicit conversions between the base type and packed
type.

%--
\interface{TypeID}{TypePacked}{}{Type basetype, Expression bitSize}{Type}
\begin{functionality}
This routine builds a new type which compresses type \code{basetype}
into \code{bitSize} bits.  
\end{functionality}

%------------------------------------------------------------------------------
\subsubsection{Alias Type Constructors}

C++ provides a type constructor for defining alias.  C++ calls these
aliases \emph{references}.  An alias does not have memory space of its
own, but rather refers to another entity's memory.  

I've considered representing references as an alias declaration which
looks better, but return types can be references.  I've also
considered treating references as indirect pointers.  However, this
approach might hurt alias analysis.

%--
\interface{TypeID}{TypeAlias}{}{Type type}{Type}

%==============================================================================
\Subsection{Type Attributes}\label{sec:typeAttributes}
This section describes how to associate an attribute with a type.
Attributes are non-type information that is associated with a type.
Attributes are associated with a type rather than a \node{TypeDecl}
because they can be associated with just a part of a type declaration.
Attributes should generally not affect type equivalence (except
perhaps in some minor cases).  The notion of associating an attribute
with a type has strong implications for the implemenation.  Ideally,
equivalent types would share the same physical representation.
However, types that differ only in attributes should be equivalent,
yet cannot share attributes.

The possible attributes are provided by the following enumeration:
\EnumOptions{TypeAttribute}{ cTraced, cUntraced, cConstantType, cVolatile,
			  cOrdered, cUnordered }
where:
\begin{Description}
%\item [brand] Supports Modula-3's branding of data types.
\item [cTraced] Supports Modula-3's traced data types.  
\item [cUntraced] Supports Modula-3's untraced data types.  This is the
default attribute for all types.
%\item [constructor] Indicates that a method is an object constructor.
%\item [destructor] Indicates that a method is an object destructor.
%\item [conversion] Marks a routine as a user-defined type conversion routine.
%\item [abstract] Marks a method as undefined for the current class,
%and therefore requiring definition in derived classes.
%\item [packed] Indicates that a type should use less than the
%natural number of bits.  This attribute has a value: the size of the
%bit field.

%This attribute should be sufficient to handle both Modula-3 packed types
%and C++ bit fields.  

%\item [auto] Marks a data value as being locally allocated (\ie on the
%stack).  Local allocation is the default form of allocation and so
%seldom needs to be explicitly specified.
%\item [register] Recommends that a value be assigned to a register.
%If the value cannot be assigned to a register, it should be allocated
%on the stack.
%\item [static] Indicates that an entity is assigned to permanently
%allocated space.
%\item [globalLinkage] Indicates that an entity has globally visible
%in the program.  This attribute implements the common case of C/C++'s
%\key{extern} construct.
%\item [foreignLinkage] Indicates that an entity is visible to a
%different source language.  This attribute eliminates C++'s
%overloading of the \key{extern} keyword.
%\item [fileLinkage] Indicates than an entity is visible only within
%the current file.
%\item [public] Specifies that an identifier is visible outside of its
%namespace.  
%\item [protected] Specifies than an identifier is visible only within
%its namespace.  This attribute may only be used for class members, in
%which case it denotes the semantics of C++'s \key{protected} construct.
%\item [private] Specifies that an identifier is visible only within
%its namespace.
\item [cConstantType] Indicates that instances of this type have a constant
value.  By default, instances of a type are mutable.

One could reasonably argue that the immutability of a value is not a
property (or attribute) of a type.  However, C++'s typedef construct
permits immutability to be included with the type.
%\item [mutable] Indicates than instances of this type are \emph{not} 
%constant.  This value is the default for all types and should
%therefore never actually be specified.
%The primary purpose of this attribute is to indicate that a component
%of a composite constant entity is not constant.
%\item [friend] Represents C++'s \key{friend} construct.
%\item [nonvirtual] Indicates that a method may not be overloaded.
%\item [virtual] Indicates that a method may be overloaded.
\item [cVolatile] Marks a value which may be changed by something which
a compiler cannot detect.
\item [cOrdered] For types with substructures, this attribute indicates
if the source language requires the data layout to preserve the order
of substructures (declaration order is assumed).  
\item [cUnordered] For types with substructures, this attribute indicates
if the type may be layed out in an arbitrary order.
\end{Description}

%--
\interface{void}{SetTypeAttribute}{TypeAttributes ta}{Type t}{Type}
\begin{functionality}
This routine associates an attribute with a type.
\end{functionality}
%--
%\interface{void}{SetTypeAttributeWithValue}{}
%	{Type t, TypeAttributes ta, Expression e}{Type}
%\begin{functionality}
%This routine associates an attribute with a type when the attribute
%has a value.
%\begin{Parameters}
%\Param{t} The type to which the attribute is being associated.
%\Param{ta} The attribute which is an element of the
%\code{TypeAttribute} enumeration.
%\Param{e} The attributes value.
%\end{Parameters}
%\end{functionality}



% -*- Mode: latex; Mode: auto-fill; -*-
\Section{Generating Declarations}\label{sec:decl}

This section covers how declarations are handled by the generation
interface.  Declarations are one way of binding attributes (or values)
to an identifier (or name).  The generation interface divides a
declaration into three parts: a name, a value, and miscellaneous
attributes.  Anonymous declarations exist, but they behave as though
the system creates a unique identifier for the declaration.  The
generation interface supports the declaration of types, values (\eg
variables and record fields), labels, procedures, and exceptions (for
Modula-3).  The interface also supports a variety of additional
attributes.

%(*-------------------------------------------------------- runtime hooks ---*)

\todo{
set_runtime_hook (n: Name;  v: Var;  o: ByteOffset);
(* declares 'n' as a runtime symbol at location 'ADR(v)+o' *)

get_runtime_hook (n: Name;  VAR v: Var; VAR o: ByteOffset);
(* returns the location of the runtime symbol 'n' *)
}

%==============================================================================
\Subsection{Declaring Types}

%--
\interface{NameID}{DeclType}{Identifier name}{Type t}{TypeDecl}
\begin{functionality}
  This routine creates a new type by making an entry for it in the
symbol table.  

\emph{Implementation Note:} To support our type matching scheme, this
routine should perform an extra step.  It should first brand \args{t}
with a \node{NameBrandType}, and then use the updated type for the
declaration.  
\end{functionality}

\interface{NameID}{NameType}{Identifier name, TypeID type}{}{TypeDecl}
\begin{functionality}
  This routine creates an alias for a type which may already have an
entry in the symbol table.  This routine implements the semantics of
C++'s \key{typedef} construct.  Notice the use of \args{TypeID} rather
than \node{Type}.

\emph{Implementation Note:} If the type already has a name brand, then
this name should simply be added to the existing name brand.
Otherwise, a name brand should be created for it.
\end{functionality}

%%--
%\interface{NameID}{DeclAnonType}{}{Type td}{TypeDecl}
%--

\subsubsection{Forward Declarations}
%--
\interface{NameID}{ForwardDeclProcedure}{Identifier
name}{Signature}{RoutineDecl}
\begin{functionality}
In Java terms, this routine has been \textbf{deprecated}.  Use
\texttt{SpcfyProcedure} instead.
\end{functionality}

%------------------------------------------------------------------------------
\subsubsection{Opaque Types}

Opaque types are type names for which the full type structure is
unknown.  Instead, we know that the type is a subtype of the specified
type.  

Modula-3 distinguishes between an opaque declaration and a partial
revelation.  This distinction is not important to the generation
interface, so a single routine exists.  From the generation
interface's point of view, an opaque declaration creates a name for a
partially specified type.

Fully revealed opaque types are really just a type declaration, and
should be represented as such.

Note that implementations will most likely wish to create a special
type constructor that marks an opaque type as representing the super
type.

\interface{NameID}{DeclOpaque}{Identifier name, TypeID superType}{}
	{TypeDecl}
\begin{functionality}
This routine handles the declaration of opaque types and partial
revelations.  An opaque type is a Modula-3 construct whereby a type is
only specified as being a subtype of another type.  Additional
information may be provided by partial revelations, which merely
provide a more specific super type.  

We considered associating partial revelations with the original opaque
declaration.  However, the interface does not need to do error
checking, so this associationg is not necessary.  

A complete revelation provides the full type definition, and is
therefore represented as a normal type declaration.

\emph{Implementation suggestion:} An opaque declaration associates a
name with a type.  However, the name is not an instance of the type,
but rather an instance of a subtype of that type.  Hence, we recommend
creating a \emph{subtype} type constructor.  The symbol table entry
for an opaque type may then point to a subtype node in the type table.
Note that the generation interface does not provide any routines to
directly generate a subtype type node.
\end{functionality}

%==============================================================================
\Subsection{Declaring Values}

Many source languages optionally allow programmers to give a default
value (default value for formal parameters and initial value for
variables) to a named values in its declaration.  Since the generation
interface requires an expression corresponding to the default value to
be on the stack, user code should use the \method{noExpression}
routine to push a special expression onto the stack.

\emph{Implementation note}: Implementations may choose to represent
initial values for variables and formals as a simple assignment
following the declaration.  Since the generation interface is intended 
to support a variety of intermediate representations, it provides as
much support for the original syntax as possible and allows
implementations to choose how best to represent initial values.
Moreover, an initialization of a C++ reference type is different from
an assignment to it.

%--
\interface{NameID}{DeclVariable}{Identifier name}
	{Type t, Expression initialValue}{ValueDecl}
\begin{functionality}
This routine declares a variable.
\begin{Parameters}
\Param{name} The variable's name.
\Param{t} The variable's type.
\Param{initialValue} The variable's initial value.
\end{Parameters}
\end{functionality}
%--
%--
\interface{NameID}{DeclTemporary}{Identifier name}
	{Type t, Expression initialValue}{ValueDecl}
\begin{functionality}
This routine declares a temporary variable.  We provide this
routine to distinguish programmer declared variables and compiler
created variables.
\begin{Parameters}
\Param{name} The variable's name.
\Param{t} The variable's type.
\Param{initialValue} The variable's initial value.
\end{Parameters}
\end{functionality}
%--
\interface{NameID}{DeclFormal}{Identifier name, Mode m}
	{Type t, Expression defaultValue}{ValueDecl}
\begin{functionality}
This document uses the term \emph{argument} to refer to an actual
parameter and \emph{formal} to refer to a formal parameter.  

\begin{Parameters}
\Param{name} The formal's name.
\Param{m} The formal's parameter mode.
\Param{t} The formal's type.
\Param{defaultValue} The formal's default value.
\end{Parameters}

Implementations of the generation interface must provide an enumerated
list of parameter mode options, such as the following:
\EnumOptions{Mode}{cInValue, cValue, cReference, cValueResult, cResult}
where:
\begin{Description}
\item [cInValue] Pass-by-value but the formal's value may not be
altered.  This mode is used for Ada.
\item [cValue] Pass-by-value but the formal's value may be altered.
This mode is the default for Modula-3 and the only one for C/C++.
\item [cReference] Formal parameter is an alias for the argument.
Therefore, an update to the formal parameter is a direct update the
argument (actual parameter) as well.  This mode is the only parameter
mode for Fortran and represents the \key{var} mode for Modula-3.
\item [cReadOnly] Pass-by-reference, but the value cannot be updated.
This mode supports Modula-3's \key{readonly} mode, and is efficient
for large data structures.
\item [cValueResult] Represents copy-in and copy-out semantics.  This
mode is used for Ada's \key{inout} mode. 
\item [cInOut] Represents Ada's \key{inout} keyword, which allows
implementations to choose between \textbf{Reference} and
\textbf{ValueResult}.  
\item [cResult] Represents Ada's \key{out} mode, which allows
implementations to choose between \textbf{Reference} and
\textbf{ValueResult}. 
\end{Description}

Handling C++'s parameter modes is awkward.  In C++, programmers may
declare a formal to be constant (\ie pass-by-inValue) or to be a
reference (\ie pass-by-reference).  Unfortunately, this declaration
may be buried in the type declaration used in the definition of the
formal.  These unfortunate declaration semantics requires searching
through the symbol table to determine the parameter passing mode, and
would require that clients recognize the duplication and ignore it.
Hence, the generation interface dictates that for C++, all calls to
\method{declFormal} should use pass-by-value.
\end{functionality}

%--
%\interface{NameID}{DeclAnonymousFormal}{Mode m}
%	{Type t, Expression defaultValue}{ValueDecl}
%\begin{functionality}
%This routine creates an anonymous formal.  Anonymous formals may be
%used to declare forward declarations and procedure types.  See
%\method{declFormal} for a more detailed description
%\end{functionality}
%--
\interface{void}{UnknownFormals}{}{}{ValueDecl}
\begin{functionality}
This routine indicates that the remainder of a routine's signature has
an unknown number of formal parameters.  Hence, this routine
implements C++'s ``...'' construct.
\end{functionality}

%--
\interface{void}{AlternateReturnFormal}{}{}{ValueDecl}
\begin{functionality}
Fortran provides an unusual version of a return statement called an
alternate return.  An alternate return returns to one of possibly
many line numbers, where the possible return locations are provided in
special procedure parameters.  The Fortran syntax for these special
parameters is an asterik ($*$), which this routine represents.

Alternate return parameters may occur anywhere in the parameter list;
however, the alternate return construct views these special parameters
as forming a simple linear list from 1 to n.
\end{functionality}

%--
\interface{NameID}{DeclField}{Identifier name}
	{Type t, Expression initialValue}{FieldDecl}
\begin{functionality}
This routine declares a field, which is a component of an aggregate
data structure.

\begin{Parameters}
\Param{name} The field's name.
\Param{t} The field's type.
\Param{initialValue} The field's initial value.
\end{Parameters}
\end{functionality}
%--
%\interface{NameID}{DeclAnonymousField}{}
%	{Type t, Expression initialValue}{FieldDecl}
%\begin{functionality}
%This routine declares an anonymous field, as in C bit fields.  See the
%\method{declField} routine for more information.

%\begin{Parameters}
%\Param{t} The field's type.
%\Param{initialValue} The field's initial value.
%\end{Parameters}
%\end{functionality}

%--
\interface{NameID}{DeclConstant}{Identifier name}
	{Type td, Expression value}{ConstantDecl}
\begin{functionality}
This routine declares a constant value.  A constant value does not
require storage to be allocated, which is different from other
declarations with a constant attribute.

\begin{Parameters}
\Param{name} The constant's name.
\Param{t} The contant's type.
\Param{value} The contant's value.
\end{Parameters}
\end{functionality}

%(*------------------------------------------------ variable declarations ---*)

%(* Clients must declare a variable before generating any statements or
%   expressions that refer to it;  declarations of global variables and
%   temps can be intermixed with generation of statements and expressions.

%   In the declarations that follow:

%|    n: Name            is the name of the variable.  If it's M3ID.NoID, the
%|                         the back-end is free to choose its own unique name.
%|    s: ByteSize        is the size in bytes of the declared variable
%|    a: Alignment       is the minimum required alignment of the variable
%|    t: Type            is the machine reprentation type of the variable
%|    m3t: TypeUID       is the UID of the Modula-3 type of the variable,
%|                       zero is used to represent "void" or "no type"
%|    in_memory: BOOLEAN specifies whether the variable must have an address
%|    exported: BOOLEAN  specifies whether the variable must be visible in
%|                         other compilation units
%|    init: BOOLEAN      indicates whether an explicit static initialization
%|                         immediately follows this declaration.
%|    up_level: BOOLEAN  specifies whether the variable is accessed from
%|                         nested procedures.
%|    f: Frequency       is the front-end estimate of how frequently the
%|                         variable is accessed.

%*)


%import_global (n: Name;  s: ByteSize;  a: Alignment;  t: Type;
%               m3t: TypeUID): Var;
%(* imports the specified global variable. *)

%declare_segment (n: Name;  m3t: TypeUID): Var;
%bind_segment (seg: Var;  s: ByteSize;  a: Alignment;  t: Type;
%              exported, init: BOOLEAN);
%(* Together declare_segment and bind_segment accomplish what
%   declare_global does, but declare_segment gives the front-end a
%   handle on the variable before its size, type, or initial values
%   are known.  Every declared segment must be bound exactly once. *)

%segment_types (seg: Var);

%declare_global (n: Name;  s: ByteSize;  a: Alignment;  t: Type;
%                m3t: TypeUID;  exported, init: BOOLEAN): Var;
%(* declares a global variable. *)

%declare_constant (n: Name;  s: ByteSize;  a: Alignment;  t: Type;
%              m3t: TypeUID;  exported, init: BOOLEAN): Var;
%(* declares a read-only global variable *)
 
%declare_local (n: Name;  s: ByteSize;  a: Alignment;  t: Type;
%               m3t: TypeUID;  in_memory, up_level: BOOLEAN;
%               f: Frequency): Var;
%(* declares a local variable.  Local variables must be declared in the
%   procedure that contains them.  The lifetime of a local variable extends
%   from the beginning to end of the closest enclosing begin_block/end_block. *)

%declare_param (n: Name;  s: ByteSize;  a: Alignment;  t: Type;
%               m3t: TypeUID;  in_memory, up_level: BOOLEAN;
%               f: Frequency): Var;
%(* declares a formal parameter.  Formals are declared in their lexical
%   order immediately following the 'declare_procedure' or
%   'import_procedure' that contains them.  *)

%declare_field (n: Name;  o: BitOffset;  s: BitSize;  t: TypeUID);

%declare_temp (s: ByteSize;  a: Alignment;  t: Type;
%              in_memory: BOOLEAN): Var;
%(* declares an anonymous local variable.  Temps are declared
%   and freed between their containing procedure's begin_procedure and
%   end_procedure calls.  Temps are never referenced by nested procedures. *)

%free_temp (v: Var);
%(* releases the space occupied by temp 'v' so that it may be reused by
%   other new temporaries. *)

\todo{
(*---------------------------------------- static variable initialization ---*)

(* Global variables may be initialized only once.  All of their init_*
   calls must be bracketed by begin_init and end_init.  Within a begin/end
   pair, init_* calls must be made in ascending offset order.  Begin/end
   pairs may not be nested.  Any space in a global variable that's not
   explicitly initialized is zeroed.  *)

begin_init (v: Var);
end_init (v: Var);
(* must precede and follow any init calls *)

init_int (o: ByteOffset;  READONLY value: Target.Int;  t: Type);
(* initializes the integer static variable at 'ADR(v)+o' with
   the low order bits of 'value' which is of integer type 't'. *)

init_proc (o: ByteOffset;  value: Proc);
(* initializes the static variable at 'ADR(v)+o' with the address
   of procedure 'value'. *)

init_label (o: ByteOffset;  value: Label);
(* initializes the static variable at 'ADR(v)+o' with the address
   of the label 'value'.  *)

init_var (o: ByteOffset;  value: Var;  bias: ByteOffset);
(* initializes the static variable at 'ADR(v)+o' with the address
   of 'value+bias'.  *)

init_offset (o: ByteOffset;  var: Var);
(* initializes the static variable at 'ADR(v)+o' with the integer
   frame offset of the local variable 'var' relative to the frame
   pointers returned at runtime in RTStack.Frames *)

init_chars (o: ByteOffset;  value: TEXT);
(* initializes the static variable at 'ADR(v)+o' with the characters
   of 'value' *)

init_float (o: ByteOffset;  READONLY f: Target.Float);
(* initializes the static variable at 'ADR(v)+o' with the
   floating point value 'f' *)
}

%==============================================================================
\Subsection{Declaring Procedures and Methods}

Unlike most other declared entities, procedures are generally not fully
declared at the point of their declaration.  The generation interface
refers to a function declaration without a body (\ie a C/C++ prototype
or a Modula-3 procedure interface) as a \emph{specification}, and a
function declaration with a body as a \emph{declaration}.  

Hence, the compiler or linker must ultimately resolve all function
specifications back to their corresponding declarations.  For Modula-3
this process is straightforward because of the well-structured
import/export mechanism.  For C/C++, this process is a little more
work.  

C++ (\cite{ellis:90}, p. 138) and Modula-3 (\cite{Cardelli:95}, p. 27)
define slightly different type matching rules for procedures.  Since
the generation interface currently supports only statically typed
languages and then only when semantic resolution has already been
done, we ignore these differences.

%\partitle{Overrides}
%%--
%\interface{void}{Override}{}{Identifier name, ConstantExpression proc}{Member}
%	%declare_override (obj: TypeUID; n: Name; offset: BitOffset);
%%--
%\interface{void}{OverrideBegin}{}{}{}
%%--
%\interface{void}{OverrideEnd}{}{\manyPops}{Overrides}
%\begin{functionality}
%\befunc{override}{\node{Overrides}}
%\end{functionality}

\partitle{Initializers}
C++ allows a list of initializers to be provided along with
constructors.  These routines handle the representations of
initializers.

%--
\interface{void}{Initializer}{NameID initializedEntity}
	{ArgumentList al}{Initializer}
%--
\interface{void}{InitializersBegin}{}{}{}
\interface{void}{InitializersEnd}{}{\manyPops}{Initializers}
\begin{functionality}
\befunc{Initializers}{ArgumentList}
\end{functionality}
%--
\interface{void}{NoInitializers}{}{}{Initializers}
\begin{functionality}
This routine creates an empty initializer list.
\end{functionality}

\partitle{Routine Specification}
A specification indicates the name and type of a procedure but does
not provide its body.  

%--
\interface{NameID}{SpcfyProcedure}{Identifier name}{Signature s}{RoutineDecl}
%--
\interface{NameID}{SpcfyNestedProcedure}
	{Identifier name, int level, NameID parentRoutine}
	{Signature s}{RoutineDecl}

%--
\interface{NameID}{SpcfyMethod}{Identifier name}{Signature s}{RoutineDecl}

%--
\interface{NameID}{SpcfyFriend}{Identifier name}{Signature s}{RoutineDecl}

%--
\interface{NameID}{SpcfyTypeConversion}{Identifier name}{Signature s}
	{RoutineDecl}

%--
\interface{NameID}{SpcfyConstructor}{Identifier name}
	{Signature s, Initializers i}{RoutineDecl}

%--
\interface{NameID}{SpcfyDestructor}{Identifier name}{Signature s}
	{RoutineDecl}


\partitle{Routine Declaration}
A routine declaration includes the routine's body.  

%--
%\interface{NameID}{DeclProcedure}{Identifier name}{Signature s, 
%	Statement body}{RoutineDecl}
%--
\interface{NameID}{DeclProcedure}
	{Identifier name, int level, NameID parentRoutine} 
	{Signature s, Statement body}{RoutineDecl}

%--
\interface{NameID}{DeclMethod}{Identifier name, TypeID class}
	{Signature s, Statement body}{RoutineDecl}
\begin{functionality}
The class type is specified if the method is defined outside the
class.  Otherwise, use \node{NoType}.
\end{functionality}

%--
\interface{NameID}{DeclMethodReference}{Identifier name, NameID proc}
	{Signature s}{RoutineDecl}
\begin{functionality}
This routine supports Modula-3 style method definition in which a
method is defined by a top level procedure.
\end{functionality}

%--
\interface{NameID}{DeclOverride}{Identifier name, NameID proc}
	{Signature s}{RoutineDecl}
\begin{functionality}
A Modula-3 override looks a lot like a method definition; however, its
affects the method table differently.
\end{functionality}

%--
\interface{NameID}{DeclTypeConversion}{Identifier name, TypeID class}
	{Signature s, Statement body}{RoutineDecl}
\begin{functionality}
The class type is specified if the method is defined outside the
class.  Otherwise, use \node{NoType}.
\end{functionality}

%--
\interface{NameID}{DeclConstructor}{Identifier name, TypeID class}
	{Signature s, Initializers, Statement body}{RoutineDecl}
\begin{functionality}
The class type is specified if the method is defined outside the
class.  Otherwise, use \node{NoType}.
\end{functionality}

%--
\interface{NameID}{DeclDestructor}{Identifier name, TypeID class}
	{Signature s, Statement body}{RoutineDecl}
\begin{functionality}
The class type is specified if the method is defined outside the
class.  Otherwise, use \node{NoType}.
\end{functionality}

%--
\interface{NameID}{DeclEntry}{Identifier name, NameID enclosingProcedure}
	{Signature s, Statement s1}{Statement}
\begin{functionality}
Fortran~77 allows procedures to have multiple entry points.  The
handling of variables for multiple entry points is messy at best.  The
generation interface uses Fortran~77 semantics.  Since each entry can
have its own argument list, the location (on the stack) of a parameter
depends upon which entry point into the procedure was called.
Moreover, parameters for one entry point may not be defined by another
in which case the latter entry point cannot execute code which refers
to such parameters.

The operation of this routine is different from most routines in the
generation interface.  In order to mark where the entry point begins,
the generation interface requires that the first statement in the body
of the entry be passed in \code{s1}.  Instead of returning a 
\node{RoutineDecl} like the other routines in this section, it returns
a \node{statement}, which is in fact \code{s1}.  The \node{RoutineDecl} 
created by this routine is accessible through the \code{NameID} 
returned by this routine.
\begin{Parameters}
\Param{name} Name of the entry point.
\Param{s} Signature of the entry point.
\Param{enclosingProcedure} The procedure which encloses this entry
point. 
\Param{s1} The first statement in entry.  Note that \code{s1} is 
only the first statement executed, not the entire body of the routine
as used in other routine declarations.
\end{Parameters}
\end{functionality}



%%--
%\interface{void}{BeginBody}{}{}{}
%	%begin_procedure (p: Proc);
%	%(* begin generating code for the procedure 'p'.  Sets "current 
%	% procedure" to 'p'.  Implies a begin_block.  *)

%%--
%\interface{void}{EndBody}{}{}{}
%	%end_procedure (p: Proc);
%	%(* marks the end of the code for procedure 'p'.  Sets "current 
%	% procedure" to NIL.  Implies an end_block.  *)




	%declare_procedure (n: Name;  n_params: INTEGER;  return: Type;
	%                   lev: INTEGER;  cc: CallingConvention;
	%                   exported: BOOLEAN;  parent: Proc;
	%                   ret_type_uid: TypeUID): Proc;
	%(* declare a procedure named 'n' with 'n_params' formal parameters
	% at static level 'lev'.  Sets "current procedure" to this procedure.
	% If the name n is M3ID.NoID, a new unique name will be supplied by the
	%back-end.  The type of the procedure's result is specifed in 'return'.
	% If the new procedure is a nested procedure (level > 1) then 'parent'
	% is the immediately enclosing procedure, otherwise 'parent' is NIL.
	% The formal parameters are specified by the subsequent 'declare_param'
	% calls. *)
%(*----------------------------------------------------------- procedures ---*)

%(* Clients compile a procedure by doing:

%      proc := cg.declare_procedure (...)
%        ...declare formals...
%        ...declare locals...
%      cg.begin_procedure (proc)
%        ...generate statements of procedure...
%      cg.end_procedure (...)

%  Nested procedure bodies may be generated before their parent's
%  begin_procedure, after their parent's end_procedure, or inline
%  where they occur in the source.  If they are not inline,
%  note_procedure_origin is used to mark their original source
%  position.  Runtime flags passed to the front-end determine where
%  nested procedure bodies are generated.  Back-ends are free to
%  require any one of these three placements for nested procedures.
%  (At the moment:  m3cc -> inline,  m3back -> after)
%*)

%import_procedure (n: Name;  n_params: INTEGER;  return: Type;
%                  cc: CallingConvention): Proc;
%(* declare and import the external procedure with name 'n' and 'n_params'
%   formal parameters.  It must be a top-level (=0) procedure that returns
%   values of type 'return'.  'lang' is the language specified in the
%   procedure's <*EXTERNAL*> declaration.  The formal parameters are specified
%   by the subsequent 'declare_param' calls. *)

%declare_procedure (n: Name;  n_params: INTEGER;  return: Type;
%                   lev: INTEGER;  cc: CallingConvention;
%                   exported: BOOLEAN;  parent: Proc;
%                   ret_type_uid: TypeUID): Proc;
%(* declare a procedure named 'n' with 'n_params' formal parameters
%   at static level 'lev'.  Sets "current procedure" to this procedure.
%   If the name n is M3ID.NoID, a new unique name will be supplied by the
%   back-end.  The type of the procedure's result is specifed in 'return'.
%   If the new procedure is a nested procedure (level > 1) then 'parent' is
%   the immediately enclosing procedure, otherwise 'parent' is NIL.
%   The formal parameters are specified by the subsequent 'declare_param'
%   calls. *)

%begin_procedure (p: Proc);
%(* begin generating code for the procedure 'p'.  Sets "current procedure"
%   to 'p'.  Implies a begin_block.  *)

%end_procedure (p: Proc);
%(* marks the end of the code for procedure 'p'.  Sets "current procedure"
%   to NIL.  Implies an end_block.  *)

%note_procedure_origin (p: Proc);
%(* note that nested procedure 'p's body occured at the current location
%   in the source.  In particular, nested in whatever procedures,
%   anonymous blocks, or exception scopes surround this point. *)

%==============================================================================
\Subsection{Declaring Exceptions}
The routines in this section are for declaring exceptions.  Modula-3
has entities of Exception type; C++ never actually creates an
exception entity.  Rather than pass an entity of type Exception, when a
C++ program throws an exception, it passes the arguments to an
exception handler.

%--
\interface{NameID}{DeclException}{Identifier name, TypeID type}
	{}{ExceptionDecl}
\begin{functionality}
For exceptions without an argument, use \node{NoType} as the
\args{type} argument.
\end{functionality}

	%declare_exception (n: Name;  arg_type: TypeUID;  raise_proc: BOOLEAN;
	%                     base: Var;  offset: INTEGER);

	%(* declares exception 'n' identified with the address 'base+offset'
	% that carries an argument of type 'arg_type'.  If 'raise_proc', then
	% 'base+offset+BYTESIZE(ADDRESS)' is a pointer to the procedure that
	% packages the argument and calls the runtime to raise the exception.*)
   
%==============================================================================
\Subsection{Declaring Labels}

C and C++ treat case alternatives as labels.  This bizarre notion
stems from their excessively low-level view of the switch statement.  
The generation interface handles case alternatives differently from
general labels.  

%--
\interface{NameID}{DeclLabel}{Identifier i}{}{LabelDecl}
	%(*------------------------------------------------- ID counters ---*)
	%next_label (n: INTEGER := 1): Label;
	%(* reserve 'n' consecutive labels and return the first one *)

%==============================================================================
\Subsection{Code Units}\label{sec:unit}
A code unit groups related code.  Code units include modules and
namespaces.  

% Module names are declared, whereas file names are not
% visible to the program text.

%------------------------------------------------------------------------------
\subsubsection{Namespace}\label{sec:namespace}
A C++ namespace can be thought of as a named scope.  

%--
\interface{void}{DeclNamespaceUnitBegin}{}{}{}
%%--
%\interface{NameID}{DeclAnonNamespaceBegin}{}{}{}
%--
\interface{NameID}{DeclNamespaceUnitEnd}{Identifier name}
	{\manyPops}{NamespaceDecl nd}
\begin{functionality}
\befunc{DeclNamespaceUnit}{\node{Declaration}}
\end{functionality}

%------------------------------------------------------------------------------
\subsubsection{File}
These routines are meant for demarcating a C/C++ file scope.  These
routines are not meant for annotating a graph with which source file a
particular piece of code originates from.  
%--
\interface{void}{FileUnitBegin}{}{}{}
%--
\interface{void}{FileUnitEnd}{String name}{\manyPops}{Unit}
\begin{functionality}
\befunc{FileUnit}{\node{Declaration}}
\end{functionality}

%------------------------------------------------------------------------------
\subsubsection{Interface}
%--
\interface{void}{DeclInterfaceUnitBegin}{}{}{}
%--
\interface{NameID}{DeclInterfaceUnitEnd}{Identifier name}
	{\manyPops}{InterfaceDecl}
\begin{functionality}
\befunc{DeclInterfaceUnit}{\node{Declaration}}
\end{functionality}

%------------------------------------------------------------------------------
\subsubsection{Module}
%--
\interface{void}{ImportUnit}{NameID interface, Identifier name}{}{Import}
\begin{functionality}
This routine imports interface \code{i} as \code{name}.  If
\code{name} is NoName, then the interface is not renamed.
\end{functionality}
%--
\interface{void}{ImportMember}{NameID interface, NameID declaration}
	{}{Import}
\begin{functionality}
This routine imports a single member (\code{declaration}) of
\code{interface}.  Note that this routine accepts more information
(\ie \code{interface}) than it strictly needs.
\end{functionality}
%--
\interface{void}{ImportsBegin}{}{}{}
%--
\interface{void}{ImportsEnd}{}{\manyPops}{Imports}
\begin{functionality}
\befunc{Imports}{\node{Import}}
\end{functionality}

%--
\interface{void}{ExportsBegin}{}{}{}
%--
\interface{void}{ExportsEnd}{}{\manyPops}{Exports}
\begin{functionality}
\befunc{Exports}{\node{Identifier}}
\end{functionality}

%--
\interface{void}{DeclModuleUnitBegin}{}{}{}
%--
\interface{NameID}{DeclModuleUnitEnd}{Identifier name}{\manyPops}{Unit}
\begin{functionality}
\befunc{DeclModuleUnit}{\node{Declaration}}
\end{functionality}

%==============================================================================
%\Subsection{Miscellaneous Declarations}
%This section includes those declarations that do not warrant their own
%individual sections.  Generally, these are constructs which are
%limited to a single language.  

%------------------------------------------------------------------------------
%\subsubsection{Initializers}
%%--
%\interface{void}{DefaultValue}{}{Declaration d, Expression e}{Declaration}
%\begin{functionality}
%This routine specifies a default value for an entity.  A default value
%is much like an initial value, except that a default value can be
%overridden.  This routine will generally only be used with formal
%parameters.
%\end{functionality}

%%--
%\interface{void}{InitialValue}{}{Declaration d, Expression e}{Declaration}
%\begin{functionality}
%This routine specifies an initial value.  
%\end{functionality}

%==============================================================================
\Subsection{Setting Declaration Attributes}

Declarations may have attributes associated with them.  The
current design specifies the attributes in an enumerated type.  

An alternate design would use a separate routine for each attribute, so
that a compiler/linker could catch non-compliance to the interface.
However, it seems unnecessarily verbose when compared to a single
function with an enumerated type

The attributes break down into roughly five categories.  The first
group of attributes define which entities have values that can change.
The second group of attributes indicate in what part of memory an
entity should be located.  The most important aspect of the location
decision is whether or not the entity's value is saved across
procedure calls.  The third group of attributes determine the
visibility of the entity.  These attributes correspond to C/C++'s
notion of linkage.  The remaining two groups of attributes control
method visibility and overloading.

The \code{DeclAttributes} enumerated type is defined as follows:

\EnumOptions{DeclAttributes}{ cConstantDecl, cMutable, cLocationStack,
    cLocationRegister, cLocationStatic, cLocationInline, cLinkageLocal,
    cLinkageFile, cLinkageGlobal, cLinkageForeign, cPublic, cProtected,
    cPrivate, cAbstract, cNonvirtual, cVirtual, cNoConstructorCalls,
    cNoDestructorCalls, cDestructorCalls }

where:
\begin{Description}
\item [] \partitle{Entity mutability}
The mutability attributes only apply to variables, fields, and routines.
\item [cConstantDecl] Indicates that an entity is a constant.  This
attribute is equivalent to the type attribute of the same name, but
this one is preferred.
\item [cMutable] Indicates than an entity is \emph{not} a constant.
This is the default value for mutability.  It only needs to be
explicitly specified to indicate that a component of a constant
aggregate entity is not constant (as in C++).

\item [] \partitle{Entity location}
The location attributes only apply to variables and fields.
\item [cLocationStack] Marks a data value as being locally 
allocated (\ie on the stack).  Local allocation is the default form of
allocation and so seldom (if ever) needs to be explicitly specified.
This attribute represents C/C++'s \key{auto} construct.
\item [cLocationRegister] Recommends that a value be assigned to a register.
If the value cannot be assigned to a register, it should be allocated
on the stack.
\item [cLocationStatic] Indicates that an entity is assigned to permanently
allocated space.
\item [cLocationInline] Recommends that a procedure be inlined.  This
attribute applies only to procedures and methods.

\item [] \partitle{Entity visibility}
The entity visibility attributes apply to all declarations.  
\item [cLinkageLocal] Indicates that an entity is visible only within
its current scope.  This is the default linkage.
\item [cLinkageFile] Indicates that the entity is visible only within
its file scope.  This attribute is intended to represent one meaning
of the C/C++ \key{static} construct.  Admittedly, this attribute
should be redundant with \code{linkageLocal} when at the file scope
level.  However, this captures the notion of C/C++'s \key{static}
construct; whereas, \code{linkageLocal} should never be used.
\item [cLinkageGlobal] Indicates that an entity has globally visible
in the program.  This attribute implements the common case of C/C++'s
\key{extern} construct.
\item [cLinkageForeign] Indicates that an entity is visible to a
different source language.  This attribute eliminates C++'s
overloading of the \key{extern} keyword.

\item [] \partitle{Method visibility}
The method visibility attributes potentially apply to all
declarations, but for currently support languages, the attributes only
affect fields and methods.
\item [cPublic] Specifies that an identifier is visible outside of its
namespace.  This value is the default.
\item [cProtected] Specifies than an identifier is visible only within
its namespace.  This attribute may only be used for class members, in
which case it denotes the semantics of C++'s \key{protected} construct.
\item [cPrivate] Specifies that an identifier is visible only within
its namespace.

\item [] \partitle{Method overloading}
The method overloading attributes apply only to methods.
\item [cAbstract] Marks a method as undefined for the current class,
and therefore requiring definition in derived classes.  By default,
the generation interface assumes that all methods are fully defined.
\item [cNonvirtual] Indicates that a method may not be overloaded.
\item [cVirtual] Indicates that a method may be overloaded.  This value
is the default.

\item [] \partitle{Automatic methods}
The automatic method attributes apply only to class declarations.
\item [cNoConstructorCalls] Marks a class as not requiring automatic
invocation of its constructor.  This is the default value.
\item [cConstructorCalls] Marks a class as requiring automatic
invocation of its constructor.
\item [cNoDestructorCalls] Marks a class as requiring automatic
invocation of its destructor.  This is the default value.
\item [cDestructorCalls] Marks a class as requiring automatic
invocation of its destructor.  Note that automatic constructor and destructor
are indicated separately in order to support Java which does not
have destructors.
\item [cMainProcedure] Marks the procedure as the main procedure in
the program.

\end{Description}

%--
\interface{void}{SetDeclarationAttribute}{DeclAttributes da}
	{Declaration d}{Declaration}
\begin{functionality}
This routine assigns attribute \code{da} to declaration \node{d}.
For those attributes which do not require a value, use NoExpression.  
\end{functionality}




% -*- Mode: latex; Mode: auto-fill; -*-

\Section{Generating Statements}\label{sec:stmt}
%(*----------------------------------------------------------- statements ---*)

This section describes the routines available for representing
statements.  

%(*------------------------------------------------------------ load/store ---*)
%(* Note: When an Int_A..Int_D value is loaded, it is sign extended to
%   full Int width.  When a Word_A..Word_D value is loaded, it is zero
%   extended to full Word width.  When an Int_A..Int_D or Word_A..Word_D
%   value is stored, it is truncated from full width to the indicated size.

%   Note: all loads and stores are aligned according to the specified type.
%*)

%load (v: Var;  o: ByteOffset;  t: MType);
%(* push; s0.t := Mem [ ADR(v) + o ].t *)

%load_address (v: Var;  o: ByteOffset := 0);
%(* push; s0.A := ADR(v) + o *)

%load_indirect (o: ByteOffset;  t: MType);
%(* s0.t := Mem [s0.A + o].t *)

%qualify_expr (holder: TypeUID; field: TEXT; 
%t: Type; lval: BOOLEAN; obj: BOOLEAN);

%osubscript_expr (t: Type; elt_size: INTEGER);
%(* s1 := s0[s1]; s0 is an open array type. pop *)

%(* Note that in osubscript and fsubscript, the order of array and index
%   on the stack is different.  Unfortunate... *)
%fsubscript_expr (t: Type; elt_size: INTEGER);
%(* s0 := s1[s0]; s1 is a fixed array type. pop *)

%array_size (n: INTEGER);
%(* s0 := number of elements in nth dimension of open array s0 *)

%store (v: Var;  o: ByteOffset;  t: MType);
%(* Mem [ ADR(v) + o : s ].t := s0.t; pop *)

%store_indirect (o: ByteOffset;  t: MType);
%(* Mem [s1.A + o].t := s0.t; pop (2) *)


%store_ref (v: Var;  o: ByteOffset := 0);
%(* == store (v, o, Type.Addr), but also does reference counting *)

%store_ref_indirect (o: ByteOffset;  var: BOOLEAN);
%(* == store_indirect (o, Type.Addr), but also does reference counting.
%     If "var" is true, then reference counting depends on whether the
%     effective address is in the heap or stack. *)

%==============================================================================
\Subsection{Block Statements}
Different programming languages have different rules about when a
block opens a scope.  The generation interface separates the issues of
grouping statements from that of declaring scopes (see
Section~\ref{sec:symbolTable}.  Hence, none of the routines in this
section imply the creation of a new scope.  A block of statements
may have an associated scope (\eg a procedure
with local variables).  Client code is responsible for
associating a scope with a block of statements.

\interface{void}{StmtBlockBegin}{}{}{}
\interface{void}{StmtBlockEnd}{}{\manyPops}{Statement}
\begin{functionality}
\befunc{StmtBlock}{\node{Statement}}
\end{functionality}

%==============================================================================
\Subsection{Labeled Statements}\label{sec:label}
This section describes routines for representing labeled statements.

\interface{void}{StmtLabel}{}{LabelDecl, Statement}{LabelStmt}

%==============================================================================
\Subsection{Conditional Statements}\label{sec:cond}

This section describes routines for representing decision statements.
This section also defines routines for specifying conditions.  

%------------------------------------------------------------------------------
\subsubsection{If statements}
The routines in this section handle \key{if} statements.  The
interface does not provide direct support for \key{else-if} clauses,
so these clauses will have to be transformed to nested \key{if}
statements.  The design of these routines require that for nested if
statements, outer \key{if} statements remain on the stack until inner
\key{if} statements have been processed.


%--
\interface{void}{StmtIfThenElse}{}
	{Expression e, Statement s1, Statement s2}{Statement}
\begin{functionality}
The expression representing the test must be of boolean type.  If the
statement does not have an \key{else} clause, then a null statement
should be used for \node{s2}.
\end{functionality}

%--
\interface{void}{StmtArithmeticIf}{NameID lessLabel, NameID equalLabel,
	NameID moreLabel}{Expression e}{Statement}
\begin{functionality}
This routine implements the semantics of Fortran~77's arithmetic if
statement.  
The expression representing the test must be of boolean type.
\end{functionality}

%------------------------------------------------------------------------------
\subsubsection{Multi-way Branch Statements}
These routines perform a single test and jump to one of potentially
many points in the program.

%--
\interface{void}{LabelsBegin}{}{}{}
%--
\interface{void}{LabelsEnd}{}{}{Labels}
\begin{functionality}
\befunc{Labels}{LabelRef}  
\end{functionality}
%--
\interface{void}{StmtComputedGoto}{}{Labels l, Expression e}{Statement}
\begin{functionality}
\fortranSemantics{computed goto}
It is similar to a switch statement, except that the
target labels are not limited to a single statement of code.
\end{functionality}
%--
\interface{void}{StmtAssignLabel}{NameID value, NameID label}{}{Statement}
\begin{functionality}
\fortranSemantics{assign}

Note that the semantics of this operation could be represented with a
type conversion from integer to label.  However, this would require
creating variables of type label.  Since labels are deprecated, this
more direct representation has been chosen.
\end{functionality}
%--
\interface{void}{StmtAssignedGoto}{}{Expression e, Labels l}{Statement}
\begin{functionality}
\fortranSemantics{assigned goto}
\end{functionality}

%------------------------------------------------------------------------------
\subsubsection{Case/Switch Statements}
The generation interface supports two types of multiple test
conditional statement.  The first form is a well structured statement,
as found in Modula-3.  We refer to this form as a case statement.  The
body of a case statement consists of blocks of code, each of which
when paired with their case keys is called a case alternatives.
Each time a program passes through a case statement, it will execute
zero or one of the case alternatives.  A case alternative is executed
when the value of the conditional expression matches one of its case
keys.  The other form matches the unstructured form of C/C++'s
\key{switch} statement.   This form still uses case keys, but does not
separate the \key{switch} statement body into case alternatives.

The generation interface follows Modula-3's terminology of \key{case}
statement.  Each unique body of code within a \key{case} statement is
termed a case alternative, and each case alternative may have more
than one case key.  A case key is the value which is compared against
the expression at the top of the \key{case} statement.

\interface{void}{CaseKeyBegin}{}{}{}
\interface{void}{CaseKeyEnd}{}{\manyPops}{CaseKeys}
\begin{functionality}
\befunc{CaseKey}{\node{Expression}}
\end{functionality}
\interface{void}{CaseKey}{}{Expression e}{CaseKeys}
\begin{functionality}
The routines \method{CaseKeyBegin} and \method{CaseKeyEnd}
build a list of case keys.  The \method{CaseKey} routine is a short
cut for building a list with one member.
\end{functionality}

\interface{void}{CaseAlt}{}{CaseKeys ck, Statement s}{CaseAlt}
%\interface{void}{CaseAlt}{}{CaseKeys ck, StatementList s}{CaseAlt}

\interface{void}{CaseOthersAlt}{}{StatementList}{CaseAlt}
\begin{functionality}
This routine represents the default case alternative.
\end{functionality}

\interface{void}{CaseBodyBegin}{}{}{}
\interface{void}{CaseBodyEnd}{}{\manyPops}{CaseAlts}
\begin{functionality}
The routines \method{CaseBodyBegin} and \method{CaseBodyEnd}
build the body of a case statement.

\befunc{CaseBody}{\node{CaseAlt}}
\end{functionality}

\interface{void}{StmtCase}{}{Expression e, CaseAlts}{Statement}
\begin{functionality}
This routine implements the semantics of Modula-3's \key{case}
statement, in which the order of the case alternatives does not matter.
The expression representing the test must be of boolean type.
\end{functionality}

\interface{void}{StmtSwitch}{}{Expression e, Statement s}{Statement}
\begin{functionality}
This routine implements the semantics of C/C++'s \key{switch} construct.
The expression representing the test must be of boolean type.
\end{functionality}

%------------------------------------------------------------------------------
\subsubsection{Typecase}
Modula-3's \key{typecase} statement allows a conditional expression
based on an expressions type.  The routines provided for handling a
\key{typecase} mirror those provided for the \key{case} statement.

%--
\interface{void}{TypecaseKey}{TypeID type, NameID variable}
	{}{TypecaseKey}
\begin{functionality}
If \code{variable} is NoName, then no variable has been specified.
\end{functionality}
%--
\interface{void}{TypecaseKeyBegin}{}{}{}
%--
\interface{void}{TypecaseKeyEnd}{}{\manyPops}{TypecaseKeys}
\begin{functionality}
\befunc{TypecaseKey}{\node{TypecaseKeys}}
\end{functionality}

%--
\interface{void}{TypecaseAlt}{}{TypecaseKeys tck, Statement s}{TypecaseAlt}
%\interface{void}{TypecaseAlt}{}{TypecaseKeys tck, StatementList s}{TypecaseAlt}

%--
\interface{void}{TypecaseBodyBegin}{}{}{}
%--
\interface{void}{TypecaseBodyEnd}{}{\manyPops}{TypecaseAlts}
\begin{functionality}
\befunc{TypecaseBody}{\node{TypecaseAlt}}
\end{functionality}
%--
\interface{void}{StmtTypecase}{}{Expression e, TypeCaseAlts}{Statement}
\begin{functionality}
This routine implements the semantics of Modula-3's \key{case}
statement.  In a \key{typecase} the order of the case alternatives
does matter.
The expression representing the test must be of boolean type.
\end{functionality}

%case_jump (READONLY labels: ARRAY OF Label);
%(* tmp := s0.I; pop; GOTO labels[tmp]  (NOTE: no range checking on s0.I) *)

\todo{
if_true  (l: Label;  f: Frequency);
%(* tmp := s0.I; pop; IF (tmp # 0) GOTO l *)

if_false (l: Label;  f: Frequency);
%(* tmp := s0.I; pop; IF (tmp = 0) GOTO l *)
}

%if_eq (l: Label;  t: ZType;  f: Frequency); (*== eq(t); if_true(l,f)*)
%if_ne (l: Label;  t: ZType;  f: Frequency); (*== ne(t); if_true(l,f)*)
%if_gt (l: Label;  t: ZType;  f: Frequency); (*== gt(t); if_true(l,f)*)
%if_ge (l: Label;  t: ZType;  f: Frequency); (*== ge(t); if_true(l,f)*)
%if_lt (l: Label;  t: ZType;  f: Frequency); (*== lt(t); if_true(l,f)*)
%if_le (l: Label;  t: ZType;  f: Frequency); (*== le(t); if_true(l,f)*)

%==============================================================================
\Subsection{Looping Statements}

\interface{void}{StmtWhileLoop}{}{Expression e, Statement s}{Statement}
\begin{functionality}
The expression representing the test must be of boolean type.
\end{functionality}

\interface{void}{StmtRepeatWhileLoop}{}{Statement s, Expression e}{Statement}
\begin{functionality}
The expression representing the test must be of boolean type.
\end{functionality}

\interface{void}{StmtRepeatUntilLoop}{}{Statement s, Expression e}{Statement}
\begin{functionality}
The expression representing the test must be of boolean type.
\end{functionality}

\interface{void}{StmtLoop}{}{Statement s}{Statement}

%--
\interface{void}{StmtDoLoop}{NameID index}{Expression first, 
	Expression last, Expression step, Statement s}{Statement}
\begin{functionality}
This routine represents an iterating loop.  When \args{step} is
negative, the loop terminates when \args{index} becomes lower than
\code{last}.  Otherwise, the loop terminates when \args{index} exceeds
\args{last}.  The \args{first}, \args{last}, and \args{step}
expressions are evaluated once, at entry to the loop.  
\begin{Parameters}
\Param{index} Variable (or other assignable name) being incremented.
\Param{first} Initial value of \code{index}.
\Param{last} Maximum value of \code{index}.
\Param{step} Amount by which to increment \code{index} each time
through the loop.
\end{Parameters}

This routine follows Fortran~77 semantics, which requires that the
index be updated.  However, some source languages do not guarantee
that the value of the index is meaningful after the loop finishes,
which provides additional opportunities for optimizations.  For such
source languages, user code may create a local (\ie temporary)
variable for the loop index.  In this case, the index will have no
uses beyond the loop, so client code may optimize more aggressively.

\end{functionality}
%--
\interface{void}{StmtForLoop}{}
	{Expression d, Expression e, Expression e, Statement s}{Statement}
\begin{functionality}
The expression representing the test must be of boolean type.
\end{functionality}
%==============================================================================
\Subsection{Branch Statements}

This section lists branching statements.  The statements included here
are simple branches, Fortran~77 has several elaborate branch
statements which may be found in Section~\ref{sec:cond}.

\interface{void}{StmtBreak}{}{}{Statement}
\interface{void}{StmtContinue}{}{}{Statement}
\interface{void}{StmtGoto}{NameID label}{}{Statement}
\interface{void}{StmtReturn}{}{Expression e}{Statement}
\begin{functionality}
If the return statement does not have an expression, then
\node{NoExpression} should be used.
\end{functionality}
\interface{void}{StmtExit}{}{Expression e}{Statement}
\begin{functionality}
This routine terminates a program.  If no expression is specified,
\node{NoExpression} should be used.
\end{functionality}
	%exit_proc (t: Type);
	%(* Returns s0.t if t is not Void, otherwise returns no value. *)
	%jump (l: Label);
	%(* GOTO l *)

\interface{void}{StmtThrow}{NameID value}{}{Statement}
\begin{functionality}
This routine throws a C++ style exception and branches to an
appropriate exception handler.
\end{functionality}

\interface{void}{StmtRaise}{NameID exception}{Expression e}{Statement}
\begin{functionality}
This routine raises a Modula-3 style exception and branches to an
appropriate exception handler.  If no expression is available,
\node{NoExpression} should be used.
\end{functionality}

%--
\interface{void}{StmtAlternateReturn}{}{Expression e}{Statement}
\begin{functionality}
This routine implements the semantics of Fortran~77's alternate
return construct.  The expression indexes the special alternate return
parameters.  
\end{functionality}

%==============================================================================
\Subsection{Exception Handling Statements}

Though exception handling is similar in Modula-3 and C++, The
generation interface provides separate routines for these languages.
The primary difference is that a Modula-3 exception passes an entity
of type \node{Exception}, whereas a C++ exception passes an argument
list for an exception handler.  The syntax of their respective
\key{try} blocks is also slightly different.

\partitle{Modula-3 style exceptions}
%--
\interface{void}{ExceptionKey}{NameID e}{}{ExceptionKey}
%--
\interface{void}{ExceptionKeyBegin}{}{}{}
%--
\interface{void}{ExceptionKeyEnd}{}{\manyPops}{ExceptionKeys}
\begin{functionality}
\befunc{ExceptionKey}{\node{ExceptionKey}}
\end{functionality}

%--
\interface{void}{ExceptionHandler}{}{ExceptionKeys eks, Statement s}{Handler}
%--
\interface{NameID}{ExceptionHandlerWithArgument}
	{NameID exception, Identifier i}{Statement s}{Handler}
\begin{functionality}
This routine represents an exception with an argument.  Generation
interface implementations are responsible for generating the
declaration for \args{i}.
\end{functionality}
%--
\interface{void}{ElseHandler}{}{Statement s}{Handler}
\begin{functionality}
This routine implements Modula-3's else handler, which is called for
any exception that reaches it.
\end{functionality}

%--
\interface{void}{HandlerBegin}{}{}{}
%--
\interface{void}{HandlerEnd}{}{\manyPops}{Handlers}
\begin{functionality}
Each try block has a list of elements of type \node{Handler}
associated with it.  

\befunc{Handler}{\node{Handler}}
\end{functionality}

%--
\interface{void}{StmtTryExcept}{}{Statement s, Handlers hs}{Statement}
\begin{functionality}
This routine implements Modula-3's \key{tryExcept} construct.  
\end{functionality}
%--
\interface{void}{StmtTryFinally}{}{Statement s1, Statement s2}{Statement}
\begin{functionality}
This routine implements Modula-3's \key{tryFinally} construct.  
\end{functionality}

\partitle{C++ style exceptions}

%--
\interface{void}{CatchException}{}{FormalDecl vd, Statement stmt}{Catch}
%--
\interface{void}{CatchAll}{}{Statement stmt}{Catch}
\begin{functionality}
This routine represents a catch clause that can handle any exception.
This routine represents C++'s ``...'' notation.
\end{functionality}

%--
\interface{void}{CatchBegin}{}{}{}
%--
\interface{void}{CatchEnd}{}{\manyPops}{Catchers}
\begin{functionality}
Each try block has a list of elements of type \node{Catch}
associated with it.  

\befunc{Catch}{\node{Catch}}
\end{functionality}

%--
\interface{void}{StmtTry}{}{Statement s, Catchers cs}{Statement}
\begin{functionality}
This routine implements C++'s \key{try} construct.  
\end{functionality}

\todo{
(*------------------------------------------------ traps & runtime checks ---*)

assert_fault   ();
narrow_fault   ();
return_fault   ();
case_fault     ();
typecase_fault ();
(* Abort *)

check_nil ();
(* IF (s0.A = NIL) THEN Abort *)

check_lo (READONLY i: Target.Int);
(* IF (s0.I < i) THEN Abort *)

check_hi (READONLY i: Target.Int);
(* IF (i < s0.I) THEN Abort *)

check_range (READONLY a, b: Target.Int);
(* IF (s0.I < a) OR (b < s0.I) THEN Abort *)

check_index ();
(* IF NOT (0 <= s1.I < s0.I) THEN Abort END; pop
   s0.I is guaranteed to be positive so the unsigned
   check (s0.W < s1.W) is sufficient. *)

check_eq ();
(* IF (s0.I # s1.I) THEN Abort;  Pop (2) *)
}

%==============================================================================
\Subsection{Miscellaneous Statements}

\interface{void}{StmtSpcfyUsing}{NameID id}{}{Statement}

\interface{void}{StmtUsingDirective}{NameID namespace}{}{Statement}

\interface{NameID}{StmtWithAlias}{Identifier id}
	{Expression e, Statement s}{Statement}

\interface{void}{StmtEval}{}{Expression e}{Statement}
\begin{functionality}
This routine converts an expression to a statement.  It executes an
expression and then discards the result.  Hence, the expression is
being executed only for its side effect.  This routine implements
Modula-3's \key{eval} construct and is used to represent C/C++'s
implicit conversion from an expression to a statement.  
\end{functionality}

\interface{void}{StmtDeclStmt}{}{Declaration d}{Statement}
\begin{functionality}
This routine converts a declaration to a statement, which allows a
statement block to consist of all statements.
\end{functionality}

\interface{void}{StmtNull}{}{}{Statement}
\begin{functionality}
This routine represents a null statement.  
\end{functionality}

\todo{
set_label (l: Label;  barrier: BOOLEAN := FALSE);
(* define 'l' to be at the current pc, if 'barrier', 'l' bounds an exception
   scope and no code is allowed to migrate past it. *)

%(*---------------------------------------------------------------- misc. ---*)
comment (a, b, c, d: TEXT := NIL);
(* annotate the output with a&b&c&d as a comment.  Note that any of a,b,c or d
   may be NIL. *)
}


% -*- Mode: latex; Mode: auto-fill; -*-

\Section{Generating Expressions}\label{sec:op}

This section shows the interface routines used to specify expressions.
Note that the interface implementation is responsible for identifying
type specific versions of the operators.

Operators can represent user-defined functions in languages which
allow operators to be overloaded (\eg C++).  In the generation
interface, operators represent only language defined operators, not
overloaded operators.  Users should translate overloaded operators to
function calls.

Tables~\ref{tab:ops1},~\ref{tab:ops2}, and~\ref{tab:ops3} show the
correspondences between the interface's routines and operators of a few
languages.  This table shows correspondences between operators which
programmers may use and the routines of the generation interface.
However, user code may use operators which the programmer cannot
directly use (\eg C/C++ implicitly truncate real numbers but users of
this interface must explicitly call a conversion routine).

\begin{table}
\centering
\begin{tabular}{|l||c|c|c|}\hline
\multicolumn{4}{|c|}{Operator Correspondences-- Part I} \\\hline
\textbf{Routine Name} & \textbf{C++} & \textbf{Modula-3} & \textbf{Fortran~77}
\\\hline
\multicolumn{4}{|c|}{\tabhead{Assignment Operators}} \\\hline
ExpAssignment 			& =	& :=	& =	\\\hline
\multicolumn{4}{|c|}{\tabhead{Arithmetic Operators}} \\\hline
ExpPositive			& +	& +	& \na	\\\hline
ExpNegative			& $-$	& $-$	& $-$	\\\hline
ExpAbsoluteValue			& \na	& abs	& abs, ?abs \\\hline
ExpMinimum				& \na	& min	& min, ?min$*$ \\\hline
ExpMaximum				& \na	& max	& max, ?max$*$ \\\hline
ExpAddition			& +	& +	& +	\\\hline
ExpSubtraction 			& $-$	& $-$	& $-$	\\\hline
ExpMultiplication			& $*$	& $*$	& $*$	\\\hline
ExpDivision 			& /	& /	& /	\\\hline
ExpModulus 			& \na	& \na 	& \na 	\\\hline
ExpRemainder 			& \%	& mod	& mod, ?mod \\\hline
ExpExponentiation			& \na	& \na	& $**$	\\\hline
ExpPreDecrement			& $--$	& \na 	& \na	\\\hline
ExpPreIncrement			& ++	& \na	& \na	\\\hline
ExpPostDecrement 		& $--$	& \na	& \na	\\\hline
ExpPostIncrement 		& ++	& \na	& \na	\\\hline
\multicolumn{4}{|c|}{\tabhead{Relational Operators}} \\\hline
ExpEquality 			& ==	& =	& .EQ.	\\\hline
ExpNotEqual 			& !=	& \#	& .NE.	\\\hline
ExpGreater 			& $>$	& $>$	& .GT.	\\\hline
ExpGreaterEqual			& $>=$	& $>=$	& .GE.	\\\hline
ExpLess				& $<$	& $<$	& .LT.	\\\hline
ExpLessEqual 			& $<$	& $<=$	& .LE.	\\\hline
\multicolumn{4}{|c|}{\tabhead{Bitwise Operators}} \\\hline
ExpBitComplement 		& $\sim$& \na	& \na	\\\hline
ExpBitAnd 			& \&	& \na	& \na	\\\hline
ExpBitXor 			& $\wedge$& \na	& \na	\\\hline
ExpBitOr 			& $\mid$& \na	& \na	\\\hline
ExpBitShiftLeft 		& $<<$	& \na	& \na	\\\hline
ExpBitShiftRight 		& $>>$	& \na	& \na	\\\hline
\end{tabular}
\caption{\label{tab:ops1}Correspondence between generation interface
ExpRoutine names and language operators.  
ExpContinued in Table~\protect\ref{tab:ops2}.}
\end{table}

%--

\begin{table}
\centering
\begin{tabular}{|l||c|c|c|}\hline
\multicolumn{4}{|c|}{Operator Correspondences-- Part II} \\\hline
\textbf{Routine Name} & \textbf{C++} & \textbf{Modula-3} & \textbf{Fortran~77}
\\\hline
\multicolumn{4}{|c|}{\tabhead{Compound Assignment Operators}} \\\hline
ExpMultiplicationAssignment 	& $*=$	& \na	& \na	\\\hline
ExpDivisionAssignment		& /=	& \na	& \na	\\\hline
ExpRemainderAssignment		& \%=	& \na	& \na	\\\hline
ExpAdditionAssignment		& +=	& \na	& \na	\\\hline
ExpSubtractionAssignment 	& $-\!=$& \na	& \na	\\\hline
ExpBitShiftRightAssignment	& $>>$=	& \na	& \na	\\\hline
ExpBitShiftLeftAssignment	& $<<$=	& \na	& \na	\\\hline
ExpBitAndAssignment		& \&=	& \na	& \na	\\\hline
ExpBitXorAssignment		& $\wedge\!=$& \na & \na\\\hline
ExpBitOrAssignment 		& $\mid=$& \na	& \na	\\\hline
\multicolumn{4}{|c|}{\tabhead{Logical Operators}} \\\hline
ExpNot				& !	& not	& .NOT.	\\\hline
ExpAnd				& \na	& \na	& .AND.	\\\hline
ExpOr				& \na	& \na	& .OR.	\\\hline
ExpAndConditional 		& \&\&	& and	& \na	\\\hline
ExpOrConditional 		& $\mid\mid$& or & \na	\\\hline
ExpArithmeticIf			& ?:	& \na	& \na	\\\hline
\multicolumn{4}{|c|}{\tabhead{Set Operators}} \\\hline
ExpSetEquality			& \na	& =	& \na	\\\hline
ExpUnion				& \na	& +	& \na	\\\hline
ExpDifference 			& \na	& $-$	& \na	\\\hline
ExpIntersection			& \na	& $*$	& \na	\\\hline
ExpSymmetricDifference		& \na	& /	& \na	\\\hline
ExpSubset				& \na	& $<\!=$& \na	\\\hline
ExpSuperset 			& \na	& $>\!=$& \na	\\\hline
ExpElement 			& \na	& in	& \na	\\\hline
\multicolumn{4}{|c|}{\tabhead{Pointer Operators}} \\\hline
ExpAddress 			& \&	& adr	& \na	\\\hline
ExpDereference 			& $*$	& $\sim$& \na	\\\hline
ExpNil	 			& \na	& \textsl{nil}&	\na \\\hline
\multicolumn{4}{|c|}{\tabhead{Aggregate Operators}} \\\hline
ExpThis				& this	& \na	& \na	\\\hline
ExpSelect				& .	& .	& \na	\\\hline
ExpSelectIndirect 		& $-\!>$& .	& \na	\\\hline
ExpSelectRelative	 	& $.*$	& \na	& \na	\\\hline
ExpSelectRelativeIndirect		& $-\!>\!*$& \na & \na	\\\hline
\end{tabular}
\caption{\label{tab:ops2}Correspondence between generation interface
ExpRoutine names and language operators.
ExpContinued from Table~\protect\ref{tab:ops1}.  
ExpContinued in Table~\protect\ref{tab:ops3}.}
\end{table}

%--

\begin{table}
\centering
\begin{tabular}{|l||c|c|c|}\hline
\multicolumn{4}{|c|}{Operator Correspondences-- Part III} \\\hline
\textbf{Routine Name} & \textbf{C++} & \textbf{Modula-3} & \textbf{Fortran~77}
\\\hline
\multicolumn{4}{|c|}{\tabhead{Array Operators}} \\\hline
ExpSubscript 			& $[]$	& $[]$	& ()	\\\hline
ExpArrayEquality			& \na	& =	& .EQ.	\\\hline
ExpArrayInequality			& \na	& \#	& .NE.	\\\hline
ExpArrayGreater			& \na	& \na	& .GT.	\\\hline
ExpArrayGreaterEqual		& \na	& \na	& .GE.	\\\hline
ExpArrayLess			& \na	& \na	& .LT.	\\\hline
ExpArrayLessEqual			& \na	& \na	& .LE.	\\\hline
ExpSlice				& \na	& subarray & : 	\\\hline
ExpRemainingSlide			& \na	& \na	& : 	\\\hline
ExpConcatenation			& \na	& \na	& //	\\\hline
\multicolumn{4}{|c|}{\tabhead{Invocation Operators}} \\\hline
ExpCallProcedure			& ()	& ()	& ()	\\\hline
ExpCallMethod 			& () 	& () 	& \na 	\\\hline
\multicolumn{4}{|c|}{\tabhead{Heap Operators}} \\\hline
ExpAllocate 			& new	& new	& \na	\\\hline
ExpDelete				& delete& dispose & \na	\\\hline
ExpDeleteArray			& delete[]& dispose &\na\\\hline
\multicolumn{4}{|c|}{\tabhead{Type Operators}} \\\hline
ExpBytesizeVariable		& sizeof& bytesize &\na	\\\hline
ExpBytesizeType			& sizeof& bytesize &\na	\\\hline
ExpBitsizeVariable			& \na	& bitsize & \na	\\\hline
ExpBitsizeType 			& \na	& bitsize & \na	\\\hline
ExpAdrsizeVariable		& \na	& adrsize & \na	\\\hline
ExpAdrsizeType 			& \na	& adrsize & \na	\\\hline
ExpIstype				& \na	& istype & \na	\\\hline
ExpNarrow				& \na	& narrow & \na	\\\hline
ExpTypecode 			& \na	& typecode &\na	\\\hline
%Ordinal 			& \na	& ord	& \na	\\\hline
%Value				& \na	& val	& \na	\\\hline
ExpNumber				& \na	& number & \na 	\\\hline
ExpFirst				& \na	& first	& \na	\\\hline
ExpLast				& \na	& last	& \na	\\\hline
\multicolumn{4}{|c|}{\tabhead{Type Conversion Operators}} \\\hline
ExpTypeConversion 		& ()	& \na	& \na	\\\hline
\multicolumn{4}{|c|}{\tabhead{Miscellaneous Operators}} \\\hline
ExpSeries				& ,	& \na 	& \na	\\\hline
ExpParentheses 			& ()	& ()	& ()	\\\hline
ExpSelectScope			& ::	& .	& \na	\\\hline
ExpAggregation pair		& \{\}	& \{\}	& \na	\\\hline
\end{tabular}
\caption{\label{tab:ops3}Correspondence between generation interface
routine names and language operators.
Continued from Table~\protect\ref{tab:ops2}.  }
\end{table}

\clearpage


Operators are first divided by the types of their arguments and then
by their type of function.  Calls to overloaded operators are
represented as routine calls, since user-defined operators may not fit
neatly in our categories.  

The generation interface does not have any rules for type conversion.
Therefore, user code is responsible for inserting explicit type
conversions.  Unless otherwise stated, binary operators require both
operands to be of the same type, and operators return an entity of the
same type as their operands.

%==============================================================================
\Subsection{Base Expressions}

A base expression is an expression without subexpressions.  Some base
expressions are listed in following sections with related operators.

%------------------------------------------------------------------------------
\subsubsection{Identifier Reference}
\interface{void}{ExpIdReference}{NameID entity}{}{Expression}
\begin{functionality}
This routine represents the use of a declared entity.
\end{functionality}

\partitle{Scope operators}
The scope operator allows programmers to access entities which are
hidden by other uses of the same identifier.  Resolving which
declared entity is intended by a particular reference is part of semantic
resolution, and therefore handled by the language parser.  Hence,
these routines are superfluous and only exist for providing additional
information to client code.  Note that ultimately user code must call the
\method{ExpIdReference} routine to gain access the entity's value.

\interface{NameID}{SelectScope}{NameID entity1, NameID entity2}
	{}{Reference}
\begin{functionality}
This operator searches the scope named by \code{entity1} for
\code{entity2}.  Note that if \code{entity2} is a type, this returns a
reference to the type's declaration, not the actual type.
\end{functionality}

\interface{NameID}{SelectGlobalScope}{NameID entity}{}{Reference}
\begin{functionality}
This operator searches the global scope named for identifier \code{entity}.
Note that if \code{entity2} is a type, this returns a
reference to the type's declaration, not the actual type.
\end{functionality}

%------------------------------------------------------------------------------
\subsubsection{Literals}
This section describes routines for expressing literals.  

\interface{void}{ExpLiteral}{TypeID type, String value}{}{Expression}
\begin{functionality}
Literals are difficult to handle since we may be cross compiling.
Hence, all literals are transfered across the interface as a String
constant.  Implementations of the interface must be able to convert
numeric constants from strings to a numeric value.  Equality of
literals should be determined by comparing numeric values, rather than
string values.
\end{functionality}

%%--
%\interface{void}{ExpConstruct}{}{Type t, Expression literal}{Expression}
%\begin{functionality}
%This routine constructs an unnamed entity of a given type from a
%literal value(s).  This routine handles the invocation of a type
%\end{functionality}

%------------------------------------------------------------------------------
\subsubsection{No Expression}

This section describes the special value, \node{NoExpression}.  This
node represents that an optional expression has not been specified.
This node may only replace a genuine expression node in those cases
where it is explicitly permitted.  This node does not correspond to
any source language construct.  

%--
\interface{void}{ExpNoexpression}{}{}{Expression}
\begin{functionality}
This routine pushes a special expression node onto the stack.  The
implementation is capable of distinguishing this special expression
node as not representing a valid expression.
\end{functionality}

%==============================================================================
\Subsection{Expression Ordering Operators}
The operators represented by routines in this section serve only order
and structure other expressions.  The functions of these operators are
independent of they type of their arguments.

\interface{void}{ExpSeries}{}{Expression e1, Expression e2}{Expression}
\begin{functionality}
This routine implementes C/C++'s comma operator.
\end{functionality}

\interface{void}{ExpParentheses}{}{Expression e}{Expression}
\begin{functionality}
This routine indicates that the programmer enclosed the associated
expression in parentheses.  The generation interface does not need
parentheses to override precendence rules.  However also specify the
order in which computations are performed.  This information may be
useful to optimizations which affect expression ordering.  
\end{functionality}

\partitle{Aggregation} 
Some languages allow instances of aggregate and array types to be
assigned to by aggrregate values.  These routines allow the
structuring of expressions into aggregate expressions.  Every
aggregate element has a position associated with it.  The generation
interface only supports positional specification of aggregate
elements.  User code must convert keyword specifications to positional
specifications.  

%--
\interface{void}{ExpPositionSingle}{}{Expression e}{Position}
\begin{functionality}
This routine builds a \node{Position} node that represents a single
position.  
\end{functionality}
%--
\interface{void}{ExpPositionRange}{}{Bounds b}{Position}
\begin{functionality}
This routine builds a \node{Position} node that represents a range of
positions.  
\end{functionality}
%--
\interface{void}{ExpPositionAny}{}{Bounds b}{Position}
\begin{functionality}
This routine builds a \node{Position} node that represents any
position.  This value corresponds to Ada's \key{others} construct in
an aggregate.  Any has lower priority than other positions; it fills
in those positions that no other aggregation element does.
\end{functionality}

%--
\interface{void}{ExpAggregationElement}{}
	{Position p, Expression e}{AggregationElement}
\begin{functionality}
This routine builds an element of an aggregation.  Each aggregation
element must specify its position in the final data type, but its
position may be a range or \emph{any}.
\end{functionality}

%--
\interface{void}{ExpAggregationBegin}{}{Type t}{Type t}
\interface{void}{ExpAggregationEnd}{}{Type t, \manyPops}{Expression}
\begin{functionality}
\befunc{ExpAggregation}{\node{AggregationElement}}  The resulting expression is
of type \args{t}.  Note that the type can be restricted to nodes of
\node{CompositeType} type.  
\end{functionality}

%==============================================================================
\Subsection{Assignment Operators}\label{sec:assignOp}

%The assignment operator works on all types.  For singleton types, it
%performs a simple copy, and for aggregate types, it does a component
%by component copy.  C/C++ does not allow arrays to be assigned as a
%whole, so the assignment operator uses Modula-3 semantics for array
%assignment.  

%--
\interface{void}{ExpAssignSimple}{}
	{Expression e1, Expression e2}{Expression}
\begin{functionality}
This routine works for all datatypes.  It performs a simple bit copy
of \code{e1} (rvalue) into \code{e2} (lvalue) for
\method{bitsize}\code{(e2)} bits.  This routine is best used for
singleton values, and the other assignment routines for aggregates and
arrays.  Note that for all assignment operators, we push the rvalue
before lvalue.
\end{functionality}
%--
\interface{void}{ExpAssignComponents}{}
	{Expression e1, Expression e2}{Expression}
\begin{functionality}
This routine does a component-wise copy for aggregates and arrays.
Hence, this routine is sensitive to the types of its arguments.  This
routine requires array arguments to be of the same shape and size and
will insert code to ensure run-time compliance, which is to say that
it follows Modula-3 semantics.
\end{functionality}
%--
\interface{void}{ExpAssignFixedString}{LengthFunction l}
	{Expression e1, Expression e2, Expression padding}{Expression}
\begin{functionality}
This routine provides string-like assignment for fixed sized arrays
(\eg static and open arrays).  This function allows
assignment of arrays of unequal lengths.  If the source expression is
longer than the target, only enough elements are copied to fill the
target.  If the target expression is longer than the source
expression, then the target expression is padded.
\begin{Parameters}
\Param{l} Indicates how to determine the the dynamic length of a
string.  The value is an element of an enumerated type:
\EnumOptions{LengthFunction}{cFixedLength, cTerminated}
\begin{Description}
\item [cFixedLength] Use the fixed length of \code{e1}.  This option is
for use with arrays with a fixed size at the point of assignment.  For
example, Modula-3 open arrays always have their length fixed before
any assignments may be done to the array.
\item [cTerminated] The array size is determined by an embedded
termination value, which is assumed to be zero.  This value would be
appropriate for C/C++, if they had this type of assignment.
\end{Description}
\Param{e1} Target of the assignment.
\Param{e2} Source value for the assignment.
\Param{padding} Value with which to pad \code{e1} if \code{e1} is
longer than \code{e2}. 
\end{Parameters}

\end{functionality}

%==============================================================================
\Subsection{Numeric Operators}
\oparg{Numeric}{integer, float, or fixed point} To provide support for
C/C++, the generation interface allows numeric operators to be applied
to pointers as well, in which case their bit patterns are
interpreted as integers.

Some routines which operate on numeric types behave differently
depending on its arguments' types.  The generation interface has
several options in how to discriminate between these behaviors.  One
solution is to provide provide a separate version of the routine for
each possible argument type, but this approach creates needless
additional routines.  Another solution is to give appropriate routines
a type parameter, but this approach provides redundant information to
the implementation, since it needs type information to handle
expressions (\eg declare temporaries).  A third solution is to require
implementations to extract type information from an operator's
arguments.  Though this approach requires implementations to extract
information user code already has, it is unlikely to create additional
work and greatly simplifies the interface.  Note, that the interface
requires user code to do complete type conversion; therefore, unless
otherwise stated both arguments must be of the same type.

Different languages have different semantics for overflow, underflow,
and divide-by-zero error conditions.  Both C/C++ and Modula-3 leave
error handling for these conditions implementation dependent, but Ada
requires them to be caught.  The routines representing operators with
possible error conditions accept an extra argument which indicates the
desired semantics.  Its value comes the following enumerated type:
\EnumOptions{ErrorHandling}{cImplementationDefined, cCatchError}
\begin{Description}
\item[cImplementationDefined] This value indicates that the
implementation is free to do as it chooses.
\item[cCatchError] This value indicates that these errors should be caught.
\end{Description}

\subsubsection{Arithmetic Operators}

\interface{void}{ExpPositive}{}{Expression e}{Expression}
\interface{void}{ExpNegative}{}{Expression e}{Expression}
\interface{void}{ExpAbsoluteValue}{}{Expression e}{Expression}
\interface{void}{ExpMinimum}{}{Expression e1, Expression e2}{Expression}
\interface{void}{ExpMaximum}{}{Expression e1, Expression e2}{Expression}

\interface{void}{ExpAddition}{ErrorHandling eh}
	{Expression e1, Expression e2}{Expression}
\interface{void}{ExpSubtraction}{ErrorHandling eh}
	{Expression e1, Expression e2}{Expression}
\interface{void}{ExpMultiplication}{ErrorHandling eh}
	{Expression e1, Expression e2}{Expression}
\interface{void}{ExpDivision}{ErrorHandling eh}
	{Expression e1, Expression e2}{Expression}
\interface{void}{ExpModulus}{ErrorHandling eh}
	{Expression e1, Expression e2}{Expression}
\interface{void}{ExpRemainder}{ErrorHandling eh}
	{Expression e1, Expression e2}{Expression}

\begin{functionality}
The generation interface defines remainder and modulus as follows:
\begin{itemize}
\item Remainder truncates towards zero, and the sign of
its result equals the sign of its right operand.
\item Modulus truncates towards negative infinity, and the sign of its
result equals the sign of its left operand. 
\end{itemize}
C/C++ (\%) and Modula-3's (\key{MOD}) modulus operators actually
implement the interface's remainder function.  The generation
interface's modulus supports Ada semantics.
\end{functionality}

\interface{void}{ExpExponentiation}{ErrorHandling eh}
	{Expression e1, Expression e2}{Expression}
\interface{void}{ExpPreDecrement}{ErrorHandling eh}
	{Expression e}{Expression}
\interface{void}{ExpPreIncrement}{ErrorHandling eh}
	{Expression e}{Expression}
\interface{void}{ExpPostDecrement}{ErrorHandling eh}
	{Expression e}{Expression}
\interface{void}{ExpPostIncrement}{ErrorHandling eh}
	{Expression e}{Expression}

\subsubsection{Relational Operators}
The operands to these operators are of Numeric type, but the resulting
value is of Boolean type.

\interface{void}{ExpEquality}{}{Expression e1, Expression e2}{Expression}
\interface{void}{ExpNotEqual}{}{Expression e1, Expression e2}{Expression}
\interface{void}{ExpGreater}{}{Expression e1, Expression e2}{Expression}
\interface{void}{ExpGreaterEqual}{}{Expression e1, Expression e2}{Expression}
\interface{void}{ExpLess}{}{Expression e1, Expression e2}{Expression}
\interface{void}{ExpLessEqual}{}{Expression e1, Expression e2}{Expression}

\subsubsection{Bitwise Operators}
Bit operators (see also Section~\ref{sec:compoundAssignOp}) interpret their
arguments as a bit pattern.  Hence, no special handling exists for
real types. 

%--
\interface{void}{ExpBitComplement}{}{Expression e}{Expression}
\begin{functionality}
\cxxOp
\end{functionality}
%--
\interface{void}{ExpBitAnd}{}{Expression e1, Expression e2}{Expression}
\begin{functionality}
\cxxOp
\end{functionality}
%--
\interface{void}{ExpBitXor}{}{Expression e1, Expression e2}{Expression}
\begin{functionality}
\cxxOp
\end{functionality}
%--
\interface{void}{ExpBitOr}{}{Expression e1, Expression e2}{Expression}
\begin{functionality}
\cxxOp
\end{functionality}
%--
\interface{void}{ExpBitShiftLeft}{}{Expression e1, Expression e2}{Expression}
\begin{functionality}
\cxxOp
\end{functionality}
%--
\interface{void}{ExpBitShiftRight}{}{Expression e1, Expression e2}{Expression}
\begin{functionality}
\cxxOp
\end{functionality}

%------------------------------------------------------------------------------
\subsubsection{Compound Assignment Operators}\label{sec:compoundAssignOp}
These operators perform two simpler operations in a single step.  The
second operation is a simple assignment.  

%--
\interface{void}{ExpMultiplicationAssignment}{ErrorHandling eh}
	{Expression e1, Expression e2}{Expression}
\begin{functionality}
\cxxOp
\end{functionality}
%--
\interface{void}{ExpDivisionAssignment}{ErrorHandling eh}
	{Expression e1, Expression e2}{Expression}
\begin{functionality}
\cxxOp
\end{functionality}
%--
\interface{void}{ExpRemainderAssignment}{ErrorHandling eh}
	{Expression e1, Expression e2}{Expression}
\begin{functionality}
\cxxOp
\end{functionality}
%--
\interface{void}{ExpAdditionAssignment}{ErrorHandling eh}
	{Expression e1, Expression e2}{Expression}
\begin{functionality}
\cxxOp
\end{functionality}
%--
\interface{void}{ExpSubtractionAssignment}{ErrorHandling eh}
	{Expression e1, Expression e2}{Expression}
\begin{functionality}
\cxxOp
\end{functionality}
%--
\interface{void}{ExpBitShiftRightAssignment}{}
	{Expression e1, Expression e2}{Expression}
\begin{functionality}
\cxxOp
\end{functionality}
%--
\interface{void}{ExpBitShiftLeftAssignment}{}
	{Expression e1, Expression e2}{Expression}
\begin{functionality}
\cxxOp
\end{functionality}
%--
\interface{void}{ExpBitAndAssignment}{}
	{Expression e1, Expression e2}{Expression}
\begin{functionality}
\cxxOp
\end{functionality}
%--
\interface{void}{ExpBitXorAssignment}{}
	{Expression e1, Expression e2}{Expression}
\begin{functionality}
\cxxOp
\end{functionality}
%--
\interface{void}{ExpBitOrAssignment}{}
{Expression e1, Expression e2}{Expression}
\begin{functionality}
\cxxOp
\end{functionality}

%==============================================================================
\subsection{Boolean Operators}\label{sec:logicOp}
\oparg{Boolean}{boolean} C/C++ specify that logical operators can
accept arguments of integral type but still return arguments of
logical type.  Since the generation interface does not permit logical
operators to have integral operands, user code must type convert
integral operands to boolean type.

Nevertheless, boolean values are considered to be an integer subrange,
so that Numeric \method{equality} and \method{inequality} apply to
boolean types.

\interface{void}{ExpTrue}{}{}{Expression}
\interface{void}{ExpFalse}{}{}{Expression}
\begin{functionality}
\method{True} and {false} act as through they are an enumerated type
with false equal to zero and true equal to one.  Hence, the type
conversion routines for enumerated types may be used on these values.
\end{functionality}

\interface{void}{ExpNot}{}{Expression e}{Expression}
\interface{void}{ExpAnd}{}{Expression e1, Expression e2}{Expression}
\interface{void}{ExpOr}{}{Expression e1, Expression e2}{Expression}
\interface{void}{ExpAndConditional}{}{Expression e1, Expression e2}{Expression}
\interface{void}{ExpOrConditional}{}{Expression e1, Expression e2}{Expression}
\begin{functionality}
The \method{*Conditional} forms implement short circuit semantics.
Hence, C and C++ should use these forms.
\end{functionality}

\interface{void}{ExpExpressionIf}{}{Expression e1, Expression e2, Expression e3}
	{Expression}
\begin{functionality}
\cxxOp
\end{functionality}

%==============================================================================
\Subsection{Pointer Operators}
\oparg{Pointer}{pointer}  For equality and inequality, pointers are
treated as bit patterns (\ie a Numeric type).

\interface{void}{ExpAddress}{}{Expression e}{Expression}
\interface{void}{ExpDereference}{}{Expression e}{Expression}
\interface{void}{ExpNil}{}{}{Expression}
\begin{functionality}
This routine generates a \emph{nil} pointer value.  Modula-3 has an
explicit nil value.  C++ specifies that when the integer value zero is
converted to a pointer it becomes the nil pointer value (regardless of
bit representation).
\end{functionality}

%==============================================================================
\Subsection{Aggregate Operators}
In each of the routines with parameters, the first argument is of
aggregate type, and the second argument is of field or routine type.

\interface{void}{ExpThis}{}{}{Expression}
\begin{functionality}
This routine returns a pointer to the current object.  It is only
valid inside of a method.
\end{functionality}

%--
\interface{void}{ExpSelect}{}{Expression e1, Expression e2}{Expression}
\begin{functionality}
\cxxOp
\end{functionality}
%--
\interface{void}{ExpSelectIndirect}{}{Expression e1, Expression e2}{Expression}
\begin{functionality}
\cxxOp
\end{functionality}
%--
\interface{void}{ExpSelectRelative}{}{Expression e1, Expression e2}{Expression}
\begin{functionality}
\cxxOp
\end{functionality}
%--
\interface{void}{ExpSelectRelativeIndirect}{}{Expression e1, Expression e2}
	{Expression}
\begin{functionality}
\cxxOp
\end{functionality}

%==============================================================================
\Subsection{Array Operators}
This section describes the operators available for manipulating
arrays.  Some array operators (\eg \method{assignment} and
\method{deleteArray}) may be found in other sections.

%------------------------------------------------------------------------------
\subsubsection{Subscripting}
Some source languages supply multiple dimension subscripts, while
others, such as C++, require repeated application of a one-dimensional
subscript operator.  The generation interface supports both
approaches.  Multiple dimension subscripts may be composed using the
\method{index} routines.  For single dimensional subscripts, either
of the \method{subscript} routines may be used.

\interface{void}{ExpIndexBegin}{}{}{}
\interface{void}{ExpIndexEnd}{}{\manyPops}{Indicies}
\begin{functionality}
\befunc{ExpIndex}{\node{Expression}}
\end{functionality}
\interface{void}{ExpSubscript}{bool boundsChecking}
	{Expression e, Indicies i}{Expression}
\begin{functionality}
This routine represents a subscript operation.  Some languages require
subscript bound checks while other languages do not.  The
\code{boundsChecking} parameter allows user code to select between
these two choices.
\end{functionality}

\interface{void}{ExpSubscript1d}{bool boundsChecking}
	{Expression e1, Expression e2}{Expression}
\begin{functionality}
This routine is a short-cut for when only one dimension is specified.
Some languages require subscript bound checks while other languages do
not.  The \code{boundsChecking} parameter allows user code to select
between these two choices.
\end{functionality}

%------------------------------------------------------------------------------
\subsubsection{String Array Operators}

%--
\interface{void}{ExpArrayEquality}{}{Expression e1, Expression e2}{Expression}
\begin{functionality}
This routine returns true if the two arrays (\code{e1} and \code{e2})
have the same size and their elements are equal.  
\end{functionality}

%--
\interface{void}{ExpArrayInequality}{}{Expression e1, Expression e2}{Expression}
\begin{functionality}
This routine is the complement of \method{arrayEquality}.  
\end{functionality}

%--
\interface{void}{ExpArrayGreater}{}{Expression e1, Expression e2}{Expression}
\begin{functionality}
\fortranOp
\end{functionality}
%--
\interface{void}{ExpArrayGreaterEqual}{}
	{Expression e1, Expression e2}{Expression}
\begin{functionality}
\fortranOp
\end{functionality}
%--
\interface{void}{ExpArrayLess}{}{Expression e1, Expression e2}{Expression}
\begin{functionality}
\fortranOp
\end{functionality}
%--
\interface{void}{ExpArrayLessEqual}{}
	{Expression e1, Expression e2}{Expression}
\begin{functionality}
\fortranOp
\end{functionality}
%--
\interface{void}{ExpSlice}{}
	{Expression s, Expression b, Expression e}{Expression}
\begin{functionality}
This routine extracts a contiguous section (\ie substring) of a one
dimensional array.  
\begin{Parameters}
\Param{s} The array to be operated on.
\Param{l} The beginning position of the substring.
\Param{e} The ending position of the substring.
\end{Parameters}
\end{functionality}
%--
\interface{void}{ExpRemainingSlice}{}{Expression s, Expression l}{Expression}
\begin{functionality}
This routine extracts the substring of elements from position \code{l}
to the end of the one dimensional array.
\begin{Parameters}
\Param{s} The array to be operated on.
\Param{l} The beginning position of the substring.
\end{Parameters}
\end{functionality}
%--
\interface{void}{ExpConcatenation}{}{Expression s1, Expression s2}{Expression}
\begin{functionality}
\fortranOp
Concatentation is performed into a temporary variable.
\end{functionality}

%==============================================================================
\Subsection{Set Operators}

%--
\interface{void}{ExpSetEquality}{}{Expression se}{Expression}
\begin{functionality}
\modulaOp
\end{functionality}
%--
\interface{void}{ExpUnion}{}{Expression se1, Expression se2}{Expression}
\begin{functionality}
\modulaOp
\end{functionality}
%--
\interface{void}{ExpDifference}{}{Expression se1, Expression se2}{Expression}
\begin{functionality}
\modulaOp
\end{functionality}
%--
\interface{void}{ExpIntersection}{}{Expression se1, Expression se1}{Expression}
\begin{functionality}
\modulaOp
\end{functionality}
%--
\interface{void}{ExpSymmetricDifference}{}{Expression se1, Expression se2}
	{Expression}
\begin{functionality}
\modulaOp
\end{functionality}
%--
\interface{void}{ExpSubset}{}{Expression se1, Expression se2}{Expression}
\begin{functionality}
\modulaOp
\end{functionality}
%--
\interface{void}{ExpSuperset}{}{Expression se1, Expression se2}{Expression}
\begin{functionality}
\modulaOp
\end{functionality}
%--
\interface{void}{ExpElement}{}{Expression e, Expression se}{Expression}
\begin{functionality}
\modulaOp
\end{functionality}

%==============================================================================
\Subsection{Call Operators}
Call operators represent calls to routines.  These calls may
be either function calls or method calls.

\interface{void}{ExpPositionalArgument}{}{Expression e}{Argument}
\interface{void}{ExpNamedArgument}{Identifier name}{Expression e}{Argument}
\begin{functionality}
Arguments may be either positional or named.  C++ uses only positional
arguments, but Modula-3 and Ada use both.  Note that
\method{positionalArgument} really only type converts from
\node{Expression} to \node{Argument}.  
\end{functionality}

%--
\interface{void}{ArgumentsBegin}{}{}{}
%--
\interface{void}{ArgumentsEnd}{}{\manyPops}{ArgumentList}
\begin{functionality}
\befunc{ExpArguments}{\node{Argument}}
\end{functionality}

\interface{void}{ExpCallFunction}{}{Expression function, ArgumentList al}
	{Expression}
\interface{void}{ExpCallMethod}{}{Expression object, Expression
	method, ArgumentList al}{Expression}
\begin{functionality}
This routine generates the representation for a method invocation.
The \code{object} parameter represents the object associate with
the method.  The \code{method} formal is the name of the method
being called (It should be an \node{IdReference} node).

\end{functionality}

%==============================================================================
\Subsection{Heap Operators}
The behavior of heap operators are affected by several compilation
unit attributes (see Section~\ref{sec:compilationUnit}).  The
allocation operators call constructors, if appropriate.  The
delete operators call destructors, if appropriate.  In addition,
implementations of the interface must be aware of whether or not
garbage collection is being used.

\interface{void}{ExpAllocate}{TypeID type}{}{Expression}
\interface{void}{ExpAllocateDefault}{TypeID type}{Expression e}
	{Expression}
\begin{functionality}
This routine is used when the allocation has a default value.
For a class object with an initializer, the expression \code{e} is a
function call to the class' initializer.
\end{functionality}
\interface{void}{ExpAllocatePlacement}{TypeID type}{Expression e,
	Expression p}{Expression}
\begin{functionality}
This routine allows both a default and a placement parameter (for C++).
The argument \code{p} is a placement list parameter.
\end{functionality}
\interface{void}{ExpAllocateOpenArray}{TypeID type}{Indicies i}
	{Expression}
\begin{functionality}
This routine allocates an open array (as found in Modula-3), and sets
the size of the array in each dimension.
\end{functionality}
\interface{void}{ExpAllocateSettingFields}{TypeID type}
	{ArgumentList al}{Expression}
\begin{functionality}
This routine allocates an aggregate entity and then uses the
positional arguments specified in \code{al} to initialize the entity.
\end{functionality}

\interface{void}{ExpDelete}{}{}{Expression e}
\interface{void}{ExpDeleteArray}{}{}{Expression e}
\begin{functionality}
This routine implements the semantics of C++'s delete array operator.
Note that we do not support C++'s archaic number of elements parameter
to the delete array operator.
\end{functionality}

%==============================================================================
\Subsection{Type Operators}

\subsubsection{Type Query Operators}
%--
\interface{void}{ExpBytesizeVariable}{}{Expression variable}{Expression}
\begin{functionality}
This routine implements the Modula-3 \key{bytesize} operator as well as the
C/C++ \key{sizeof} operator.
\end{functionality}
%--
\interface{void}{ExpBytesizeType}{TypeID type}{}{Expression}
\begin{functionality}
This routine implements the Modula-3 \key{bytesize} operator as well as the
C/C++ \key{sizeof} operator.
\end{functionality}
%--
\interface{void}{ExpBitsizeVariable}{}{Expression variable}{Expression}
\begin{functionality}
\modulaOp
\end{functionality}
%--
\interface{void}{ExpBitsizeType}{TypeID type}{}{Expression}
\begin{functionality}
\modulaOp
\end{functionality}
%--
\interface{void}{ExpAdrsizeVariable}{}{Expression variable}{Expression}
\begin{functionality}
\modulaOp
\end{functionality}
%--
\interface{void}{ExpAdrsizeType}{TypeID type}{}{Expression}
\begin{functionality}
\modulaOp
\end{functionality}

%--
\interface{void}{ExpIstype}{TypeID type}{Type t}{Expression}
\begin{functionality}
\modulaOp
\end{functionality}
%--
\interface{void}{ExpNarrow}{TypeID type}{Type t}{Expression}
\begin{functionality}
\modulaOp
\end{functionality}
%--
\interface{void}{ExpTypecode}{}{Expression e}{Expression}
\begin{functionality}
\modulaOp
\end{functionality}
%--
\interface{void}{ExpNumber}{}{Expression e}{Expression}
\begin{functionality}
\modulaOp
\end{functionality}
%--
\interface{void}{ExpFirst}{}{Expression e}{Expression}
\begin{functionality}
\modulaOp
\end{functionality}
%--
\interface{void}{ExpLast}{}{Expression e}{Expression}
\begin{functionality}
\modulaOp
\end{functionality}

%------------------------------------------------------------------------------
\subsubsection{Type Conversion Operators}

Type conversion rules vary substantially between languages.  Hence,
user code is responsible for ensuring that all implicit conversions
are made explicit.  

Some languages provide a fixed set of type conversions, but C++ allows
users to define conversion routines for classes.  Hence in the
generation interface, a type conversion is a triple consisting of the
expression to be converted, the type to which it is to be converted,
and a routine for performing the conversion.  For language defined
conversions, the generation interface provides an enumerated list of
recognized conversions.  The elements of the following enumeration
follow Modula-3 semantics:
\EnumOptions{ConversionRoutines}{cReal, cFloor, cCeiling, cRound,
	cTruncate, cOrdinal, cEnumerationValue, cLoophole, cCast, cComplex}

\interface{void}{ExpTypeConversion}{TypeID type, NameID routine}
	{Expression e}{Expression}
\interface{void}{ExpTypeConversion}{TypeID type, ConversionRoutines cr}
	{Expression e}{Expression}

\todo{
(*----------------------------------------------------------- expressions ---*)

(*  The code to evaluate expressions is generated by calling the
    procedures listed below.  Each procedure corresponds to an
    instruction for a simple stack machine.  Values in the stack
    have a type and a size.  Operations on the stack values are
    also typed.  Type mismatches may cause bad code to be generated
    or an error to be signaled.  Explicit type conversions must be used.

    Integer values on the stack, regardless of how they are loaded,
    are sign-extended to full-width values.  Similarly, word values
    on the stack are always zero-extened to full-width values.
    
    The expression stack must be empty at each label, jump, or call.
    The stack must contain exactly one value prior to a conditional
    or indexed jump.

    All addresses are byte addresses.  There is no boolean type;  boolean
    operators yield [0..1].

    Operations on word values are performed MOD the word size and are
    not checked for overflow.  Operations on integer values may or may not
    cause checked runtime errors depending on the particular code generator.

    The operators are declared below with a definition in terms of
    what they do to the execution stack.  For example,  ceiling(Reel)
    returns the ceiling, an integer, of the top value on the stack,
    a real:  s0.I := CEILING (s0.R).

    Unless otherwise indicated, operators have the same meaning as in
    the Modula-3 report.
*)

(*-------------------------------------------------------------- literals ---*)

load_nil     ();                         (*push; s0.A := NIL*)
load_integer (READONLY i: Target.Int);   (*push; s0.I := i *)
load_float   (READONLY f: Target.Float); (*push; s0.t := f *)

(*------------------------------------------------------------ arithmetic ---*)

(* when any of these operators is passed t=Type.Word, the operator
   does the unsigned comparison or arithmetic, but the operands
   and the result are of type Integer *)
   
eq       (t: ZType);   (* s1.I := (s1.t = s0.t); pop *)
ne       (t: ZType);   (* s1.I := (s1.t # s0.t); pop *)
gt       (t: ZType);   (* s1.I := (s1.t > s0.t); pop *)
ge       (t: ZType);   (* s1.I := (s1.t >= s0.t); pop *)
lt       (t: ZType);   (* s1.I := (s1.t < s0.t); pop *)
le       (t: ZType);   (* s1.I := (s1.t <= s0.t); pop *)
add      (t: AType);   (* s1.t := s1.t + s0.t; pop *)
subtract (t: AType);   (* s1.t := s1.t - s0.t; pop *)
multiply (t: AType);   (* s1.t := s1.t * s0.t; pop *)
divide   (t: RType);   (* s1.t := s1.t / s0.t; pop *)
negate   (t: AType);   (* s0.t := - s0.t *)
abs      (t: AType);   (* s0.t := ABS (s0.t) (noop on Words) *)
max      (t: ZType);   (* s1.t := MAX (s1.t, s0.t); pop *)
min      (t: ZType);   (* s1.t := MIN (s1.t, s0.t); pop *)
round    (t: RType);   (* s0.I := ROUND (s0.t) *)
trunc    (t: RType);   (* s0.I := TRUNC (s0.t) *)
floor    (t: RType);   (* s0.I := FLOOR (s0.t) *)
ceiling  (t: RType);   (* s0.I := CEILING (s0.t) *)
cvt_float(t: AType;  u: RType);     (* s0.u := FLOAT (s0.t, u) *)
div      (t: IType;  a, b: Sign);   (* s1.t := s1.t DIV s0.t;pop*)
mod      (t: IType;  a, b: Sign);   (* s1.t := s1.t MOD s0.t;pop*)

(* In div and mod, "a" is the sign of s1 and "b" is the sign of s0. *)

(*------------------------------------------------------------------ sets ---*)

(* Sets are represented on the stack as addresses. *)

set_union (s: ByteSize);          (* s2.B := s1.B + s0.B; pop(3) *)
set_difference (s: ByteSize);     (* s2.B := s1.B - s0.B; pop(3) *)
set_intersection (s: ByteSize);   (* s2.B := s1.B * s0.B; pop(3) *)
set_sym_difference (s: ByteSize); (* s2.B := s1.B / s0.B; pop(3) *)
set_member (s: ByteSize);         (* s1.I := (s0.I IN s1.B); pop *)
set_eq (s: ByteSize);             (* s1.I := (s1.B = s0.B); pop *)
set_ne (s: ByteSize);             (* s1.I := (s1.B # s0.B); pop *)
set_lt (s: ByteSize);             (* s1.I := (s1.B < s0.B); pop *)
set_le (s: ByteSize);             (* s1.I := (s1.B <= s0.B); pop *)
set_gt (s: ByteSize);             (* s1.I := (s1.B > s0.B); pop *)
set_ge (s: ByteSize);             (* s1.I := (s1.B >= s0.B); pop *)
set_range (s: ByteSize);          (* s2.A[s1.I..s0.I] := 1; pop(3) *)
set_singleton (s: ByteSize);      (* s1.A [s0.I] := 1; pop(2) *)

(*------------------------------------------------- Word.T bit operations ---*)

not ();  (* s0.I := Word.Not (s0.I) *)
and ();  (* s1.I := Word.And (s1.I, s0.I); pop *)
or  ();  (* s1.I := Word.Or  (s1.I, s0.I); pop *)
xor ();  (* s1.I := Word.Xor (s1.I, s0.I); pop *)

shift        ();  (* s1.I := Word.Shift  (s1.I, s0.I); pop *)
shift_left   ();  (* s1.I := Word.Shift  (s1.I, s0.I); pop *)
shift_right  ();  (* s1.I := Word.Shift  (s1.I, -s0.I); pop *)
rotate       ();  (* s1.I := Word.Rotate (s1.I, s0.I); pop *)
rotate_left  ();  (* s1.I := Word.Rotate (s1.I, s0.I); pop *)
rotate_right ();  (* s1.I := Word.Rotate (s1.I, -s0.I); pop *)

extract (sign: BOOLEAN);
(* s2.I := Word.Extract(s2.I, s1.I, s0.I);
   IF sign THEN SignExtend s2; pop(2) *)

extract_n (sign: BOOLEAN;  n: INTEGER);
(* s1.I := Word.Extract(s1.I, s0.I, n);
   IF sign THEN SignExtend s1; pop(1) *)

extract_mn (sign: BOOLEAN;  m, n: INTEGER);
(* s0.I := Word.Extract(s0.I, m, n);
   IF sign THEN SignExtend s0 *)

extract_typecode (equiv: Var);
(* s0.I := typecode of object pointed to by S0 *)

istype (type: TypeUID; t, f: Label);

insert ();
(* s3.I := Word.Insert (s3.I, s2.I, s1.I, s0.I); pop(3) *)

insert_n (n: INTEGER);
(* s2.I := Word.Insert (s2.I, s1.I, s0.I, n); pop(2) *)

insert_mn (m, n: INTEGER);
(* s1.I := Word.Insert (s1.I, s0.I, m, n); pop(1) *)

(*------------------------------------------------ misc. stack/memory ops ---*)

swap (a, b: Type);
(* tmp := s1.a; s1.b := s0.b; s0.a := tmp *)

pop (t: Type);
(* pop(1) discard s0, not its side effects *)

copy_n (t: MType;  overlap: BOOLEAN);
(* copy s0.I units with 't's size and alignment from s1.A to s2.A; pop(3).
   'overlap' is true if the source and destination may partially overlap
   (ie. you need memmove, not just memcpy). *)

copy (n: INTEGER;  t: MType;  overlap: BOOLEAN);
(* copy 'n' units with 't's size and alignment from s0.A to s1.A; pop(2).
   'overlap' is true if the source and destination may partially overlap
   (ie. you need memmove, not just memcpy). *)

zero_n (t: MType);
(* zero s0.I units with 't's size and alignment starting at s1.A; pop(2) *)

zero (n: INTEGER;  t: MType);
(* zero 'n' units with 't's size and alignment starting at s0.A; pop(1) *)

(*----------------------------------------------------------- conversions ---*)

loophole (from, two: ZType);
(* s0.two := LOOPHOLE(s0.from, two) *)

(*---------------------------------------------------- address arithmetic ---*)

add_offset (i: INTEGER);
(* s0.A := s0.A + i bytes *)

index_address (size: INTEGER);
(* s1.A := s1.A + s0.I * size; pop  -- where 'size' is in bytes *)

(*------------------------------------------------------- procedure calls ---*)

(* To generate a direct procedure call:
    
      cg.start_call_direct (proc, level, t);
    
      for each actual parameter i
          <generate value for parameter i>
          cg.pop_param ();  -or-  cg.pop_struct();
        
      cg.call_direct (proc, t);

   or to generate an indirect call:

      cg.start_call_indirect (t);
    
      for each actual parameter i
          <generate value for parameter i>
          cg.pop_param ();  -or-  cg.pop_struct();

      If the target is a nested procedure,
          <evaluate the static link to be used>
          cg.pop_static_link ();

      <evaluate the address of the procedure to call>
      cg.call_indirect (t);
*)

start_call_direct (p: Proc;  lev: INTEGER;  t: Type);
(* begin a procedure call to procedure 'p' at static level 'lev' that
   will return a value of type 't'. *)

call_direct (p: Proc;  t: Type);
(* call the procedure 'p'.  It returns a value of type t. *)

start_call_indirect (t: Type;  cc: CallingConvention);
(* begin an indirect procedure call that will return a value of type 't'. *)

call_indirect (t: Type;  cc: CallingConvention);
(* call the procedure whose address is in s0.A and pop s0.  The
   procedure returns a value of type t. *)

(* Allocate something of type alloc_type *)
start_alloc_call (t: Type; alloc_type: TypeUID);

(* l_to_r = args are passed from left to right *)
method_expr (holder, basetype: TypeUID; method: Name);

call_closure (proc: Var; t: Type);
(* Call closure s0.  Pop s0 *)

pop_param (t: ZType);
(* pop s0.t and make it the "next" parameter in the current call. *)

pop_struct (s: ByteSize;  a: Alignment);
(* pop s0.A, it's a pointer to a structure occupying 's' bytes that's
  'a' byte aligned; pass the structure by value as the "next" parameter
  in the current call. *)

pop_static_link ();
(* pop s0.A for the current indirect procedure call's static link  *)

(*------------------------------------------- procedure and closure types ---*)

load_procedure (p: Proc);
(* push; s0.A := ADDR (p's body) *)

load_static_link (p: Proc);
(* push; s0.A := (static link needed to call p, NIL for top-level procs) *)
}

% -*- Mode: latex; Mode: auto-fill; -*-

\Section{Compilation Units}\label{sec:compilationUnit}

All compilations begin with a compilation unit.  In terms of an
abstract syntax tree, a compilation unit is the root of the tree.  
Compilation units collect together information from outside the source
program, which generally means source language information.

The generation interface requires that user code specify the source
language of each compilation unit.  Implementations may use this
information to define language specific constants, types, routines,
etc.  Moreover, the source language information is used to preset
attributes of the compilation unit.  Programmers use the following
enumeration to specify source languages:

\begin{center}
\EnumOptions{LanguageId}{cLangC, cLangCxx, cLangModula3, cLangFortran77}
\end{center}

%\Subsection{Compilation Units}
%--
\interface{void}{CompilationUnitBegin}{LanguageId l}{}{CompilationUnit}
%--
\interface{void}{CompilationUnitEnd}{}{\manyPops}{CompilationUnit}
\begin{functionality}
\befunc{CompilationUnit}{UnitDecl}  Note that both begin and end
routines return a \node{CompilationUnit} node, and this node is the
same.  The begin routine returns the partially complete node so that
attributes may be associated with the node.
\end{functionality}

%\Subsection{Setting Compilation Unit Attributes}\label{sec:unitAttributes}

%--
\interface{void}{SetIdentifierCase}{IdentifierCase ic}
	{CompilationUnit cu}{CompilationUnit}
\begin{functionality}
Implemenations of the generation interface are required to handle
both case sensitive and insensitive identifiers.  The default is case
sensitive.  We could have required the front end to homogenize case
for case insensitive languages; however, this information may prove
vital to debuggers.
\EnumOptions{IdentifierCase}{cSensitive, cInsensitive}
\end{functionality}

%--
\interface{void}{SetMemoryManagement}{MemoryManagement mm}
	{CompilationUnit cu}{CompilationUnit}
\begin{functionality}
This routine specifies how dynamic memory is managed by the source language.
\EnumOptions{MemoryManagement}{cUserManaged, cGarbageCollected}
\end{functionality}

%--
\interface{void}{SetRecordFieldOrderRule}{bool orderMatters}
	{CompilationUnit cu}{CompilationUnit}
\begin{functionality}
This routine specifies whether or not the source language uses the
order of fields to distinguish record types.  The default value is
true, order does matter.
\end{functionality}

%--
\interface{void}{SetClassFieldOrderRule}{bool orderMatters}
	{CompilationUnit cu}{CompilationUnit}
\begin{functionality}
This routine specifies whether or not the source language uses the
order of fields to distinguish class types.  The default value is
true, order does matter.
\end{functionality}

%--
\interface{void}{SetMethodsRule}{bool methodsMatter}
	{CompilationUnit cu}{CompilationUnit}
\begin{functionality}
This routine specifies whether or not methods are used to distinguish
class types.  The default value is true, methods do matter.
\end{functionality}






\todo{
(*----------------------------------------------------- compilation units ---*)

begin_unit (optimize: INTEGER := 0);
(* called before any other method to initialize the compilation unit. *)

end_unit ();
(* called after all other methods to finalize the unit and write the
   resulting object.  *)

import_unit (n: Name);
export_unit (n: Name);
(* note that the current compilation unit imports/exports the interface 'n' *)
}

% -*- Mode: latex; Mode: auto-fill; -*-
\Section{Annotations}\label{sec:annote}

The generation interface allows user code to pass information that
does not directly correspond to the structure of the program.  We
refer to this information as annotations.  The interface currently
provides two kinds of annotations: point annotations and range
annotations.  A \emph{point annotation} applies to a specific piece of
the program (\eg an expression or function).  A \emph{range
annotation} may apply to several pieces of the program and is marked
with a begin/end pair.  

\Subsection{Point Annotations}
A point annotation applies to the top element of the stack.  A point
annotation may modify the top element of the stack but does not
(permanently) remove it.

\interface{void}{SetSourceLine}{int line}{}{Node}

\Subsection{Range Annotations}
A range annotation applies to all elements pushed onto the stack after
the begin routine and before the corresponding end routine.

\interface{void}{SourceFileBegin}{String filename}{}{}
\interface{void}{SourceFileEnd}{String filename}{\manyPops}{}
\begin{functionality}
Strictly speaking, the file name would not have to be specified at
both the beginning and end, but doing so may help in debugging.
\end{functionality}






