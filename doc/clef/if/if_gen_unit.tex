% -*- Mode: latex; Mode: auto-fill; -*-

\Section{Compilation Units}\label{sec:compilationUnit}

All compilations begin with a compilation unit.  In terms of an
abstract syntax tree, a compilation unit is the root of the tree.  
Compilation units collect together information from outside the source
program, which generally means source language information.

The generation interface requires that user code specify the source
language of each compilation unit.  Implementations may use this
information to define language specific constants, types, routines,
etc.  Moreover, the source language information is used to preset
attributes of the compilation unit.  Programmers use the following
enumeration to specify source languages:

\begin{center}
\EnumOptions{LanguageId}{cLangC, cLangCxx, cLangModula3, cLangFortran77}
\end{center}

%\Subsection{Compilation Units}
%--
\interface{void}{CompilationUnitBegin}{LanguageId l}{}{CompilationUnit}
%--
\interface{void}{CompilationUnitEnd}{}{\manyPops}{CompilationUnit}
\begin{functionality}
\befunc{CompilationUnit}{UnitDecl}  Note that both begin and end
routines return a \node{CompilationUnit} node, and this node is the
same.  The begin routine returns the partially complete node so that
attributes may be associated with the node.
\end{functionality}

%\Subsection{Setting Compilation Unit Attributes}\label{sec:unitAttributes}

%--
\interface{void}{SetIdentifierCase}{IdentifierCase ic}
	{CompilationUnit cu}{CompilationUnit}
\begin{functionality}
Implemenations of the generation interface are required to handle
both case sensitive and insensitive identifiers.  The default is case
sensitive.  We could have required the front end to homogenize case
for case insensitive languages; however, this information may prove
vital to debuggers.
\EnumOptions{IdentifierCase}{cSensitive, cInsensitive}
\end{functionality}

%--
\interface{void}{SetMemoryManagement}{MemoryManagement mm}
	{CompilationUnit cu}{CompilationUnit}
\begin{functionality}
This routine specifies how dynamic memory is managed by the source language.
\EnumOptions{MemoryManagement}{cUserManaged, cGarbageCollected}
\end{functionality}

%--
\interface{void}{SetRecordFieldOrderRule}{bool orderMatters}
	{CompilationUnit cu}{CompilationUnit}
\begin{functionality}
This routine specifies whether or not the source language uses the
order of fields to distinguish record types.  The default value is
true, order does matter.
\end{functionality}

%--
\interface{void}{SetClassFieldOrderRule}{bool orderMatters}
	{CompilationUnit cu}{CompilationUnit}
\begin{functionality}
This routine specifies whether or not the source language uses the
order of fields to distinguish class types.  The default value is
true, order does matter.
\end{functionality}

%--
\interface{void}{SetMethodsRule}{bool methodsMatter}
	{CompilationUnit cu}{CompilationUnit}
\begin{functionality}
This routine specifies whether or not methods are used to distinguish
class types.  The default value is true, methods do matter.
\end{functionality}






\todo{
(*----------------------------------------------------- compilation units ---*)

begin_unit (optimize: INTEGER := 0);
(* called before any other method to initialize the compilation unit. *)

end_unit ();
(* called after all other methods to finalize the unit and write the
   resulting object.  *)

import_unit (n: Name);
export_unit (n: Name);
(* note that the current compilation unit imports/exports the interface 'n' *)
}