% -*- Mode: latex; Mode: auto-fill; -*-
\Section{Source Language Specific Comments}\label{sec:lang}

This section records suggestions of how to translate specific source
languages using the generation interface.  The generation interface
directly reflects the syntax and semantics of C/C++ and Modula-3, so
these languages are likely to have fewer comments.  

%==============================================================================
\Subsection{Modula-3}

\begin{itemize}
\item inc/dec 

Modula-3's \key{inc} and \key{dec} statements are not
directly supported in the generation interface.  User code should
represent them as \method{eval}ed expressions.

\item assignment 

In Modula-3 assignment is a statement, but the
generation interface provides only an expression form of assignment.
User code should represent an assignment statement as an
\method{eval}ed assignment expression.

\item Refany and Address

Refany is represented as an unattributed pointer to void type.
Address is represented as a pointer to void type with the untraced
attribute.

\item Safe modules

The language parser and user code is responsible for checking whether
or not modules are safe.  This information is not passed through the
generation interface.

\end{itemize}

%==============================================================================
\Subsection{Fortran~77}

This section consists of a list of issues and our recommendations for
handling them.

\begin{itemize}
\item Generic functions

Fortran~77 allows a small amount of function overloading by permitting
library writers to create generic functions.  The identification of
generic functions is considered a front end issue, so the generation
interface only provides specific function names.

\item Declarations

Declarations in Fortran~77 are distributed across several statements
rather than collected together as expected by the generation interface.  
Basically, user code is responsible for collecting the distributed
information together and making the appropriate generation interface
call.  The following lists some of the declaration statements that
must be transformed:
\begin{itemize}
\item Implicit
\end{itemize}

\item Strings

User code should transform Fortran~77 strings into character arrays.

\item Intrinsic Functions

Some intrinsic functions are mapped to primitive operators, while
others will have to be implemented in a library.  See
Tables~\ref{tab:ops1}--\ref{tab:ops2}.  
[<
\item At least once \key{do} loops.

Some older version of Fortran have \emph{at least once} semantics for
their \key{do} loops.  The generation interface does not directly
provide these semantics, so user code will have to build equivalent
code from the available routines.

\end{itemize}

%==============================================================================
\Subsection{C++}

\begin{itemize}
\item Declarations in conditions and for loop initialization

C++ allows programmers to declare variables in conditions (\ie the
test in loops and in conditional statements).  The value of the
variable comes from its initialization.
\end{itemize}
