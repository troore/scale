% -*- Mode: latex; Mode: auto-fill; -*-
% This file has a bunch of macros I've written to help format the Clef book.

% Paragraph titles are convenient for minor titles.  They are not numbered.
\newcommand{\PARTITLE}[1]{\par\vspace{1ex}\noindent{\textit{ #1}}\\*\noindent}
\newcommand{\partitle}[1]{\PARTITLE{#1}} % Same as capitalized version
\newcommand{\subtitle}[1]{\item[]\hspace{-2ex}\emph{#1}}
\newcommand{\eg}{\textit{e.g.},\ }
\newcommand{\ie}{\textit{i.e.},\ }
\newcommand{\etal}{\textit{et al.}}
\newcommand{\clef}{Clef}
\newcommand{\tbs}{{\bfseries\itshape T.B.S}}
\newcommand{\manyPops}{$*$}

% These macros are for hilighting text in a consistent manner.
\newcommand{\key }[1]{\textit{#1}}
\newcommand{\args}[1]{\textit{#1}}
\newcommand{\ret }[1]{\textit{#1}}
\newcommand{\code}[1]{\texttt{#1}}
\newcommand{\node}[1]{\textsf{#1}}
\newcommand{\method}[1]{\textbf{#1}}
\newcommand{\member}[1]{\textbf{#1}}
\newcommand{\module}[1]{\textsc{#1}}

% For operator table:
\newcommand{\tabhead}[1]{\textit{#1}}
\newcommand{\na}{\textsl{N/A}}

% These macros control the appearance of each routine description.
\newcommand{\interface}[5]{
\par\vspace{1ex}\noindent
  \begin{samepage}
  %\hspace{-1ex}Interface Method:
  \textsl{#1} \textbf{#2}\index{#2} \textsl{(#3)} $\succ$ 
	$\lfloor$\textsf{#4}$\rfloor$ $\Rightarrow$ 
	$\lfloor$\textsf{#5\index{#5}}$\rfloor$\par
  \end{samepage}
}
\newcommand{\interfaceNS}[3]{
\par\vspace{1ex}\noindent
  \begin{samepage}
  %\hspace{-1ex}Interface Method:
  \textsl{#1} \textbf{#2}\index{#2} \textsl{(#3)}\par
  \end{samepage}
}

% This macro allows authors to leave short notes to themselves.
%\newcommand{\question}[1]{}
\newcommand{\question}[1]{
\begin{center}
  \boxput*(-0.75,1)
    {\psframebox*{Question!}}
    {\psshadowbox[framesep=12pt]{\parbox{6.0in}{#1}}}
\end{center}
}

% This macro standardizes the text for a begin/end routine pair.
\newcommand{\befunc}[2]{
  At the point that the \method{#1End} routine is called, the top of
  the stack (\ie the portion added since the \method{#1Begin} call)
  should contain only #2 nodes.  
}
% This macro standardizes the text for specifying the type of an
% operator's argument.
\newcommand{\oparg}[2]{
  Operands for #1 operators must be of type #2.  
}

% This macro standardizes the text for indicating that a routine
% implements a particular language's semantics.
\newcommand{\modulaOp}{
  This routine implements the corresponding Modula-3 operator (see
Table~\ref{tab:ops1}).
}
\newcommand{\modulaSemantics}[1]{
  This routine implements the semantics of Modula-3's \key{#1} construct.
}
\newcommand{\cxxOp}{
  This routine implements the corresponding C++ operator (see
Table~\ref{tab:ops1}, \ref{tab:ops2}, or \ref{tab:ops3}).
}
\newcommand{\cxxSemantics}[1]{
  This routine implements the semantics of C++'s \key{#1} construct.
}
\newcommand{\fortranOp}{
  This routine implements the corresponding Fortran~77 operator (see
Table~\ref{tab:ops1}).
}
\newcommand{\fortranSemantics}[1]{
  This routine implements the semantics of Fortran~77's \key{#1} construct.
}

% This macro simply comments out text.
\newcommand{\todo}[1]{
\par\noindent
%\begin{verbatim}
%#1
%\end{verbatim}
}

% This macro standardizes the appearance of enumeration lists.
\newcommand{\EnumOptions}[2]{
  \begin{center}
  \code{enum #1 \{#2\};}
  \end{center}
}

% These macros standarize the appearance of sections.
\newcommand{\Section}[1]{\newpage\section{#1}}
\newcommand{\Subsection}[1]{
  \begin{samepage}
    \setlength{\epsfxsize}{\textwidth}
    \vspace{3.0ex}\noindent
    \epsfbox{clefbar.eps}
    \nopagebreak\vspace{-5.0ex}\nopagebreak
    \subsection{#1}
  \end{samepage}}
\setcounter{secnumdepth}{5}  % Number subsections.
\setcounter{tocdepth}{3}     % Include subsubsections in TOC.
\newcommand{\Paragraph}[1]{\noindent\paragraph{#1}\par}

% This macro standardizes the appearance of descriptive text.
\newenvironment{functionality}
               {\list{}{\rightmargin 0em
			\listparindent \parindent}%
                \item[]\hspace{\parindent}}
               {\endlist}

%\newlength{\argwidth}
%\setlength{\argwidth}{4.5in}

% This macro standardizes the appearance of text describing a routine's
% parameters. 
\newenvironment{Parameters}
               {\begin{itemize}\item[]\textit{Parameters}:\begin{description}}
               {\end{description}\end{itemize}}
\newcommand{\Param}[1]{\item[\code{#1}]}

% This macro standardizes the appearance of a describion list in a
% functionality block.
\newenvironment{Description}
               {\begin{itemize}\item[]\begin{description}}
               {\end{description}\end{itemize}}

% These macros standardize the appearance of exception code lists.
\newenvironment{ErrorList}
               {\begin{itemize}\item[]\textit{Possible Exceptions}:\begin{description}}
               {\end{description}\end{itemize}}
%               {\begin{itemize}\item[]\begin{description}}
\newcommand{\ErrorItem}[1]{\item[#1]}

% These macros standarize the appearance of Clef node definitions.
\newenvironment{ClassDef}[2]
               {\begin{itemize}\item[]\key{class} \node{#1} : \node{#2} \{\
			\begin{description}}
               {\item[]\hspace{0.5in}$\vdots$\ \ \emph{\small Methods omitted.}\
		\end{description}\item[]\}\end{itemize}}
\newcommand{\MethodItem}[3]{\item[] \ret{#1} \method{#2} \args{(#3)};}
\newcommand{\MemberItem}[2]{\item[] \ret{#1} \member{#2};}

% This command designed to work with the environments I created.  
% It creates an unnumbered heading within an environment.

\newcommand{\NodePicture}[1]{
	\begin{samepage}
	\begin{center}
	\epsfbox{nodes/pictures/node_#1.eps}\\
	{\small The \node{#1} node and its children.}
	\end{center}
	\end{samepage}
}
