% -*- Mode: latex; Mode: auto-fill; -*-

\chapter{Introduction}\label{chap:intro}
\pagenumbering{arabic}  % Switch to arabic numerals for page numbering.

Scale is a compiler \emph{framework} which simplifies the task of
developing a compiler for new hardware.  Scale is not a compiler in
the sense that its objective is not to translate a particular language
or to generate object code for a particular machine.  Rather it is a
\emph{flexible} framework which provides all the components of a
compiler along with modules that facilitate the assembling of these
components into a complete compiler for a particular language and
machine pair.

Scale accepts multiple source languages and generates code for
multiple target architectures.  Accepting multiple source languages
largely consists of supplying a common notation for expressing the
semantices of each language.  Similarly, to generate object code for
different machines, Scale must have a common notation for expressing
the capabilities of each target.  This common notation for describing
hardware also permits Scale to generate object code for multiple
machines \emph{simultaneously}.  This ability is the base features
required by a compiler for heterogeneous systems.

Firgure~\ref{fig:scaleFlow} shows the flow of a program through the
initial phases of the Scale compiler.  In its initial design, Scale
supports the translation of imperative and object-oriented programming
languages (\ie C, C++, Modula-3, Fortran~77, and Ada-83).  Scale uses
the Edison Design Group (EDG) front ends for C and C++
(\module{edgcpfe}) and Fortran (\module{edgffe}) to handle parsing and
semantic error checking.  The EDG front ends pass a proprietary AST to
\module{Edg2Clef} which uses the \emph{generation interface} to 
translate it to the common high-level representation form used by
Scale.  For Modula-3, Scale uses the Digital Equipment Corporation
(DEC) Systems Research Center (SRC) Modula-3 front end to generate an
AST representation.  \module{M32Clef} reads the Modula-3 AST and uses
the generation interface to convert it to Scale's high-level
representation.

\begin{figure}
  \centering
  %\setlength{\epsfxsize}{\textwidth}
  \epsfbox{scaleFlow.eps}
\caption{\label{fig:scaleFlow} The Scale compiler framework.}
\end{figure}

The generation interface is a set of routines for constructing
high-level program representations such as ASTs.  The interface itself
does not determine the structure of the generated interface; however,
the Scale's initial design provides only one implementation of the
interface: \module{ClefBulder}.  \module{ClefBuilder} constructs a
\clef\ (Common Language Encoding Form) representation.  \clef\ is a
high-level AST representation which provides representations for all
the features found in source languages supported by Scale.  \clef\ may
be used for certain program transformation and annotation.  \clef\ also
provides a common representation from which the Composer module can
build the Score representation.  Score is a lower level representation
than \clef\ and is the form on which Scale performs most of its program
transformations.

This book describes components of the Scale compiler framework that
relate to the \clef\ representation.  Chapter~\ref{chap:genIf}
describes the generation interface and lists all of its routines.
Chapter~\ref{chap:nodes} presents the \clef\ representation node set,
and Chapter~\ref{chap:build} documents the design of
\module{ClefBuilder}.  For information on the EDG and SRC
representations see their respective documentation.  Score is
discussed in more detail in \cite{hcw96}.



